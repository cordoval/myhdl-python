\section{Introduction\label{conv-intro}}

\myhdl\ 0.4 supports the automatic conversion of a subset of \myhdl\ code
to synthesizable Verilog code. This feature provides a direct path
from Python to an FPGA or ASIC implementation.

\myhdl\ aims to be a complete design language, for tasks such as high
level modeling and verification, but also for implementation.
However, prior to 0.4 a user had to translate \myhdl\ code manually to
Verilog or VHDL. Needless to say, this was inconvenient. With \myhdl\
0.4, this manual step is no longer necessary.

\section{Solution description\label{conv-solution}}

The solution works as follows. The hardware description should be
modeled in \myhdl\ style, and satisfy certain constraints that are
typical for implementation-oriented hardware modeling.  Subsequently,
such a design is converted to an equivalent model in the Verilog
language, using the function \function{toVerilog} from the \myhdl\
library. Finally, a third-party \emph{synthesis tool} is used to
convert the Verilog design to a gate implementation for an ASIC or
FPGA. There are a number of Verilog synthesis tools available, varying
in price, capabilities, and target implementation technology.

The conversion does not start from source files, but from a design
that has been \emph{elaborated} by the Python interpreter. The
converter uses the Python profiler to track the interpreter's
operation and to infer the design structure and name spaces. It then
selectively compiles pieces of source code for additional analysis and
for conversion. This is done using the Python compiler package.

\section{Features\label{conv-features}}

\subsection{The design is converted after elaboration\label{conv-features-elab}}
\emph{Elaboration} refers to the initial processing of a hardware
description to achieve a representation of a design instance that is
ready for simulation or synthesis. In particular, structural
parameters and constructs are processed in this step. In \myhdl{}, the
Python interpreter itself is used for elaboration.  A
\class{Simulation} object is constructed with elaborated design
instances as arguments.  Likewise, the Verilog conversion works on an
elaborated design instance. The Python interpreter is thus used as
much as possible.

\subsection{The structural description can be arbitrarily complex and hierarchical\label{conv-features-struc}}
As the conversion works on an elaborated design instance, any modeling
constraints only apply to the leaf elements of the design structure,
that is, the co-operating generators. In other words, there are no
restrictions on the description of the design structure: Python's full
power can be used for that purpose. Also, the design hierarchy can be
arbitrarily deep.

\subsection{Generators are mapped to Verilog always or initial blocks\label{conv-features-gen}}
The converter analyzes the code of each generator and maps it
to a Verilog \code{always} blocks if possible, and to 
an \code{initial} block otherwise.
The converted Verilog design will be a flat
"net list of blocks".

\subsection{The Verilog module interface is inferred from signal usage\label{conv-features-intf}}
In \myhdl{}, the input or output direction of interface signals
is not explicitly declared. The converter investigates signal usage
in the design hierarchy to infer whether a signal is used as
input, output, or as an internal signal. Internal signals are
given a hierarchical name in the Verilog output.

\subsection{Function calls are mapped to a unique Verilog function or task call\label{conv-features-func}}
The converter analyzes function calls and function code to see if they
should be mapped to Verilog functions or to tasks. Python functions
are much more powerful than Verilog subprograms; for example, they are
inherently generic, and they can be called with named association.  To
support this power in Verilog, a unique Verilog function or task is
generated per Python function call.

\subsection{If-then-else structures may be mapped to Verilog case statements\label{conv-features-if}}
Python does not provide a case statement. However, 
the converter recognizes if-then-else structures in which a variable is
sequentially compared to items of an enumeration type, and maps
such a structure to a Verilog case statement with the appropriate
synthesis attributes.

\subsection{Choice of encoding schemes for enumeration types\label{conv-features-enum}}
The \function{enum} function in \myhdl\ returns an enumeration type. This
function takes an additional parameter \var{encoding} that specifies the
desired encoding in the implementation: binary, one hot, or one cold.
The Verilog converter generates the appropriate code.


\section{The convertible subset\label{conv-subset}}

\subsection{Introduction\label{conv-subset-intro}}

Unsurprisingly, not all \myhdl\ code can be converted to Verilog. In
fact, there are very important restrictions.  As the goal of the
conversion functionality is implementation, this should not be a big
issue: anyone familiar with synthesis is used to similar restrictions
in the \emph{synthesizable subset} of Verilog and VHDL. The converter
attempts to issue clear error messages when it encounters a construct
that cannot be converted. 

In practice, the synthesizable subset usually refers to RTL synthesis,
which is by far the most popular type of synthesis today. There are
industry standards that define the RTL synthesis subset.  However,
those were not used as a model for the restrictions of the MyHDL
converter, but as a minimal starting point.  On that basis, whenever
it was judged easy or useful to support an additional feature, this
was done. For example, it is actually easier to convert
\keyword{while} loops than \keyword{for} loops even though they are
not RTL-synthesizable.  As another example, \keyword{print} is
supported because it's so useful for debugging, even though it's not
synthesizable.  In summary, the convertible subset is a superset of
the standard RTL synthesis subset, and supports synthesis tools with
more advanced capabilities, such as behavioral synthesis.

Recall that any restrictions only apply to the design post
elaboration.  In practice, this means that they apply only to the code
of the generators, that are the leaf functional blocks in a MyHDL
design.

\subsection{Coding style\label{conv-subset-style}}

A natural restriction on convertible code is that it should be
written in MyHDL style: cooperating generators, communicating through
signals, and with \code{yield} statements specifying wait points and resume
conditions.  Supported resume conditions are a signal edge, a signal
change, or a tuple of such conditions.

\subsection{Supported types\label{conv-subset-types}}

The most important restriction regards object types. Verilog is an
almost typeless language, while Python is strongly (albeit
dynamically) typed. The converter has to infer the types of names
used in the code, and map those names to Verilog variables.

Only a limited amount of types can be converted.
Python \class{int} and \class{long} objects are mapped to Verilog
integers. All other supported types are mapped to Verilog regs (or
wires), and therefore need to have a defined bit width. The supported
types are the Python \class{bool} type, the MyHDL \class{intbv} type,
and MyHDL enumeration types returned by function \function{enum}. The
latter objects can also be used as the base object of a
\class{Signal}. 

\class{intbv} objects must be constructed so that a bit
width can be inferred. This can be done by specifying minimum
and maximum values, e.g. as follows:

\begin{verbatim}
index = intbv(0, min=0, max=2**N)
\end{verbatim}

Alternatively, a slice can be taken from an \class{intbv} object
as follows:

\begin{verbatim}
index = intbv(0)[N:]
\end{verbatim}

Such as slice returns a new \class{intbv} object, with minimum
value \code{0} , and maximum value \code{2**N}.


\subsection{Supported statements\label{conv-subset-statements}}

The following is a list of the statements that are supported by the
Verilog converter, possibly qualified with restrictions
or usage notes. 

\begin{description}

\item[The \keyword{break} statement.]

\item[The \keyword{continue} statement.]

\item[The \keyword{def} statement.]

\item[The \keyword{for} statement.]
The only supported iteration scheme is iterating through sequences of
integers returned by built-in function \function{range} or \myhdl\
function \function{downrange}.  The optional \keyword{else} clause is
not supported.

\item[The \keyword{if} statement.]
\keyword{if}, \keyword{elif}, and \keyword{else} clauses
are fully supported.

\item[The \keyword{pass} statement.]

\item[The \keyword{print} statement.]
When printing an interpolated string, the format specifiers are copied
verbatim to the Verilog output.  Printing to a file (with syntax
\code{'>>'}) is not supported.

\item[The \keyword{raise} statement.]
This statement is mapped to Verilog statements
that end the simulation with an error message.

\item[The \keyword{return} statement.]

\item[The \keyword{yield} statement.] 
The yielded expression can be a signal, a signal edge
as specified by \myhdl\ functions \function{posedge}
or \function{negedge}, or a tuple of signals and
edge specifications.

\item[The \keyword{while} statement.]
The optional \keyword{else}
clause is not supported.

\end{description}

\section{Methodology notes\label{conv-meth}}

\subsection{Simulation\label{conv-meth-sim}}

In the Python philosophy, the run-time rules. The Python compiler
doesn't attempt to detect a lot of errors beyond syntax errors, which
given Python's ultra-dynamic nature would be an almost impossible task
anyway. To verify a Python program, one should run it, preferably
using unit testing to verify each feature.

The same philosophy should be used when converting a MyHDL description
to Verilog: make sure the simulation runs fine first. Although the
converter checks many things and attempts to issue clear error
messages, there is no guarantee that it does a meaningful job unless
the simulation runs fine.

\subsection{Conversion output verification\label{conv-meth-conv}}
It is always prudent to verify the converted Verilog output.
To make this task easier, the converter also generates a
test bench that makes it possible to simulate the Verilog
design using the Verilog co-simulation interface. This 
permits to verify the Verilog code with the same test
bench used for the \myhdl\ code. This is also how
the Verilog converter development is being verified.

\subsection{Assignment issues\label{conv-meth-assign}}

\subsubsection{Name assignment in Python\label{conv-meth-assign-python}}

Name assignment in Python is a different concept than in
many other languages. This point is very important for
effective modeling in Python, and even more so
for synthesizable \myhdl\ code. Therefore, the issues are
discussed here explicitly.

Consider the following name assignments:

\begin{verbatim}
a = 4
a = ``a string''
a = False
\end{verbatim}

In many languages, the meaning would be that an
existing variable \var{a} gets a number of different values.
In Python, such a concept of a variable doesn't exist. Instead,
assignment merely creates a new binding of a name to a
certain object, that replaces any previous binding.
So in the example, the name \var{a} is bound a 
number of different objects in sequence.

The Verilog converter has to investigate name
assignment and usage in \myhdl\ code, and to map
names to Verilog variables. To achieve that,
it tries to infer the type and possibly the
bit width of each expression that is assigned
to a name.

Multiple assignments to the same name can be supported if it can be
determined that a consistent type and bit width is being used in the
assignments. This can be done for boolean expressions, numeric
expressions, and enumeration type literals. In Verilog, the
corresponding name is mapped to a single bit \code{reg}, an
\code{integer}, or a \code{reg} with the appropriate width, respectively.

In other cases, a single assignment should be used when an object is
created. Subsequent value changes are then achieved by modification of
an existing object.  This technique should be used for \class{Signal}
and \class{intbv} objects.

\subsubsection{Signal assignment\label{conv-meth-assign-signal}}

Signal assignment in \myhdl\ is implemented using attribute assignment
to attribute \code{next}.  Value changes are thus modeled by
modification of the existing object. The converter investigates the
\class{Signal} object to infer the type and bit width of the
corresponding Verilog variable.

\subsubsection{\class{intbv} objects\label{conv-meth-assign-intbv}}

Type \class{intbv} is likely to be the workhorse for synthesizable
modeling in \myhdl{}. An \class{intbv} instance behaves like a
(mutable) integer whose individual bits can be accessed and
modified. Also, it is possible to constrain its set of values. In
addition to error checking, this makes it possible to infer a bit
width, which is required for implementation.

In Verilog, an \class{intbv} instance will be mapped to a \code{reg}
with an appropriate width. As noted before, it is not possible
to modify its value using name assignment. In the following, we
will show how it can be done instead. Consider:

\begin{verbatim}
a = intbv(0)[8:]
\end{verbatim}

This is an \class{intbv} object with initial value \code{0} and
bit width 8. The change its value to \code{5}, we can use
slice assignment:

\begin{verbatim}
a[8:] = 5
\end{verbatim}

The same can be achieved by leaving the bit width unspecified, 
which has the meaning to change ``all'' bits:

\begin{verbatim}
a[:] = 5
\end{verbatim}

Often the new value will depend on the old one. For example,
to increment an \class{intbv} with the technique above:

\begin{verbatim}
a[:] = a + 1
\end{verbatim}

Python also provides \emph{augmented} assignment operators,
which can be used to implement in-place operations. These are supported
on \class{intbv} objects and by the converter, so that the increment
can also be done as follows:

\begin{verbatim}
a += 1
\end{verbatim}

\section{Converter usage\label{conv-usage}}

We will demonstrate the conversion process by showing some examples.

\subsection{A small design with a single generator\label{conv-usage-small}}

Consider the following MyHDL code for an incrementer module:

\begin{verbatim}
def inc(count, enable, clock, reset, n):
    """ Incrementer with enable.
    
    count -- output
    enable -- control input, increment when 1
    clock -- clock input
    reset -- asynchronous reset input
    n -- counter max value
    """
    def incProcess():
        while 1:
            yield posedge(clock), negedge(reset)
            if reset == ACTIVE_LOW:
                count.next = 0
            else:
                if enable:
                    count.next = (count + 1) % n
    return incProcess()
\end{verbatim}

In Verilog terminology, function \function{inc} corresponds to a
module, while generator function \function{incProcess}
roughly corresponds to an always block.

Normally, to simulate the design, we would "elaborate" an instance
as follows:

\begin{verbatim}
m = 8
n = 2 ** m
 
count = Signal(intbv(0)[m:])
enable = Signal(bool(0))
clock, reset = [Signal(bool()) for i in range(2)]

inc_inst = inc(count, enable, clock, reset, n=n)
\end{verbatim}

\code{inc_inst} is an elaborated design instance that can be simulated. To
convert it to Verilog, we change the last line as follows:

\begin{verbatim}
inc_inst = toVerilog(inc, count, enable, clock, reset, n=n)
\end{verbatim}

Again, this creates an instance that can be simulated, but as a side
effect, it also generates an equivalent Verilog module in file \file{inc_inst.v},
The Verilog code looks as follows:

\begin{verbatim}
module inc_inst (
    count,
    enable,
    clock,
    reset
);

output [7:0] count;
reg [7:0] count;
input enable;
input clock;
input reset;


always @(posedge clock or negedge reset) begin: _MYHDL1_BLOCK
    if ((reset == 0)) begin
        count <= 0;
    end
    else begin
        if (enable) begin
            count <= ((count + 1) % 256);
        end
    end
end

endmodule
\end{verbatim}

You can see the module interface and the always block, as expected
from the MyHDL design. 

\subsection{Converting a generator directly\label{conv-usage-gen}}

It is also possible to convert a generator
directly. For example, consider the following generator function:

\begin{verbatim}
def bin2gray(B, G, width):
    """ Gray encoder.

    B -- input intbv signal, binary encoded
    G -- output intbv signal, gray encoded
    width -- bit width
    """
    Bext = intbv(0)[width+1:]
    while 1:
        yield B
        Bext[:] = B
        for i in range(width):
            G.next[i] = Bext[i+1] ^ Bext[i]
\end{verbatim}

As before, you can create an instance and convert to
Verilog as follows:

\begin{verbatim}
width = 8

B = Signal(intbv(0)[width:])
G = Signal(intbv(0)[width:])

bin2gray_inst = toVerilog(bin2gray, B, G, width)
 \end{verbatim}

The generate Verilog code looks as follows:

\begin{verbatim}
module bin2gray_inst (
    B,
    G
);

input [7:0] B;
output [7:0] G;
reg [7:0] G;

always @(B) begin: _MYHDL1_BLOCK
    integer i;
    reg [9-1:0] Bext;
    Bext[9-1:0] = B;
    for (i=0; i<8; i=i+1) begin
        G[i] <= (Bext[(i + 1)] ^ Bext[i]);
    end
end

endmodule
\end{verbatim}

\subsection{A hierarchical design\label{conv-usage-hier}}
The hierarchy of convertible designs can be
arbitrarily deep.

For example, suppose we want to design an
incrementer with Gray code output. Using the
designs from previous sections, we can proceed
as follows:

\begin{verbatim}
def GrayInc(graycnt, enable, clock, reset, width):
    
    bincnt = Signal(intbv()[width:])
    
    INC_1 = inc(bincnt, enable, clock, reset, n=2**width)
    BIN2GRAY_1 = bin2gray(B=bincnt, G=graycnt, width=width)
    
    return INC_1, BIN2GRAY_1
\end{verbatim}

According to Gray code properties, only a single bit
will change in consecutive values. However, as the
\code{bin2gray} module is combinatorial, the output bits
may have transient glitches, which may not be desirable.
To solve this, let's create an additional level of
hierarchy and add an output register to the design.
(This will create an additional latency of a clock
cycle, which may not be acceptable, but we will
ignore that here.)

\begin{verbatim}
def GrayIncReg(graycnt, enable, clock, reset, width):
    
    graycnt_comb = Signal(intbv()[width:])
    
    GRAY_INC_1 = GrayInc(graycnt_comb, enable, clock, reset, width)
    
    def reg():
        while 1:
            yield posedge(clock)
            graycnt.next = graycnt_comb
    REG_1 = reg()
    
    return GRAY_INC_1, REG_1
\end{verbatim}

We can convert this hierarchical design as before:

\begin{verbatim}
width = 8
graycnt = Signal(intbv()[width:])
enable, clock, reset = [Signal(bool()) for i in range(3)]

GRAY_INC_REG_1 = toVerilog(GrayIncReg, graycnt, enable, clock, reset, width)
\end{verbatim}

The Verilog output code looks as follows:

\begin{verbatim}
module GRAY_INC_REG_1 (
    graycnt,
    enable,
    clock,
    reset
);

output [7:0] graycnt;
reg [7:0] graycnt;
input enable;
input clock;
input reset;

reg [7:0] graycnt_comb;
reg [7:0] _GRAY_INC_1_bincnt;

always @(posedge clock or negedge reset) begin: _MYHDL1_BLOCK
    if ((reset == 0)) begin
        _GRAY_INC_1_bincnt <= 0;
    end
    else begin
        if (enable) begin
            _GRAY_INC_1_bincnt <= ((_GRAY_INC_1_bincnt + 1) % 256);
        end
    end
end

always @(_GRAY_INC_1_bincnt) begin: _MYHDL4_BLOCK
    integer i;
    reg [9-1:0] Bext;
    Bext[9-1:0] = _GRAY_INC_1_bincnt;
    for (i=0; i<8; i=i+1) begin
        graycnt_comb[i] <= (Bext[(i + 1)] ^ Bext[i]);
    end
end

always @(posedge clock) begin: _MYHDL9_BLOCK
    graycnt <= graycnt_comb;
end

endmodule
\end{verbatim}

Note that the output is a flat ``net list of blocks'', and
that hierarchical signal names are generated as necessary.

\subsection{Optimizations for finite state machines\label{conv-usage-fsm}}
As often in hardware design, finite state machines deserve special attention.

In Verilog and VHDL, finite state machines are typically described
using case statements.  Python doesn't have a case statement, but the
converter recognizes particular if-then-else structures and maps them
to case statements. This optimization occurs when a variable whose
type is an enumerated type is sequentially tested against enumeration
items in an if-then-else structure. Also, the appropriate synthesis
pragmas for efficient synthesis are generated in the Verilog code.

As a further optimization, function \function{enum} was enhanced to support
alternative encoding schemes elegantly, using an additional parameter
\var{encoding}. For example:

\begin{verbatim}
t_State = enum('SEARCH', 'CONFIRM', 'SYNC', encoding='one_hot')
\end{verbatim}

The default encoding is \code{'binary'}; the other possibilities are
\code{'one_hot'} and \code{'one_cold'}. This parameter only
affects the conversion output, not the behavior of the type. Verilog
case statements are optimized for an efficient implementation
according to the encoding. Note that in contrast, a Verilog designer
needs to make nontrivial code changes to implement a different
encoding scheme.

As an example, consider the following finite state machine, whose
state variable uses the enumeration type defined above:

\begin{verbatim}
FRAME_SIZE = 8

def FramerCtrl(SOF, state, syncFlag, clk, reset_n):
    
    """ Framing control FSM.

    SOF -- start-of-frame output bit
    state -- FramerState output
    syncFlag -- sync pattern found indication input
    clk -- clock input
    reset_n -- active low reset
    
    """
    
    index = intbv(0, min=0, max=8) # position in frame
    while 1:
        yield posedge(clk), negedge(reset_n)
        if reset_n == ACTIVE_LOW:
            SOF.next = 0
            index[:] = 0
            state.next = t_State.SEARCH
        else:
            SOF.next = 0
            if state == t_State.SEARCH:
                index[:] = 0
                if syncFlag:
                    state.next = t_State.CONFIRM
            elif state == t_State.CONFIRM:
                if index == 0:
                    if syncFlag:
                        state.next = t_State.SYNC
                    else:
                        state.next = t_State.SEARCH
            elif state == t_State.SYNC:
                if index == 0:
                    if not syncFlag:
                        state.next = t_State.SEARCH
                SOF.next = (index == FRAME_SIZE-1)
            else:
                raise ValueError("Undefined state")
            index[:]= (index + 1) % FRAME_SIZE

\end{verbatim}

The conversion is done as before:

\begin{verbatim}
SOF = Signal(bool(0))
syncFlag = Signal(bool(0))
clk = Signal(bool(0))
reset_n = Signal(bool(1))
state = Signal(t_State.SEARCH)
framerctrl_inst = toVerilog(FramerCtrl, SOF, state, syncFlag, clk, reset_n)
\end{verbatim}

The Verilog output looks as follows:

\begin{verbatim}
module framerctrl_inst (
    SOF,
    state,
    syncFlag,
    clk,
    reset_n
);
output SOF;
reg SOF;
output [2:0] state;
reg [2:0] state;
input syncFlag;
input clk;
input reset_n;

always @(posedge clk or negedge reset_n) begin: _MYHDL1_BLOCK
    reg [3-1:0] index;
    if ((reset_n == 0)) begin
        SOF <= 0;
        index[3-1:0] = 0;
        state <= 3'b001;
    end
    else begin
        SOF <= 0;
        // synthesis parallel_case full_case
        casez (state)
            3'b??1: begin
                index[3-1:0] = 0;
                if (syncFlag) begin
                    state <= 3'b010;
                end
            end
            3'b?1?: begin
                if ((index == 0)) begin
                    if (syncFlag) begin
                        state <= 3'b100;
                    end
                    else begin
                        state <= 3'b001;
                    end
                end
            end
            3'b1??: begin
                if ((index == 0)) begin
                    if ((!syncFlag)) begin
                        state <= 3'b001;
                    end
                end
                SOF <= (index == (8 - 1));
            end
            default: begin
                $display("Verilog: ValueError(Undefined state)");
                $finish;
            end
        endcase
        index[3-1:0] = ((index + 1) % 8);
    end
end
endmodule
\end{verbatim}


\section{Known issues\label{conv-issues}}
\begin{description}

\item[Negative values of \class{intbv} instances are not supported.]
The \class{intbv} class is quite capable of representing negative
values. However, the \code{signed} type support in Verilog is
relatively recent and mapping to it may be tricky. In my judgment,
this was not the most urgent requirement, so
I decided to leave this for later.

\item[Verilog integers are 32 bit wide]
Usually, Verilog integers are 32 bit wide. In contrast, Python is
moving toward integers with undefined width. Python \class{int} 
and \class{long} variables are mapped to Verilog integers; so for values
wider than 32 bit this mapping is incorrect.

\item[Synthesis pragmas are specified as Verilog comments.] The recommended
way to specify synthesis pragmas in Verilog is through attribute
lists. However, my Verilog simulator (Icarus) doesn't support them
for \code{case} statements (to specify \code{parallel_case} and
\code{full_case} pragmas). Therefore, I still used the old
but deprecated method of synthesis pragmas in Verilog comments.

\item[Inconsistent place of the sensitivity list inferred from \code{always_comb}.]
The semantics of \code{always_comb}, both in Verilog and \myhdl{}, is to
have an implicit sensitivity list at the end of the code. However, this
may not be synthesizable. Therefore, the inferred sensitivity list is
put at the top of the corresponding \code{always} block.
This may cause inconsistent behavior at the start of the
simulation. The workaround is to create events at time 0.

\item[Non-blocking assignments to task arguments don't work.] 
I didn't get non-blocking (signal) assignments to task arguments to
work.  I don't know yet whether the issue is my own, a Verilog issue,
or an issue with my Verilog simulator Icarus. I'll need to check this
further.

\end{description}
