\documentclass{manual}
\usepackage{palatino}
\renewcommand{\ttdefault}{cmtt}
\renewcommand{\sfdefault}{cmss}
\newcommand{\myhdl}{\protect \mbox{MyHDL}}
\usepackage{graphicx}

\title{The \myhdl\ manual}

\author{Jan Decaluwe}
\authoraddress{
Email: \email{jan@jandecaluwe.com}
}

\date{December 19, 2005}	% XXX update before release!
\release{0.5b1}  		% software release, not documentation
\setreleaseinfo{}		% empty for final release
\setshortversion{0.5b1}		% major.minor only for software


\makeindex

\begin{document}

\maketitle

Copyright \copyright{} 2001 Python Software Foundation.
All rights reserved.

Copyright \copyright{} 2000 BeOpen.com.
All rights reserved.

Copyright \copyright{} 1995-2000 Corporation for National Research Initiatives.
All rights reserved.

Copyright \copyright{} 1991-1995 Stichting Mathematisch Centrum.
All rights reserved.

See the end of this document for complete license and permissions
information.


\begin{abstract}

\noindent

\myhdl{} is a Python package for using Python as a hardware description
and verification language. Languages such Verilog and VHDL are
compiled languages. Python with \myhdl{} can be viewed as a "scripting
language" counterpart of such languages. However, Python is more
accurately described as a very high level language (VHLL). \myhdl{} users
have access to the amazing power and elegance of Python.

The key idea behind \myhdl{} is to use Python generators for modeling
hardware concurrency. A generator is a resumable function with
internal state. In \myhdl{}, a hardware module is modeled as a function
that returns generators. With this approach, \myhdl{} directly supports
features such as named port association, arrays of instances, and
conditional instantiation. 

\myhdl{} supports the classic hardware description concepts.  It
provides a signal class similar to the VHDL signal, a class for bit
oriented operations, and support for enumeration types.  The Python
\code{yield} statement is used as a general sensitivity list to wait
on a signal change, an edge, a delay, or on the completion of another
generator. \myhdl{} supports waveform viewing by tracing signal
changes in a VCD file.

Python's rare combination of power and clarity makes it ideal for high
level modeling.  It can be expected that \myhdl{} users will often
have the ``Pythonic experience'' of finding an elegant solution to a
complex modeling problem. Moreover, Python is outstanding for rapid
application development and experimentation.

With \myhdl{}, the Python unit test framework can be used on hardware
designs.  \myhdl{} can also be used as hardware verification language for
VHDL and Verilog designs, by co-simulation with any simulator that has
a PLI.  The distribution contains a PLI module for the
Icarus Verilog simulator.

Finally, a subset of \myhdl{} code can be converted automatically into
synthesizable Verilog code. This feature provides a direct path from
Python to an FPGA or ASIC implementation.

The \myhdl{} software is open source software. It is licensed under the
GNU Lesser General Public License (LGPL).




\end{abstract}

\tableofcontents

\chapter{Background information}

\section{Prerequisites}

You need a basic understanding of Python to use \myhdl\.
If you don't know Python, you will take comfort in knowing
that it is probably one of the easiest programming languages to
learn \footnote{You must be bored by such claims, but in Python's
case it's true.}. Learning Python is also one of the better time
investments that engineering professionals can make \footnote{I am not
biased.}.

For beginners, \url{http://www.python.org/doc/current/tut/tut.html} is
probably the best choice for an on-line tutorial. For alternatives,
see \url{http://www.python.org/doc/Newbies.html}.

A working knowledge of a hardware description language such as Verilog
or VHDL is helpful. Chances are that you know one of those anyway, if
you are interested in \myhdl{}.

\section{A small tutorial on generators}

Generators are a recent feature in Python. They were introduced in
Python 2.2, which is the most recent stable version at the time of
this writing. Therefore, there isn't a lot of tutorial material
available yet. Because generators are the key concept in
\myhdl{}, I include a small tutorial here.

Consider the following nonsensical function:

\begin{verbatim}
def function():
    for i in range(5):
        return i

\end{verbatim}

You can see why it doesn't make a lot of sense. As soon as the first
loop iteration is entered, the function returns:

\begin{verbatim}
>>> function()
0
\end{verbatim}

Returning is fatal for the function call. Further loop iterations
never get a chance, and nothing is left over from the function call
when it returns.

To change the function into a generator function, we replace
\keyword{return} with \keyword{yield}:

\begin{verbatim}
def generator():
    for i in range(5):
        yield i

\end{verbatim}

Now we get:

\begin{verbatim}
>>> generator()
<generator object at 0x815d5a8>

\end{verbatim}

When a generator function is called, it returns a generator object. A
generator object supports the iterator protocol, which is an expensive
way of saying that you can let it generate subsequent values by
calling its \function{next()} method:

\begin{verbatim}
>>> g = generator()
>>> g.next()
0
>>> g.next()
1
>>> g.next()
2
>>> g.next()
3
>>> g.next()
4
>>> g.next()
Traceback (most recent call last):
  File "<stdin>", line 1, in ?
StopIteration

\end{verbatim}

Now we can generate the subsequent values from the for
loop on demand, until they are exhausted. What happens is that the
\keyword{yield} statement is like a
\keyword{return}, except that it is non-fatal: the generator remembers
its state and the point in the code when it yielded. A higher order
agent can decide when to get a further value by calling the
generator's \function{next()} method. We say that generators are
\dfn{resumable functions}.

If you are familiar with hardware description languages, this may
ring a bell. In hardware simulations, there is also a higher order
agent, the Simulator tool, that interacts with such resumable
functions; they are called processes in VHDL and always blocks in
Verilog. Like in those languages, Python generators provide an elegant
and efficient method to model concurrency, without having to resort to
some form of threading.

The use of generators to model concurrency is the first key concept in
\myhdl{}. The second key concept is a related one: in \myhdl{}, the
yielded values are used to define the condition upon which the generator
should be resumed. In other words, the \keyword{yield} statements work
as generalized sensitivity lists. If by now you are still interested,
read on to learn more!

If you want to know more about generators, consult the on-line Python
documentation, e.g. at \url{http://www.python.org/doc/2.2.2/whatsnew}. 

\begin{notice}[warning]
At the beginning of this section I said that generators were
introduced in Python 2.2. This is not entirely correct: in fact,
generators will only be enabled as a standard feature in Python 2.3.
However, a stable version of Python 2.3 has not been released yet at
the time of this writing. So, what to do?

Fortunately, Python lets you import features from its future releases
(provided that the future is not too distant). So, until you use
Python 2.3 or higher, you have to include the following line at the
start of all code that defines generator functions:

\begin{verbatim}
from __future__ import generators

\end{verbatim}

From Python 2.3 on, this line may still be in the code, though it
will not have an effect anymore.
\end{notice}


\chapter{Introduction to \myhdl\ }

\section{A basic \myhdl\ simulation}

We will introduce \myhdl\ with a classical \code{Hello World} style
example. Here are the contents of a \myhdl\ simulation script called
\file{Hello1.py}:

\begin{verbatim}
from myhdl import delay, now, Simulation

def sayHello():
    while 1:
        yield delay(10)
        print "%s Hello World!" % now()

gen = sayHello()
sim = Simulation(gen)
sim.run(30)

\end{verbatim}

When we run this script, we get the following output: 

\begin{verbatim}
% python Hello1.py
10 Hello World!
20 Hello World!
30 Hello World!
StopSimulation: Simulated for duration 30

\end{verbatim}

The first line of the script imports a
number of objects from the \code{myhdl} package. In good Python style, and
unlike most other languages, we can only use identifiers that are
\emph{literally} defined in the source file \footnote{I don't want to
explain the \samp{import *} syntax}.

Next, we define a generator function called
\code{sayHello}. This is a generator function (as opposed to
a classic Python function) because it contains a \code{yield}
statement (instead of \code{return} statement). In \myhdl\, a
\code{yield} statement has a similar purpose as a \code{wait}
statement in VHDL: the statement suspends execution of the function,
and its clauses specify when the function should resume. In this case,
there is a \code{delay} clause, that specifies the required delay.

To make sure that the generator runs ``forever'', we wrap its behavior
in a \code{while 1} loop. This is as standard Python idiom, and it is
the \myhdl\ equivalent to a Verilog \code{always} block or a
VHDL \code{process}.

In \myhdl\, the basic simulation objects are generators. Generators
are created by calling a generator function. For example, variable
\code{gen} refers to a generator. To simulate this generator, we pass
it as an argument to a code{Simulation} object constructor.  We then
run the simulation for the desired amount of time.


\section{Concurrent generators and signals}

In the previous section, we simulated a single generator. Of course,
real hardware descriptions are not like that: in fact, they are
typically massively concurrent. \myhdl\ supports this by allowing an
arbitrary number of concurrent generators. More specifically, a
\code{Simulation} constructor can take an arbitrary number of
arguments, each of which can be a generator or a nested list of
generators.

With concurrency comes the problem of determinism. Therefore, hardware
languages use special objects to support deterministic communication
between concurrent regions. \myhdl\ has as \code{Signal} object which
is roughly modelled after VHDL signals.

We will demonstrate these concepts by extending our first example. We
introduce a clock signal, driven by a second generator. The
\code{sayHello} generator function is modified to wait for a rising
edge (\code{posedge}) of the clock instead of a delay. The resulting
script is as follows:

\begin{verbatim}
from myhdl import Signal, delay, posedge, now, Simulation

clk = Signal(0)

def clkGen():
    while 1:
        yield delay(10)
        clk.next = 1
        yield delay(10)
        clk.next = 0

def sayHello():
    while 1:
        yield posedge(clk)
        print "%s Hello World!" % now()

sim = Simulation(clkGen(), sayHello())
sim.run(50)

\end{verbatim}

When we run this script, we get:

\begin{verbatim}
% python Hello2.py
10 Hello World!
30 Hello World!
50 Hello World!
StopSimulation: Simulated for duration 50

\end{verbatim}

The \code{clk} signal is constructed with an initial value
\code{0}. In the clock generator function \code{clkGen}, it is then
continuously toggled after a certain delay. In \myhdl{}, a the next
value of a signal is specified by assigning to its \code{next}
attribute. This is the \myhdl\ equivalent of VHDL signal assignments
and Verilog's non-blocking assignments.

The \code{sayHello} generator function shows a second form of a
\code{yield} statement: \samp{yield posedge(\var{aSignal})}. Again,
the generator will suspend execution at that point, but in this case
it specifies that it should resume when there is a rising edge on the
signal.

The \code{Simulation} constructor now takes two generator arguments
that run concurrently throughout the simulation.

\section{Parameters and instantiations}

So far, the generator function examples had no parameters. The signals
they operated on were defined in their enclosing scope. However, 
to make the code reusable we will want to pass arguments through a
parameter list. For example, we can change the clock generator
function as follows to make it more general and reusable:

\begin{verbatim}
def clkGen(clock, period=20):
    lowTime = int(period/2)
    highTime = period - lowTime
    while 1:
        yield delay(lowTime)
        clock.next = 1
        yield delay(highTime)
        clock.next = 0

\end{verbatim}

The clock signal is now a parameter of the function. Also, the clock
period is a parameter with a default value of \code{20}.

Similarly, the \code{sayHello} function can be made more general:

\begin{verbatim}
def sayHello(clock, to="World!"):
    while 1:
        yield posedge(clock)
        print "%s Hello %s" % (now(), to)

\end{verbatim}

XXX

Multiple generators can be created by multiple calls to a generator
function, possibly with different parameters. This is analogous to the
concept of \emph{instantiation} in hardware description
languages. \myhdl\ supports hierarchy and instantiations through
higher level functions that return multiple generators.

\begin{verbatim}
def talk():

    clk1 = Signal(0)
    clk2 = Signal(0)
    
    clkGen1 = clkGen(clk1)
    clkGen2 = clkGen(clock=clk2, period=19)
    sayHello1 = sayHello(clock=clk1)
    sayHello2 = sayHello(to="MyHDL", clock=clk2)

    return (clkGen1, clkGen2, sayHello1, sayHello2) 

sim = Simulation(talk())
sim.run(50)

\end{verbatim}

This produces the following output:

\begin{verbatim}
% python Hello3.py
9 Hello MyHDL
10 Hello World!
28 Hello MyHDL
30 Hello World!
47 Hello MyHDL
50 Hello World!
StopSimulation: Simulated for duration 50

\end{verbatim}

Like in standard Python, positional or named parameter association can
be used, or a mix of the two \footnote{All positional parameters have
to come before any named parameter.}. These styles are demonstrated in
the example. Named association is very useful if there are a lot of
parameters. 


\chapter{Modeling techniques}

\section{RTL modeling}
The present section describes how \myhdl\ supports RTL style modeling
as is typically used for synthesizable models in Verilog or VHDL.

In this domain, \myhdl\ doesn't offer advantages compared to
other solutions. However, as this modeling style is well-known,
this section may be useful for illustrative purposes.

\subsection{Combinatorial logic}

\subsubsection{Template}

Combinatorial logic is described with a generator function code template as
follows: 

\begin{verbatim}
def combinatorialLogic(<arguments>)
    while 1:
        yield <input signal arguments>
        <functional code>

\end{verbatim}

The overall code is wrapped in a \code{while 1} statement to keep the
generator alive. All input signals are clauses in the \code{yield}
statement, so that the generator resumes whenever one of the inputs
changes. 

\subsubsection{Example}

The following is an example of a combinatorial multiplexer:

\begin{verbatim}
def mux(z, a, b, sel):
    """ Multiplexer.
    
    z -- mux output
    a, b -- data inputs
    sel -- control input: select a if asserted, otherwise b
    """
    while 1:
        yield a, b, sel
        if sel == 1:
            z.next = a
        else:
            z.next = b
\end{verbatim}

To verify, let's simulate this logic with some random patterns. The
\code{random} module in Python's standard library comes in handy for
such purposes. The function \code{randrange(\var{n})} returns a random
natural integer smaller than \var{n}. It is used in the test bench
code to produce random input values:

\begin{verbatim}
from random import randrange

(z, a, b, sel) = [Signal(0) for i in range(4)]

MUX_1 = mux(z, a, b, sel)

def test():
    print "z a b sel"
    for i in range(8):
        a.next, b.next, sel.next = randrange(8), randrange(8), randrange(2)
        yield delay(10)
        print "%s %s %s %s" % (z, a, b, sel)
        
Simulation(MUX_1, test()).run() 
   
\end{verbatim}

Because of the randomness, the simulation output varies between runs
\footnote{It also possible to have a reproducible random output, by
explicitly providing a seed value. See the documentation of the
\code{random} module}. One particular run produced the following
output:

\begin{verbatim}
% python mux.py
z a b sel
6 6 1 1
7 7 1 1
7 3 7 0
1 2 1 0
7 7 5 1
4 7 4 0
4 0 4 0
3 3 5 1
StopSimulation: No more events
\end{verbatim}


\subsection{Sequential logic}

\subsubsection{Template}
Sequential RTL models are sensitive to a clock edge. In addition, they
may be sensitive to a reset signal. We will describe one of the most
common patterns: a template with a rising clock edge and an
asynchronous reset signal. Other templates are similar.

\begin{verbatim}
def sequentialLogic(<arguments>, clock, ..., reset, ...)
    while 1:
        yield posedge(clock), negedge(reset)
        if reset == <active level>:
            <reset code>
        else:
            <functional code>

\end{verbatim}


\subsubsection{Example}
The following code is a description of an incrementer with enable, and
an asynchronous power-up reset.

\begin{verbatim}
ACTIVE_LOW, INACTIVE_HIGH = 0, 1

def Inc(count, enable, clock, reset, n):
    """ Incrementer with enable.
    
    count -- output
    enable -- control input, increment when 1
    clock -- clock input
    reset -- asynchronous reset input
    n -- counter max value
    """
    while 1:
        yield posedge(clock), negedge(reset)
        if reset == ACTIVE_LOW:
            count.next = 0
        else:
            if enable:
                count.next = (count + 1) % n

\end{verbatim}

For the test bench, we will use an independent clock generator, stimulus
generator, and monitor. After applying enough stimulus patterns, we
can raise the \code{myhdl.StopSimulation} exception to stop the
simulation run. The test bench for a small incrementer and a small
number of patterns is a follows:

\begin{verbatim}
count, enable, clock, reset = [Signal(intbv(0)) for i in range(4)]

INC_1 = Inc(count, enable, clock, reset, n=4)

def clockGen():
    while 1:
        yield delay(10)
        clock.next = not clock

def stimulus():
    reset.next = ACTIVE_LOW
    yield negedge(clock)
    reset.next = INACTIVE_HIGH
    for i in range(12):
        enable.next = min(1, randrange(3))
        yield negedge(clock)
    raise StopSimulation

def monitor():
    print "enable  count"
    yield posedge(reset)
    while 1:
        yield posedge(clock)
        yield delay(1)
        print "   %s      %s" % (enable, count)
        
Simulation(clockGen(), stimulus(), monitor(), INC_1).run()

\end{verbatim}

The simulation produces the following output:
\begin{verbatim}
% python inc.py
enable  count
   0      0
   1      1
   0      1
   1      2
   1      3
   1      0
   0      0
   1      1
   0      1
   0      1
   0      1
   1      2
StopSimulation

\end{verbatim}


\section{High level modeling}

test

\chapter{Unit testing}

\section{Introduction}

Many aspects in the design flow of modern digital hardware design can
be viewed as a special kind of software development. From that
viewpoint, it is a natural question whether advances in software
design techniques can not also be applied to hardware design.

One software design approach that gets a lot of attention recently is
\emph{Extreme Programming} (XP). It is a fascinating set of techniques and
guidelines that often seems to go against the conventional wisdom. On
other occasions, XP just seems to emphasize the common sense, which
doesn't always coincide with common practice. For example, XP stresses
the importance of normal workweeks, if we are to have the
fresh mind needed for good software development.

It is not my intention nor qualification to present a tutorial on
Extreme Programming. Instead, in this section I will highlight one XP
concept which I think is very relevant to hardware design: the
importance and methodology of unit testing.

\section{The importance of unit tests}

Unit testing is one of the corner stones of Extreme Programming. Other
XP concepts, such as collective ownership of code and continuous
refinement, are only possible by having unit tests. Moreover, XP
emphasizes that writing unit tests should be automated, that they should
test everything in every class, and that they should run perfectly all
the time. 

I believe that these concepts apply directly to hardware design. In
addition, unit tests are a way to manage simulation time. For example,
a state machine that runs very slowly on infrequent events may be
impossible to verify at the system level, even on the fastest
simulator. On the other hand, it may be easy to verify it exhaustively
in a unit test, even on the slowest simulator.

It is clear that unit tests have compelling advantages. On the other
hand, if we need to test everything, we have to write
lots of unit tests. So it should be easy and pleasant
to create, manage and run them. Therefore, XP emphasizes the need for
a unit test framework that supports these tasks. In this chapter,
we will explore the use of the \code{unittest} module from
the standard Python library for creating unit tests for hardware
designs.


\section{Unit test development}

In this section, we will informally explore the application of unit
test techniques to hardware design. We will do so by a (small)
example: testing a binary to Gray encoder as introduced in
section~\ref{gray}. 

\subsection{Defining the requirements}

We start by defining the requirements. For a Gray encoder, we want to
the output to comply with Gray code characteristics. Let's define a
\dfn{code} as a list of \dfn{codewords}, where a codeword is a bit
string. A code of order \code{n} has \code{2**n} codewords.

A well-known characteristic is the one that Gray codes are all about:

\newtheorem{reqGray}{Requirement}
\begin{reqGray} 
Consecutive codewords in a Gray code should differ in a single bit.
\end{reqGray}

Is this sufficient? Not quite: suppose for example that an
implementation returns the lsb of each binary input. This would comply
with the requirement, but is obviously not what we want. Also, we don't
want the bit width of Gray codewords to exceed the bit width of the
binary codewords.

\begin{reqGray} 
Each codeword in a Gray code of order n must occur exactly once in the
binary code of the same order.
\end{reqGray}

With the requirements written down we can proceed.

\subsection{Writing the test first}

A fascinating guideline in the XP world is to write the unit test
first. That is, before implementing something, first write the test
that will verify it. This seems to go against our natural inclination,
and certainly against common practices. Many engineers like to
implement first and think about verification afterwards.

But if you think about it, it makes a lot of sense to deal with
verification first. Verification is about the requirements only --- so
your thoughts are not yet cluttered with implementation details. The
unit tests are an executable description of the requirements, so they
will be better understood and it will be very clear what needs to be
done. Consequently, the implementation should go smoother. Perhaps
most importantly, the test is available when you are done
implementing, and can be run anytime by anybody to verify changes.

Python has a standard \code{unittest} module that facilitates writing,
managing and running unit tests. With \code{unittest}, a test case is 
written by creating a class that inherits from
\code{unittest.TestCase}. Individual tests are created by methods of
that class: all method names that start with \code{test} are
considered to be tests of the test case.

We will define a test case for the Gray code properties, and then
write a test for each of the requirements. The outline of the test
case class is as follows:

\begin{verbatim}
from unittest import TestCase

class TestGrayCodeProperties(TestCase):

    def testSingleBitChange(self):
     """ Check that only one bit changes in successive codewords """
     ....


    def testUniqueCodeWords(self):
        """ Check that all codewords occur exactly once """
    ....
\end{verbatim}

Each method will be a small test bench that tests a single
requirement. To write the tests, we don't need an implementation of
the Gray encoder, but we do need the interface of the design. We can
specify this by a dummy implementation, as follows:

\begin{verbatim}
def bin2gray(B, G, width):
    ### NOT IMPLEMENTED YET! ###
    yield None
\end{verbatim}

For the first requirement, we will write a test bench that applies all
consecutive input numbers, and compares the current output with the
previous one for each input. Then we check that the difference is a
single bit. We will test all Gray codes up to a certain order
\code{MAX_WIDTH}.

\begin{verbatim}
    def testSingleBitChange(self):
        """ Check that only one bit changes in successive codewords """
        
        def test(B, G, width):
            B.next = intbv(0)
            yield delay(10)
            for i in range(1, 2**width):
                G_Z.next = G
                B.next = intbv(i)
                yield delay(10)
                diffcode = bin(G ^ G_Z)
                self.assertEqual(diffcode.count('1'), 1)
        
        for width in range(1, MAX_WIDTH):
            B = Signal(intbv(-1))
            G = Signal(intbv(0))
            G_Z = Signal(intbv(0))
            dut = bin2gray(B, G, width)
            check = test(B, G, width)
            sim = Simulation(dut, check)
            sim.run(quiet=1)
\end{verbatim}

Note how the actual check is performed by a \code{self.assertEqual}
method, defined by the \code{unittest.TestCase} class.

Similarly, we write a test bench for the second requirement. Again, we
simulate all numbers, and put the result in a list. The requirement
implies that if we sort the result list, we should get a range of
numbers:

\begin{verbatim}
    def testUniqueCodeWords(self):
        """ Check that all codewords occur exactly once """

        def test(B, G, width):
            actual = []
            for i in range(2**width):
                B.next = intbv(i)
                yield delay(10)
                actual.append(int(G))
            actual.sort()
            expected = range(2**width)
            self.assertEqual(actual, expected)
       
        for width in range(1, MAX_WIDTH):
            B = Signal(intbv(-1))
            G = Signal(intbv(0))
            dut = bin2gray(B, G, width)
            check = test(B, G, width)
            sim = Simulation(dut, check)
            sim.run(quiet=1)
\end{verbatim}


\subsection{Test-driven implementation}

With the test written, we begin with the implementation. For
illustration purposes, we will intentionally write some incorrect
implementations to see how the test behaves.

The easiest way to run tests defined with the \code{unittest}
framework, is to put a call to its \code{main} method at the end of
the test module:

\begin{verbatim}
unittest.main()
\end{verbatim}

Let's run the test using the dummy Gray encoder shown earlier:

\begin{verbatim}
% python test_gray.py -v
Check that only one bit changes in successive codewords ... FAIL
Check that all codewords occur exactly once ... FAIL
<trace backs not shown>
\end{verbatim}

As expected, this fails completely. Let us try an incorrect
implementation, that puts the lsb of in the input on the output:

\begin{verbatim}
def bin2gray(B, G, width):
    ### INCORRECT - DEMO PURPOSE ONLY! ###
    while 1:
        yield B
        G.next = B[0]
\end{verbatim}


Running the test produces:

\begin{verbatim}
% python test_gray.py -v
Check that only one bit changes in successive codewords ... ok
Check that all codewords occur exactly once ... FAIL

======================================================================
FAIL: Check that all codewords occur exactly once
----------------------------------------------------------------------
Traceback (most recent call last):
  File "test_gray.py", line 109, in testUniqueCodeWords
    sim.run(quiet=1)
...
  File "test_gray.py", line 104, in test
    self.assertEqual(actual, expected)
  File "/usr/local/lib/python2.2/unittest.py", line 286, in failUnlessEqual
    raise self.failureException, \
AssertionError: [0, 0, 1, 1] != [0, 1, 2, 3]

----------------------------------------------------------------------
Ran 2 tests in 0.785s
\end{verbatim}

Now the test passes the first requirement, as expected, but fails the
second one. After the test feedback, a full traceback is shown that
can help to debug the test output.

Finally, if we use the correct implementation as in
section~\ref{gray}, the output is:

\begin{verbatim}
% python test_gray.py -v
Check that only one bit changes in successive codewords ... ok
Check that all codewords occur exactly once ... ok

----------------------------------------------------------------------
Ran 2 tests in 6.364s

OK
\end{verbatim}



\subsection{Changing requirements}

In the previous section, we concentrated on the general requirements
of a Gray code. It is possible to specify these without specifying the
actual code. It is easy to see that there are several codes
that satisfy these requirements. In good XP style, we only tested
the requirements and nothing more.

It may be that more control is needed. For example, the requirement
may be for a particular code, instead of compliance with general
properties. As an illustration, we will show how to test for
\emph{the} original Gray code, which is one specific instance that
satisfies the requirements of the previous section. In this particular
case, this test will actually be easier than the previous one.

We denote the original Gray code of order \code{n} as \code{Ln}. Some
examples: 

\begin{verbatim}
L1 = ['0', '1']
L2 = ['00', '01', '11', '10']
L3 = ['000', '001', '011', '010', '110', '111', '101', 100']
\end{verbatim}

It is possible to specify these codes by a recursive algorithm, as
follows:

\begin{enumerate}
\item L1 = ['0', '1']
\item Ln+1 can be obtained from Ln as follows. Create a new code Ln0 by
prefixing all codewords of Ln with '0'. Create another new code Ln1 by
prefixing all codewords of Ln with '1', and reversing their
order. Ln+1 is the concatenation of Ln0 and Ln1.
\end{enumerate}

Python is well-known  for its elegant algorithmic
descriptions, and this is a good example. We can write the algorithm
in Python as follows:

\begin{verbatim}
def nextLn(Ln):
    """ Return Gray code Ln+1, given Ln. """
    Ln0 = ['0' + codeword for codeword in Ln]
    Ln1 = ['1' + codeword for codeword in Ln]
    Ln1.reverse()
    return Ln0 + Ln1
\end{verbatim}

The code \samp{['0' + codeword for ...]} is called a \dfn{list
comprehension}. It is a concise way to describe lists built by short
computations in a for loop.

The requirement is now that the output code matches the
expected code Ln. We use the \code{nextLn} function to compute the
expected result. The new test case code is as follows:

\begin{verbatim}
class TestOriginalGrayCode(TestCase):

    def testOriginalGrayCode(self):
        """ Check that the code is an original Gray code """

        Rn = []
        
        def stimulus(B, G, n):
            for i in range(2**n):
                B.next = intbv(i)
                yield delay(10)
                Rn.append(bin(G, width=n))
        
        Ln = ['0', '1'] # n == 1
        for n in range(2, MAX_WIDTH):
            Ln = nextLn(Ln)
            del Rn[:]
            B = Signal(intbv(-1))
            G = Signal(intbv(0))
            dut = bin2gray(B, G, n)
            stim = stimulus(B, G, n)
            sim = Simulation(dut, stim)
            sim.run(quiet=1)
            self.assertEqual(Ln, Rn)
\end{verbatim}

As it happens, our implementation is apparently an original Gray code:

\begin{verbatim}
% python test_gray.py -v TestOriginalGrayCode
Check that the code is an original Gray code ... ok

----------------------------------------------------------------------
Ran 1 tests in 3.091s

OK
\end{verbatim}


 


\chapter{MyHDL as a hardware verification language}

\section{Introduction}

One of the most exciting possibilities of \myhdl\
is to use it as a hardware verification language (HVL).
A HVL is a language used to write test benches and
verification environments, and to control simulations.

Nowadays, it is generally acknowledged that HVLs should be equipped
with modern software techniques, such as object orientation. The
reason is that verification it the most complex and time-consuming
task of the design process: consequently every useful technique is
welcome. Moreover, test benches are not required to be
implementable. Therefore, unlike synthesizable code, there
are no constraints on creativity.

Technically, verification of a design implemented in
another language requires cosimulation. \myhdl\ is 
enabled for cosimulation with any HDL simulator that
has a procedural language interface (PLI). The \myhdl\
side is designed to be independent of a particular
simulator, On the other hand, for each HDL simulator a specific
PLI module will have to be written in C. Currently,
the \myhdl\ release contains a PLI module to interface
to the Icarus Verilog simulator. This interface will
be used in the examples.

\section{The HDL side}

To introduce cosimulation, we will continue to use the Gray encoder
example from the previous chapters. Suppose that we want to
synthesize it and write it in Verilog for that purpose. Clearly we would
like to reuse our unit test verification environment. This is exactly
what \myhdl\ offers.

To start, let's recall how the Gray encoder in \myhdl{} looks like:

\begin{verbatim}
def bin2gray(B, G, width):
    """ Gray encoder.

    B -- input intbv signal, binary encoded
    G -- output intbv signal, gray encoded
    width -- bit width
    """
    while 1:
        yield B
        for i in range(width):
            G.next[i] = B[i+1] ^ B[i]

\end{verbatim}

To show the cosimulation flow, we don't need the Verilog
implementation yet, but only the interface.  Our Gray encoder in
Verilog would have the following interface:

\begin{verbatim}
module bin2gray(B, G);

   parameter width = 8;
   input [width-1:0]  B;     
   output [width-1:0] G;
   ....

\end{verbatim}

To write a test bench, one creates a new module that instantiates the
design under test (DUT).  The test bench declares nets and
regs (or signals in VHDL) that are attached to the DUT, and to
stimulus generators and response checkers. In an all-HDL flow, the
generators and checkers are written in the HDL itself, but we will
want to write them in \myhdl{}. To make the connection, we need to
declare which regs \& nets are driven and read by the \myhdl\
simulator. For our example, this is done as follows:

\begin{verbatim}
module dut_bin2gray;

   reg [`width-1:0] B;
   wire [`width-1:0] G;

   initial begin
      $from_myhdl(B);
      $to_myhdl(G);
   end

   bin2gray dut (.B(B), .G(G));
   defparam dut.width = `width;

endmodule

\end{verbatim}

The \code{\$from_myhdl} task call declares which regs are driven by
\myhdl{}, and the \code{\$to_myhdl} task call which regs \& nets are read
by it. These tasks take an arbitrary number of arguments.  They
are defined in a PLI module written in C. They are made available to
the simulation in a simulator-dependent manner.  In Icarus Verilog,
the tasks are defined in a \code{myhdl.vpi} module that is compiled
from C source code.

\section{The \myhdl\ side}

\myhdl\ supports cosimulation by a \code{Cosimulation} object. 
A \code{Cosimulation} object must know how to run a HDL cosimulation.
Therefore, the first argument to its constructor is a command string
to execute a simulation. The way to generate and run an
simulation executable is simulator dependent.
For example, in Icarus Verilog, a simulation executable for our
example can be obtained obtained by running the \code{iverilog}
compiler as follows:

\begin{verbatim}
% iverilog -o bin2gray -Dwidth=4 bin2gray.v dut_bin2gray.v

\end{verbatim}

This generates a \code{bin2gray} executable for a parameter \code{width}
of 4, by compiling the contributing verilog files.

The simulation itself is run by the \code{vvp} command:

\begin{verbatim}
% vvp -m ./myhdl.vpi bin2gray

\end{verbatim}

This runs the \code{bin2gray} simulation, and specifies to use the
\code{myhdl.vpi} PLI module present in the current directory. (This is 
just a command line usage example; actually simulating with the
\code{myhdl.vpi} module is only meaningful from a
\code{Cosimulation} object.)

We can use a \code{Cosimulation} object to provide a HDL cosimulation
version of a design to the \myhdl\ simulator. Instead of a generator
function, we write a function that returns a \code{Cosimulation}
object. For our example and the Icarus Verilog simulator, this is done
as follows:

\begin{verbatim}
import os

from myhdl import Cosimulation

cmd = "iverilog -o bin2gray -Dwidth=%s bin2gray.v dut_bin2gray.v"
      
def bin2gray(B, G, width):
    os.system(cmd % width)
    return Cosimulation("vvp -m ./myhdl.vpi bin2gray", B=B, G=G)

\end{verbatim}

After the executable command argument, the \code{Cosimulation}
constructor takes an arbitrary number of keyword arguments. Those
arguments make the link between \myhdl\ Signals and HDL nets, regs, or
signals, by named association. The keyword is the name of the argument
in a \code{\$to_myhdl} or \code{\$from_myhdl} call; the argument is
the \myhdl\ Signal.

With all this in place, we can now use the existing unit test
to verify the Verilog implementation. Note that we kept the
same name and parameters for the the \code{bin2gray} function:
all we need to do is to provide this alternative definition
to the existing unit test.

Let's quickly try it just to be sure:

\begin{verbatim}
module bin2gray(B, G);

   parameter width = 8;
   input [width-1:0]  B;
   output [width-1:0] G;
   reg [width-1:0] G;
   integer i;

   always @(B) begin
      for (i=0; i < width-1; i=i+1)
        G[i] <= B[i+1] ^ B[i];
   end

endmodule

\end{verbatim}

If we run our unit test we get:

\begin{verbatim}

% python test_bin2gray.py   
Check that only one bit changes in successive codewords ... ERROR
Check that all codewords occur exactly once ... FAIL
Check that the code is an original Gray code ... ERROR
...

\end{verbatim}

Oops! It seems we still have a bug! Oh yes, but of course, 
we need to zero-extend the input to get the msb output bit
correctly:

\begin{verbatim}
module bin2gray(B, G);

   parameter width = 8;
   input [width-1:0]  B;
   output [width-1:0] G;
   reg [width-1:0] G;
   integer i;
   wire [width:0] extB;

   assign extB = {1'b0, B};

   always @(extB) begin
      for (i=0; i < width; i=i+1)
        G[i] <= extB[i+1] ^ extB[i];
   end

endmodule

\end{verbatim}

And now:

\begin{verbatim}
% python test_bin2gray.py 
Check that only one bit changes in successive codewords ... ok
Check that all codewords occur exactly once ... ok
Check that the code is an original Gray code ... ok

----------------------------------------------------------------------
Ran 3 tests in 2.729s

OK

\end{verbatim}


\section{Restrictions}

In the ideal case, it should be possible to simulate
any HDL description seamlessly with \myhdl{}. Moreover
the communicating signals at each side should act
transparently as a single one, enabling fully race free
operation.

For various reasons, it may not be possible or desirable
to achieve full generality. As anyone that has developed
applications with the Verilog PLI can testify, the
restrictions in a particular simulator, and the 
differences over various simulators, can be quite 
frustrating. Moreover, full generality may require
a disproportionate amount of development work compared
to a slightly less general solution that may
be sufficient for the target application.

Consequently, I have tried to achieve a solution
which is simple enough so that one can reasonably 
expect that any PLI-enabled simulator can support it,
and that is relatively easy to verify and maintain.
At the same time, the solution is sufficiently general 
to cover the target application space.

The result is a compromise that places certain restrictions
on the HDL code. In this section, these restrictions 
are presented.

\subsection{Only passive HDL can be cosimulated}

The most important restriction of the \myhdl\ cosimulation solution is
that only ``passive'' HDL can be cosimulated.  This means that the HDL
code should not contain any statements with time delays. In other
words, the \myhdl\ simulator should be the master of time; in
particular, any clock signal should be generated at the \myhdl\ side.

At first this may seem like an important restriction, but if one
considers the target application for cosimulation, it probably
isn't. 

\myhdl\ support cosimulations so that test benches for HDL
designs can be written in Python.
Let's consider the nature of the target HDL designs. For high-level,
behavioral models that are not intended for implementation, it should
come as no surprise that I would recommend to write them in \myhdl\
directly; that is exactly the target of the \myhdl\ effort. Likewise,
gate level designs with annotated timing are not the target
application: static timing analysis is a much better verification
method for such designs.

Rather, the targeted HDL designs are naturally models that are
intended for implementation.  Most likely, this will be through
synthesis. As time delays are meaningless in synthesizable code, the
restriction is compatible with the target application.

\subsection{Race sensitivity issues}

In a typical TTL code, some events cause other events to occur in the
same time step. For example, when a clock signal triggers some signals
may change in the same time step. For race-free operation, an HDL
must differentiate between such events within a time step. This is done
by the concept of ``delta'' cycles. In a fully general, race free
cosimulation, the cosimulators would communicate at the level of delta
cycles. However, in \myhdl\ cosimulation, this is not entirely the
case.

Delta cycles from the \myhdl\ simulator toward the HDL cosimulator are
preserved. However, in the opposite direction, they are not. The
signals changes are only returned to the \myhdl\ simulator after all delta
cycles have been performed in the HDL cosimulator.

What does this mean? Let's start with the good news. As explained in
the previous section, the logic of the \myhdl\ cosimulation implies
that clocks are generated at the \myhdl\ side.  \emph{When using a
\myhdl\ clock and its corresponding HDL signal directly as a clock,
cosimulation operation is race free.} In other words, the case
that most closely reflects the \myhdl\ cosimulation approach, is race free.

The situation is different when you want to use a signal driven by the
HDL (and the corresponding MyHDL signal) as a clock. 
Communication triggered by such a clock is not race free. The solution
is to treat such an interface as a chip interface instead of an RTL
interface.  For example, when data is triggered at positive clock
edges, it can safely be sampled at negative clock edges.
Alternatively, the \myhdl\ data signals can be declared with a delay
value, so that they are guaranteed to change after the clock
edge.


\section{Implementation notes}

\begin{quote}
\em
This section requires some knowledge of PLI terminology.
\end{quote}

Enabling a simulator for cosimulation requires a PLI module
written in C. In Verilog, the PLI is part of the ``standard''.
However, different simulators implement different versions 
and portions of the standard. Worse yet, the behavior of
certain PLI callbacks is not defined on some essential points. 
As a result, one should plan to write a specific PLI module
for any simulator.

The present release contains a PLI module for the 
open source Icarus simulator. I would like to add
modules for any popular simulator in the future,
either from external contributions, or by getting
access to them myself. The same holds for VHDL
simulators: it would be great to have an interface
to the Modelsim VHDL simulator.

This section documents
the current approach and status of the PLI module
implementation in Icarus, and some reflections
on future implementations in other simulators.

\subsection{Icarus Verilog}

To make cosimulation work, a specific type of PLI callback is
needed. The callback should be run when all pending events have been
processed, while allowing the creation of new events in the current
time step (e.g. by the \myhdl\ simulator).  In some Verilog simulators,
the \code{cbReadWriteSync} callback does exactly that. However,
in others, including Icarus, it does not. The callback's behavior is
not fully standardized; some simulators run the callback before
non-blocking assignment events have been processed.

Consequently, I had to look for a workaround. One half of the solution
is to use the \code{cbReadOnlySync} callback.  This callback runs
after all pending events have been processed.  However, it does not
permit to create new events in the current time step.  The second half
of the solution is to map \myhdl\ delta cycles onto Verilog time steps.
Note that there is some freedom here because of the restriction that
only passive HDL code can be cosimulated.

I chose to make the time granularity in the Verilog simulator a 1000
times finer than in the \myhdl{} simulator. For each \myhdl\ time step,
1000 Verilog time steps are available for  \myhdl\ delta cycles. In practice,
only a few delta cycles per time step should be needed. More than 1000
almost certainly indicates an error. This limit is checked at
run-time. The factor 1000 also makes it easy to distinguish ``real''
time from delta cycle time when printing out the Verilog time.

\subsection{Other Verilog simulators}

The Icarus module is written with VPI calls, which are provided by the
most recent generation of the Verilog PLI. Some simulators may only
support TF/ACC calls, requiring a complete redesign of the interface
module.

If the simulator supports VPI, the Icarus module should be reusable to
a large extent. However, it may be possible to improve on it.  The
workaround described in the previous section may not be necessary. In
some simulators, the \code{cbReadWriteSync} callback occurs after all
events (including non-blocking assignments) have been processed. In
that case, the functionality can be supported without a finer time
granularity in the Verilog simulator.

There are also Verilog standardization efforts underway to resolve the
ambiguity of the \code{cbReadWriteSync} callback. The solution will be
to introduce new, well defined callbacks. From reading some proposals,
I conclude that the \code{cbEndOfSimTime} callback would provide the
required functionality.

\subsection{VHDL}

It would be great to have an interface to the Modelsim VHDL
simulator. This will require a redesign from scratch with the
appropriate PLI.  One feature which I would like to keep if possible
is the way to declare the communicating signals.  In the Verilog
solution, it is not necessary to define and instantiate any special
entity (module). Rather, the participating signals can be declared
directly in the \code{to_myhdl} and \code{from_myhdl} task calls.


\chapter{Conversion to Verilog\label{conversion}}
\section{Introduction\label{conv-intro}}

MyHDL 0.4 provides a path to automatic implementation, by converting
a subset of MyHDL code into synthesizable Verilog code.

MyHDL aims to be a complete design language, for high level modeling,
verification, but also for implementation. However, prior to \myhdl\
0.4 a \myhdl\ user had to translate synthesizable code manually to
Verilog or VHDL. Needless to say, this is inconvenient. With \myhdl\
0.4, this manual step should no longer be necessary.  The automatic
conversion provides a direct path from Python to an FPGA or ASIC
implementation.

\section{Solution description\label{conv-solution}}

The solution works as follows. The hardware description should be
modeled in \myhdl\ style, and satisfy certain constraints that are
typical for implementation-oriented hardware modeling.  Subsequently,
such a design is converted to an equivalent model in the Verilog
language, using the function \function{toVerilog} from the \myhdl\
library. Finally, a third-party \emph{synthesis tool} is used to
convert the Verilog design into a gate implementation for an ASIC or
FPGA. There are a number of Verilog synthesis tools available, varying
in price, capabilities, and target implementation space.

The conversion does not start from source files, but from a design
that has been "elaborated" by the Python interpreter.  This has
important advantages. First, there are no restrictions on how to
describe structure, as all "structural" constructs and parameters are
processed before the conversion starts. Second, the work of the Python
interpreter is "reused". The converter uses the Python profiler to
track the interpreter's operation and to infer the design structure
and name spaces. It then selectively compiles pieces of source code
for additional analysis and for conversion. This is done using the
Python compiler package.

\section{Features\label{conv-features}}

\subsection{The design is converted after elaboration\label{conv-features-elab}}
\emph{Elaboration} refers to the initial processing of a hardware
description to achieve a representation of a design instance that is
ready for simulation or synthesis. In particular, structural
parameters and constructs are processed in this step. In \myhdl{}, the
Python interpreter itself is used for elaboration.  A
\class{Simulation} object is constructed with elaborated design
instances as arguments.  Likewise, the Verilog conversion works on an
elaborated design instance. The Python interpreter is thus used as
much as possible.

\subsection{The structural description can be arbitrarily complex and hierarchical\label{conv-features-struc}}
As the conversion works on an elaborated design instance, any modeling
constraints only apply to the leaf elements of the design structure,
that is, the co-operating generators. In other words, there are no
restrictions on the description of the design structure: Python's full
power can be used for that purpose. Also, the design hierarchy can be
arbitrarily deep.

\subsection{Generators are mapped to Verilog always or initial blocks\label{conv-features-gen}}
The converter analyzes the code of each generator and maps it
into a Verilog \code{always} blocks if possible, and to 
an \code{initial} block otherwise.
The converted Verilog design will be a flat
"net list of blocks".

\subsection{The Verilog interface is inferred from signal usage\label{conv-features-intf}}
In \myhdl{}, the input or output direction of interface signals
is not explicitly declared. The converter investigates signal usage
in the design hierarchy to infer whether a signal is used as an
input, output, or an internal signal. Internal signals are
given a hierarchical name in the Verilog output.

\subsection{Function calls are mapped to a unique Verilog function or task\label{conv-features-func}}
The converter analyzes function calls and function code to see if they
should be mapped to Verilog functions or to tasks. Python functions
are much more powerful than Verilog subprograms; for example, they are
inherently generic, and they can be called with named association.  To
support this power in Verilog, a unique Verilog function or task is
generated per Python function call.

\subsection{If-then-else structures may be mapped to Verilog case statements\label{conv-features-if}}
Python does not provide a case statement. However, 
the converter recognizes if-then-else structures in which a variable is
sequentially compared to items of an enumeration type, and maps
such a structure to a Verilog case statement with the appropriate
synthesis attributes.

\subsection{Choice of encoding schemes for enumeration types\label{conv-features-enum}}
The \function{enum} function in \myhdl\ returns an enumeration type. This
function takes an additional parameter \var{encoding} that specifies the
desired encoding in the implementation: binary, one hot, or one cold.
The Verilog converter generates the appropriate code.


\section{The convertible subset\label{conv-subset}}

\subsection{Introduction\label{conv-subset-intro}}

Unsurprisingly, not all Python code can be converted into Verilog. In
fact, there are very important restrictions.  As the goal of the
conversion functionality is implementation, this should not be a big
issue: anyone familiar with synthesis is used to similar restrictions
in the \emph{synthesizable subset} of Verilog and VHDL. The converter
attempts to issue clear error messages when it encounters a construct
that cannot be converted. 

In practice, the synthesizable subset usually refers to RTL synthesis,
which is by far the most popular type of synthesis today. There are
industry standards that define the RTL synthesis subset.  However,
those were not used as a model for the restrictions of the MyHDL
converter, but as a minimal starting point.  On that basis, whenever
it was judged easy or useful to support an additional feature, this
was done. For example, it is actually easier to convert while loops
than for loops even though they are not RTL-synthesizable.  As another
example, \keyword{print} is supported because it's so useful for
debugging, even though it's not synthesizable.  In summary, the
convertible subset is a superset of the standard RTL synthesis subset,
and supports synthesis tools with more advanced capabilities, such as
behavioral synthesis.

Recall that any restrictions only apply to the design post
elaboration.  In practice, this means that they apply only to the code
of the generators, that are the "leaf" functional blocks in a MyHDL
design.

\subsection{Coding style\label{conv-subset-style}}

A natural restriction on convertible code is that it should be
written in MyHDL style: cooperating generators, communicating through
signals, and with \code{yield} statements specifying wait points and resume
conditions.  Supported resume conditions are a signal edge, a signal
change, or a tuple of such conditions.

\subsection{Supported types\label{conv-subset-types}}

The most important restriction regards object types. Verilog is an
almost typeless language, while Python is strongly (albeit
dynamically) typed. The converter needs to infer the types of
variables and map them to Verilog types. Therefore, it does type
inferencing of object constructors and expressions.

Only a limited amount of types can be converted.
Python \class{int} and \class{long} objects are mapped to Verilog
integers. All other supported types are mapped to Verilog regs (or
wires), and therefore need to have a defined bit width. The supported
types are the Python \class{bool} type, the MyHDL \class{intbv} type,
and MyHDL enumeration types returned by function \function{enum}. The
latter objects can also be used as the base object of a
\class{Signal}. 

\class{intbv} objects need to be constructed so that a bit
width can be inferred. This can be done by specifying minimum
and maximum values, e.g. as follows:

\begin{verbatim}
index = intbv(0, min=0, max=2**N)
\end{verbatim}

Alternatively, a slice can be taken from an \class{intbv} object
as follows:

\begin{verbatim}
index = intbv(0)[N:]
\end{verbatim}

Such as slice returns a new \class{intbv} object, with minimum
value \code{0} , and maximum value \code{2**N}.


\subsection{Supported statements\label{conv-subset-statements}}

The following is a list of the statements that are supported by the
Verilog converter, possibly qualified with restrictions
or usage notes. Recall that
this list only applies to the design post elaboration: in practice,
this means it applies to the code of the generators that are the leaf
blocks in a design.

\begin{description}

\item[The \keyword{break} statement.]

\item[The \keyword{continue} statement.]

\item[The \keyword{def} statement.]

\item[The \keyword{for} statement.]
The only supported iteration scheme is iterating through sequences of
integers returned by built-in function \function{range} or \myhdl\
function \function{downrange}.  The optional \keyword{else} clause is
not supported.

\item[The \keyword{if} statement.]
\keyword{if}, \keyword{elif}, and \keyword{else} clauses
are fully supported.

\item[The \keyword{pass} statement.]

\item[The \keyword{print} statement.]
The only supported expression for printing is
a single literal string.
The string can be interpolated, but the format specifiers
are copied verbatim to the Verilog output.
Print to a file (with syntax \code{'>>'}) is not supported.

\item[The \keyword{raise} statement.]
This statement is mapped to Verilog statements
that end the simulation with an error message.

\item[The \keyword{return} statement.]

\item[The \keyword{yield} statement.] 
The yielded expression can be a signal, a signal edge
as specified by \myhdl\ functions \function{posedge}
or \function{negedge}, or a tuple of signals and
edge specifications.

\item[The \keyword{while} statement.]
The optional \keyword{else}
clause is not supported.

\end{description}

\section{Methodology notes\label{conv-meth}}

\subsection{Simulation\label{conv-meth-sim}}

In the Python philosophy, the run-time rules. The Python compiler
doesn't attempt to detect a lot of errors beyond syntax errors, which
given Python's ultra-dynamic nature would be an almost impossible task
anyway. To verify a Python program, one should run it, preferably
using unit testing to verify each feature.

The same philosophy should be used when converting a MyHDL description
to Verilog: make sure the simulation runs fine first. Although the
converter checks many things and attempts to issue clear error
messages, there is no guarantee that it does a meaningful job unless
the simulation runs fine.

\subsection{Conversion output verification\label{conv-meth-conv}}
It is always prudent to verify the converted Verilog output.
To make this task easier, the converter also generates a
test bench that makes it possible to simulate the Verilog
design using the Verilog co-simulation interface. This 
permits to verify the Verilog code with the same test
bench used for the \myhdl\ code. This is also how
the Verilog converter development is being verified.

\subsection{Assignment issues\label{conv-meth-assign}}

\subsubsection{Name assignment in Python\label{conv-meth-assign-python}}

Name assignment in Python is a different concept than in
many other languages. This point is very important for
effective modeling in Python, and even more so
for synthesizable \myhdl\ code. Therefore, the issues are
discussed here.

Consider the following name assignments:

\begin{verbatim}
a = 4
a = ``a string''
a = False
\end{verbatim}

In many languages, the meaning would be that an
existing variable \var{a} gets a number of different values.
In Python, such a concept of a variable doesn't exist. Instead,
assignment merely creates a new binding of a name to a
certain object, that replaces any previous binding.
So in the example, the name \var{a} is bound a 
number of different objects in sequence.

The Verilog converter has to investigate name
assignment and usage in \myhdl\ code, and to map
names to Verilog variables. To achieve that,
it tries to infer the type and possibly the
bit width of each expression that is assigned
to a name.

Multiple assignments to the same name can be supported if it can be
determined that a consistent type and bit width is being used in the
assignments. This can be done for boolean expressions, numeric
expressions, and enumeration type literals. In Verilog, the
corresponding name is mapped to a single bit \code{reg}, an
\code{integer} or a \code{reg} of the appropriate width, respectively.

In other cases, a single assignment should be used when an object is
created. Subsequent value changes are then achieved by modification of
an existing object.  This technique should be used for \class{Signal}
and \class{intbv} objects.

\subsubsection{Signal assignment\label{conv-meth-assign-signal}}

Signal assignment in \myhdl\ is implemented using attribute assignment
to attribute \code{next}.  Value changes are thus modeled by
modification of the existing object. The converter investigates the
\class{Signal} object to infer the type and bit width of the
corresponding Verilog variable.

\subsubsection{\class{intbv} objects\label{conv-meth-assign-intbv}}

Type \class{intbv} is likely to be the workhorse for synthesizable
modeling in \myhdl{}. An \class{intbv} instance behaves like a
(mutable) integer whose individual bits can be accessed and
modified. Also, it is possible to constrain its set of values. In
addition to error checking, this makes it possible to infer a bit
width, which is required for implementation.

In Verilog, an \class{intbv} instance will be mapped to a \code{reg}
with an appropriate width. As noted before, it is not possible
to modify its value using name assignment. In the following, we
will show how it can be done instead. Consider:

\begin{verbatim}
a = intbv(0)[8:]
\end{verbatim}

This is an \class{intbv} object with initial value \code{0} and
bit width 8. The change its value to \code{5}, we can use
slice assignment:

\begin{verbatim}
a[8:] = 5
\end{verbatim}

The same can be achieved by leaving the bit width unspecified, 
which has the meaning to change ``all'' bits:

\begin{verbatim}
a[:] = 5
\end{verbatim}

Often the new value will depend on the old one. For example,
to increment an \class{intbv} with the technique above:

\begin{verbatim}
a[:] = a + 1
\end{verbatim}

Python also provides \emph{augmented} assignment operators,
which can be used to implement in-place operations. These are supported
on \class{intbv} objects and by the converter, so that the increment
can also be done as follows:

\begin{verbatim}
a += 1
\end{verbatim}

\section{Converter usage\label{conv-usage}}

We will demonstrate the conversion process by showing some examples.

\subsection{A small design with a single generator\label{conv-usage-small}}

Consider the following MyHDL code for an incrementer module:

\begin{verbatim}
def inc(count, enable, clock, reset, n):
    """ Incrementer with enable.
    
    count -- output
    enable -- control input, increment when 1
    clock -- clock input
    reset -- asynchronous reset input
    n -- counter max value
    """
    def incProcess():
        while 1:
            yield posedge(clock), negedge(reset)
            if reset == ACTIVE_LOW:
                count.next = 0
            else:
                if enable:
                    count.next = (count + 1) % n
    return incProcess()
\end{verbatim}

In Verilog terminology, function \function{inc} corresponds to a
module, while generator function \function{incProcess}
roughly corresponds to an always block.

Normally, to simulate the design, we would "elaborate" an instance
as follows:

\begin{verbatim}
m = 8
n = 2 ** m
 
count = Signal(intbv(0)[m:])
enable = Signal(bool(0))
clock, reset = [Signal(bool()) for i in range(2)]

inc_inst = inc(count, enable, clock, reset, n=n)
\end{verbatim}

\code{inc_inst} is an elaborated design instance that can be simulated. To
convert it to Verilog, we change the last line as follows:

\begin{verbatim}
inc_inst = toVerilog(inc, count, enable, clock, reset, n=n)
\end{verbatim}

Again, this creates an instance that can be simulated, but as a side
effect, it also generates a Verilog module in file \file{inc_inst.v},
that is supposed to have identical behavior. The Verilog code
is as follows:

\begin{verbatim}
module inc_inst (
    count,
    enable,
    clock,
    reset
);

output [7:0] count;
reg [7:0] count;
input enable;
input clock;
input reset;


always @(posedge clock or negedge reset) begin: _MYHDL1_BLOCK
    if ((reset == 0)) begin
        count <= 0;
    end
    else begin
        if (enable) begin
            count <= ((count + 1) % 256);
        end
    end
end

endmodule
\end{verbatim}

You can see the module interface and the always block, as expected
from the MyHDL design. 

\subsection{Converting a generator directly\label{conv-usage-gen}}

It is also possible to convert a generator
directly. For example, consider the following generator function:

\begin{verbatim}
def bin2gray(B, G, width):
    """ Gray encoder.

    B -- input intbv signal, binary encoded
    G -- output intbv signal, gray encoded
    width -- bit width
    """
    Bext = intbv(0)[width+1:]
    while 1:
        yield B
        Bext[:] = B
        for i in range(width):
            G.next[i] = Bext[i+1] ^ Bext[i]
\end{verbatim}

As before, you can create an instance and convert to
Verilog as follows:

\begin{verbatim}
width = 8

B = Signal(intbv(0)[width:])
G = Signal(intbv(0)[width:])

bin2gray_inst = toVerilog(bin2gray, B, G, width)
 \end{verbatim}

The generate Verilog module is as follows:

\begin{verbatim}
module bin2gray_inst (
    B,
    G
);

input [7:0] B;
output [7:0] G;
reg [7:0] G;

always @(B) begin: _MYHDL1_BLOCK
    integer i;
    reg [9-1:0] Bext;
    Bext[9-1:0] = B;
    for (i=0; i<8; i=i+1) begin
        G[i] <= (Bext[(i + 1)] ^ Bext[i]);
    end
end

endmodule
\end{verbatim}

\subsection{A hierarchical design\label{conv-usage-hier}}
The hierarchy of convertible designs can be
arbitrarily deep.

For example, suppose we want to design an
incrementer with Gray code output. Using the
designs from previous sections, we can proceed
as follows:

\begin{verbatim}
def GrayInc(graycnt, enable, clock, reset, width):
    
    bincnt = Signal(intbv()[width:])
    
    INC_1 = inc(bincnt, enable, clock, reset, n=2**width)
    BIN2GRAY_1 = bin2gray(B=bincnt, G=graycnt, width=width)
    
    return INC_1, BIN2GRAY_1
\end{verbatim}

According to Gray code properties, only a single bit
will change in consecutive values. However, as the
\code{bin2gray} module is combinatorial, the output bits
may have transient glitches, which may not be desirable.
To solve this, let's create an additional level of
hierarchy an add an output register to the design.
(This will create an additional latency of a clock
cycle, which may not be acceptable, but we will
ignore that here.)

\begin{verbatim}
def GrayIncReg(graycnt, enable, clock, reset, width):
    
    graycnt_comb = Signal(intbv()[width:])
    
    GRAY_INC_1 = GrayInc(graycnt_comb, enable, clock, reset, width)
    
    def reg():
        while 1:
            yield posedge(clock)
            graycnt.next = graycnt_comb
    REG_1 = reg()
    
    return GRAY_INC_1, REG_1
\end{verbatim}

We can convert this hierarchical design as before:

\begin{verbatim}
width = 8
graycnt = Signal(intbv()[width:])
enable, clock, reset = [Signal(bool()) for i in range(3)]

GRAY_INC_REG_1 = toVerilog(GrayIncReg, graycnt, enable, clock, reset, width)
\end{verbatim}

The Verilog output module looks as follows:

\begin{verbatim}
module GRAY_INC_REG_1 (
    graycnt,
    enable,
    clock,
    reset
);

output [7:0] graycnt;
reg [7:0] graycnt;
input enable;
input clock;
input reset;

reg [7:0] graycnt_comb;
reg [7:0] _GRAY_INC_1_bincnt;

always @(posedge clock or negedge reset) begin: _MYHDL1_BLOCK
    if ((reset == 0)) begin
        _GRAY_INC_1_bincnt <= 0;
    end
    else begin
        if (enable) begin
            _GRAY_INC_1_bincnt <= ((_GRAY_INC_1_bincnt + 1) % 256);
        end
    end
end

always @(_GRAY_INC_1_bincnt) begin: _MYHDL4_BLOCK
    integer i;
    reg [9-1:0] Bext;
    Bext[9-1:0] = _GRAY_INC_1_bincnt;
    for (i=0; i<8; i=i+1) begin
        graycnt_comb[i] <= (Bext[(i + 1)] ^ Bext[i]);
    end
end

always @(posedge clock) begin: _MYHDL9_BLOCK
    graycnt <= graycnt_comb;
end

endmodule
\end{verbatim}

Note that the output is a flat ``net list of blocks'', and
that hierarchical signal names are generated as necessary.

\subsection{Optimizations for finite state machines\label{conv-usage-fsm}}
As often in hardware design, finite state machines deserve special attention.

In Verilog and VHDL, finite state machines are typically described
using case statements.  Python doesn't have a case statement, but the
converter recognizes particular if-then-else structures and maps them
to case statements. This optimization occurs when a variable whose
type is an enumerated type is sequentially tested against enumeration
items in an if-then-else structure. Also, the appropriate synthesis
pragmas for efficient synthesis are generated in the Verilog code.

As a further optimization, function \function{enum} was enhanced to support
alternative encoding schemes elegantly, using an additional parameter
'encoding'. For example:

\begin{verbatim}
t_State = enum('SEARCH', 'CONFIRM', 'SYNC', encoding="one_hot")
\end{verbatim}

The default encoding is \code{binary}; the other possibilities are \code{one_hot} and
\code{one_cold}. This parameter only affects the conversion output, not the
behavior of the type. Verilog case statements are optimized for an
efficient implementation according to the encoding. Note that in
contrast, a Verilog designer needs to make nontrivial code changes to
implement a different encoding scheme.

As an example, consider the following finite state machine, whose
state variable used the enumeration type defined above:

\begin{verbatim}
FRAME_SIZE = 8

def FramerCtrl(SOF, state, syncFlag, clk, reset_n):
    
    """ Framing control FSM.

    SOF -- start-of-frame output bit
    state -- FramerState output
    syncFlag -- sync pattern found indication input
    clk -- clock input
    reset_n -- active low reset
    
    """
    
    index = intbv(0, min=0, max=8) # position in frame
    while 1:
        yield posedge(clk), negedge(reset_n)
        if reset_n == ACTIVE_LOW:
            SOF.next = 0
            index[:] = 0
            state.next = t_State.SEARCH
        else:
            SOF.next = 0
            if state == t_State.SEARCH:
                index[:] = 0
                if syncFlag:
                    state.next = t_State.CONFIRM
            elif state == t_State.CONFIRM:
                if index == 0:
                    if syncFlag:
                        state.next = t_State.SYNC
                    else:
                        state.next = t_State.SEARCH
            elif state == t_State.SYNC:
                if index == 0:
                    if not syncFlag:
                        state.next = t_State.SEARCH
                SOF.next = (index == FRAME_SIZE-1)
            else:
                raise ValueError("Undefined state")
            index[:]= (index + 1) % FRAME_SIZE

\end{verbatim}

The conversion is done as before:

\begin{verbatim}
SOF = Signal(bool(0))
syncFlag = Signal(bool(0))
clk = Signal(bool(0))
reset_n = Signal(bool(1))
state = Signal(t_State.SEARCH)
framerctrl_inst = toVerilog(FramerCtrl, SOF, state, syncFlag, clk, reset_n)
\end{verbatim}

The Verilog output looks as follows:

\begin{verbatim}
module framerctrl_inst (
    SOF,
    state,
    syncFlag,
    clk,
    reset_n
);
output SOF;
reg SOF;
output [2:0] state;
reg [2:0] state;
input syncFlag;
input clk;
input reset_n;

always @(posedge clk or negedge reset_n) begin: _MYHDL1_BLOCK
    reg [3-1:0] index;
    if ((reset_n == 0)) begin
        SOF <= 0;
        index[3-1:0] = 0;
        state <= 3'b001;
    end
    else begin
        SOF <= 0;
        // synthesis parallel_case full_case
        casez (state)
            3'b??1: begin
                index[3-1:0] = 0;
                if (syncFlag) begin
                    state <= 3'b010;
                end
            end
            3'b?1?: begin
                if ((index == 0)) begin
                    if (syncFlag) begin
                        state <= 3'b100;
                    end
                    else begin
                        state <= 3'b001;
                    end
                end
            end
            3'b1??: begin
                if ((index == 0)) begin
                    if ((!syncFlag)) begin
                        state <= 3'b001;
                    end
                end
                SOF <= (index == (8 - 1));
            end
            default: begin
                $display("Verilog: ValueError(Undefined state)");
                $finish;
            end
        endcase
        index[3-1:0] = ((index + 1) % 8);
    end
end
endmodule
\end{verbatim}


\section{Known issues\label{conv-issues}}
\begin{description}

\item[Negative values of \class{intbv} instances are not supported.]
The \class{intbv} class is quite capable of representing negative
values. However, the \code{signed} type support in Verilog is
relatively recent and mapping to it may be tricky. In my judgment,
this is not the most urgent requirement, so
I decided to leave this for later.

\item[Verilog integers are 32 bit wide]
Usually, Verilog integers are 32 bit wide. In contrast, Python is
moving toward integers with undefined width. Python \class{int} 
and \class{long} variables are mapped to Verilog integers; so for values
larger than 32 bit this mapping is incorrect.

\item[Synthesis pragmas are specified as Verilog comments.] The recommended
way to specify synthesis pragmas in Verilog is through attribute
lists. However, my Verilog simulator (Icarus) doesn't support them
for \code{case} statements (to specify \code{parallel_case} and
\code{full_case} pragmas). Therefore, I still used the old
but deprecated method of synthesis pragmas in Verilog comments.

\item[Inconsistent place of the sensitivity list inferred from \code{always_comb}.]
The semantics of \code{always_comb}, both in Verilog and \myhdl{}, is to
have an implicit sensitivity list at the end of the code. However, this
may not be synthesizable. Therefore, the inferred sensitivity list is
put at the top of the corresponding \code{always} block.
This may cause inconsistent behavior at the start of the
simulation. The workaround is to create events at time 0.

\item[Non-blocking assignments to task arguments don't work.] 
I didn't get non-blocking (signal) assignments to task arguments to
work.  I don't know yet whether the issue is my own, a Verilog issue,
or an issue with my Verilog simulator Icarus. I'll need to check this
further.

\end{description}


\chapter{Reference manual}


\myhdl\ is implemented as a Python package called \code{myhdl}. This
chapter describes the objects that are exported by this package.

\section{The \class{Simulation} class}
\begin{classdesc}{Simulation}{arg \optional{, arg \moreargs}}
Class to construct a new simulation. Each argument is either be a
\myhdl\ generator, or a nested sequence of such generators. (A nested
sequence is defined as a sequence in which each item may itself be a
nested sequence.) See section~\ref{myhdl-generators} for the
definition of \myhdl\ generators and their interaction with a
\class{Simulation} object.
\end{classdesc}

A \class{Simulation} object has the following method:

\begin{methoddesc}[Simulation]{run}{\optional{duration}}
Run the simulation forever (by default) or for a specified duration.
\end{methoddesc}

\section{The \class{Signal} class}
\label{signal}
\begin{classdesc}{Signal}{val \optional{, delay}}
This class is used to construct a new signal and to initialize its
value to \var{val}. Optionally, a delay can be specified.
\end{classdesc}

A \class{Signal} object has the following attributes:

\begin{memberdesc}[Signal]{next}
Read-write attribute that represents the next value of the signal.
\end{memberdesc}

\begin{memberdesc}[Signal]{val}
Read-only attribute that represents the current value of the signal.

This attribute is always available to access the current value;
however in many practical case it will not be needed. Whenever there
is no ambiguity, the Signal object's current value is used
implicitly. In particular, all Python's standard numeric, bit-wise,
logical and comparison operators are implemented on a Signal object by
delegating to its current value. The exception is augmented
assignment. These operators are not implemented as they would break
the rule that the current value should be a read-only attribute. In
addition, when a Signal object is directly assigned to the \code{next}
attribute of another Signal object, its current value is assigned
instead.
\end{memberdesc}



\section{\myhdl\ generators and trigger objects}
\label{myhdl-generators}
\myhdl\ generators are standard Python generators with specialized
\keyword{yield} statements. In hardware description languages, the equivalent
statements are called \emph{sensitivity lists}. The general format
of \keyword{yield} statements in in \myhdl\ generators is:

\hspace{\leftmargin}\keyword{yield} \var{clause \optional{, clause ...}}

After a simulation object executes a \keyword{yield} statement, it
suspends execution of the generator. At the same time, each
\var{clause} is a \emph{trigger object} which defines the condition
upon which the generator should be resumed. However, per invocation of a
\keyword{yield} statement, the generator is resumed exactly once,
regardless of the number of clauses. This happens as soon as one
of the objects triggers; subsequent triggers are
neglected. (However, as a result of the resumption, it is possible
that the same \keyword{yield} statement is invoked again, and that a
subsequent trigger still triggers the generator.)

In this section, the trigger objects and their functionality will be
described. 

\begin{funcdesc}{posedge}{signal}
Return a trigger object that specifies that the generator should
resume on a rising edge on the signal. A rising edge means a change
from false to true.
\end{funcdesc}

\begin{funcdesc}{negedge}{signal}
Return a trigger object that specifies that the generator should
resume on a falling edge on the signal. A falling edge means a change
from true to false.
\end{funcdesc}

\begin{funcdesc}{delay}{t}
Return a trigger object that specifies that the generator should
resume after a delay \var{t}.
\end{funcdesc}

\begin{funcdesc}{join}{arg \optional{, arg \moreargs}}
Join a number of trigger objects together and return a joined
trigger object.  The effect is that the joined trigger object will
trigger when \emph{all} of its arguments have triggered.
\end{funcdesc}

In addition, some objects can directly function as trigger
objects. These are the objects of the following types:

\begin{datadesc}{Signal}
For the full description of the \class{Signal} class, see
section~\ref{signal}.

A signal is a trigger object. Whenever a signal changes value, the
generator is triggered.
\end{datadesc}

\begin{datadesc}{GeneratorType}
\myhdl\ generators can itself be used as trigger objects. 
This corresponds to spawning a new generator, while the original
generator waits for it to complete.  In other words, the original
generator is triggered when the spawned generator completes.
\end{datadesc}

In addition, as a special case, the Python \code{None} object can be
present in  a \code{yield} statement:

\begin{datadesc}{None}
This is the do-nothing trigger object. The generator immediately
resumes, as if no \code{yield} statement were present. This can be
useful if the \code{yield} statement also has generator clauses: those
generators are spawned, while the original generator resumes
immediately.

\end{datadesc}



\section{Miscellaneous objects}

The following objects can be convenient in \myhdl\ modeling.

\begin{excclassdesc}{StopSimulation}{}
Base exception that is caught by the \code{Simulation.run} method to
stop a simulation. Can be subclassed and raised in generator code.
\end{excclassdesc}

\begin{funcdesc}{now}{}
Return the current simulation time.
\end{funcdesc}

\begin{funcdesc}{downrange}{high \optional{, low=0}}
Generates a downward range list of integers. Modeled after the
standard \code{range} function, but works in the downward
direction. The returned interval is half-open, with the \var{high}
index not included. \var{low} is optional and defaults to zero.

This function is especially useful with the \class{intbv} class, that
also works with downward indexing.
\end{funcdesc}

\begin{funcdesc}{bin}{num \optional{, width}}
Return a representation as a bit string.  If \var{width} is provided,
and if it is larger than the width of the default representation, the
bit string is padded with the sign bit.

This function complements the standard Python conversion functions
\code{hex} and \code{oct}. A binary string representation is often
needed in hardware design.
\end{funcdesc}


\section{The \class{intbv} class}

\begin{classdesc}{intbv}{arg}
This class represents \class{int}-like objects with some additional
features that make it suitable for hardware design. The constructor
argument can be an \class{int}, a \class{long}, an \class{intbv} or a
bit string (a string with only '0's or '1's). For a bit string
argument, the value is calculated as \code{int(\var{bitstring}, 2)}. 
\end{classdesc}

Unlike \class{int} objects, \class{intbv} objects are mutable; this is
also the reason for their existence. Mutability is needed to support
assignment to indexes and slices, as is common in hardware design. For
the same reason, \class{intbv} is not a subclass from \class{int},
even though \class{int} provides most of the desired
functionality. (It is not possible to derive a mutable subtype from
an immutable base type.)

An \class{intbv} object supports the same comparison, numeric,
bitwise, logical, and conversion operations as \class{int} objects. See
\url{http://www.python.org/doc/current/lib/typesnumeric.html} for more
information on such operations. In all binary operations,
\class{intbv} objects can work together with \class{int} objects; in
those cases the return type is an \class{intbv} object.

In addition, \class{intbv} objects support indexing and slicing
operations:

\begin{tableiii}{clc}{code}{Operation}{Result}{Notes}
  \lineiii{\var{bv}[\var{i}]}
	  {item \var{i} of \var{bv}}
	  {(1)}
  \lineiii{\var{bv}[\var{i}] = \var{x}}  
	  {item \var{i} of \var{bv} is replaced by \var{x}} 
          {(1)}
  \lineiii{\var{bv}[\var{i}:\var{j}]} 
          {slice of \var{bv} from \var{i} downto \var{j}} 
          {(2)(3)}
  \lineiii{\var{bv}[\var{i}:\var{j}] = \var{t}} 
  	  {slice of \var{bv} from \var{i} downto \var{j} is replaced
          by \var{t}} 
          {(2)(4)}
\end{tableiii}

\begin{description}
\item[(1)] Indexing follows the most common hardware design
	  conventions: the lsb bit is the rightmost bit, and it has
	  index 0. This has the following desirable property: if the
	  \class{intbv} value is decomposed as a sum of powers of 2,
	  the bit with index \var{i} corresponds to the term
	  \code{2**i}.

\item[(2)] It follows from the indexing convention that slicing ranges
	  are downward, in contrast to standard Python. However, the
	  Python convention of half-open ranges is followed. In
	  accordance with standard Python, the high index is not
	  included. However, it is the \emph{leftmost} index in this
	  case. As in standard Python, this takes care of one-off
	  issues in many practical cases: in particular,
	  \code{bv[\var{i}:]} returns \var{i} bits;
	  \code{bv[\var{i}:\var{j}]} has \code{\var{i}-\var{j}}
	  bits. As \class{intbv} objects have no explicitly defined
	  bit width, the high index \var{j} has no default value and
	  cannot be omitted, while the low index \var{j} defaults to
	  \code{0}.

\item[(3)] The value returned from a slicing operation is always
	  positive; higher order bits are implicitly assumed to be
	  zero. The bit width is implicitly stored in the returned bit
	  width, so that the returned object can be used in
	  concatenations and as an iterator.

\item[(4)] In setting a slice, it is checked whether the slice is wide
	  enough to accept all significant bits of the value.
\end{description}

In addition, \class{intbv} objects support a concatenation method:

\begin{methoddesc}[intbv]{concat}{\optional{arg \moreargs}}
Concatenate the arguments to an \class{intbv} object. Naturally, the
concatenation arguments need to have a defined bit width. Therefore,
if they are \class{intbv} objects, they have to be the return values
of a slicing operation. Alternatively, they may be bit strings.

In contrast to all other arguments, the implicit \var{self} argument
doesn't need to have a defined bit with. This is due to the fact that
concatenation occurs at the lsb (rightmost) side.

It may be clearer to call this method as an unbound method with an
explicit first \class{intbv} argument.
\end{methoddesc}

In addition, an \class{intbv} object supports the iterator protocol. This
makes it possible to iterate over all its bits, from the high index to
index 0. This is only possible for \class{intbv} objects with a
defined bit width.













\documentclass{manual}
\usepackage{palatino}
\renewcommand{\ttdefault}{cmtt}
\renewcommand{\sfdefault}{cmss}
\newcommand{\myhdl}{\protect \mbox{MyHDL}}
\usepackage{graphicx}

\title{The \myhdl\ manual}

\author{Jan Decaluwe}
\authoraddress{
Email: \email{jan@jandecaluwe.com}
}

\date{December 19, 2005}	% XXX update before release!
\release{0.5b1}  		% software release, not documentation
\setreleaseinfo{}		% empty for final release
\setshortversion{0.5b1}		% major.minor only for software


\makeindex

\begin{document}

\maketitle

Copyright \copyright{} 2001 Python Software Foundation.
All rights reserved.

Copyright \copyright{} 2000 BeOpen.com.
All rights reserved.

Copyright \copyright{} 1995-2000 Corporation for National Research Initiatives.
All rights reserved.

Copyright \copyright{} 1991-1995 Stichting Mathematisch Centrum.
All rights reserved.

See the end of this document for complete license and permissions
information.


\begin{abstract}

\noindent

The goal of the \myhdl{} project is to empower hardware designers with
the elegance and simplicity of the Python language.

\myhdl{} is a free, open-source (LGPL) package for using Python as a
hardware description and verification language. Python is a very high
level language, and hardware designers can use its full power to model
and simulate their designs. Moreover, \myhdl{} can convert a design to
Verilog. In combination with an external synthesis tool, it provides a
complete path from Python to a silicon implementation.

\emph{Modeling}


Python's power and clarity make \myhdl{} an ideal solution for high level
modeling. Python is famous for enabling elegant solutions to complex
modeling problems. Moreover, Python is outstanding for rapid
application development and experimentation.

The key idea behind \myhdl{} is the use of Python generators to model
hardware concurrency. Generators are best described as resumable
functions. In \myhdl{}, generators are used in a specific way so that
they become similar to always blocks in Verilog or processes in VHDL.

A hardware module is modeled as a function that returns any number of
generators. This approach makes it straightforward to support features
such as arbitrary hierarchy, named port association, arrays of
instances, and conditional instantiation.

Furthermore, \myhdl{} provides classes that implement traditional
hardware description concepts. It provides a signal class to support
communication between generators, a class to support bit oriented
operations, and a class for enumeration types.

\emph{Simulation and Verification}

The built-in simulator runs on top of the Python interpreter. It
supports waveform viewing by tracing signal changes in a VCD file.

With \myhdl{}, the Python unit test framework can be used on hardware
designs. Although unit testing is a popular modern software
verification technique, it is not yet common in the hardware design
world, making it one more area in which \myhdl{} innovates.

\myhdl{} can also be used as hardware verification language for VHDL and
Verilog designs, by co-simulation with traditional HDL simulators.

\emph{Conversion to Verilog}

The converter to Verilog works on an instantiated design that has been
fully elaborated. Consequently, the original design structure can be
arbitrarily complex.

The converter automates certain tasks that are tedious or hard in
Verilog directly. Notable features are the possibility to choose
between various FSM state encodings based on a single attribute, the
mapping of certain high-level objects to RAM and ROM descriptions, and
the automated handling of signed arithmetic issues.



\end{abstract}

\tableofcontents

\chapter{Background information}

\section{Prerequisites}

You need a basic understanding of Python to use \myhdl\.
If you don't know Python, you will take comfort in knowing
that it is probably one of the easiest programming languages to
learn \footnote{You must be bored by such claims, but in Python's
case it's true.}. Learning Python is also one of the better time
investments that engineering professionals can make \footnote{I am not
biased.}.

For beginners, \url{http://www.python.org/doc/current/tut/tut.html} is
probably the best choice for an on-line tutorial. For alternatives,
see \url{http://www.python.org/doc/Newbies.html}.

A working knowledge of a hardware description language such as Verilog
or VHDL is helpful. Chances are that you know one of those anyway, if
you are interested in \myhdl{}.

\section{A small tutorial on generators}

Generators are a recent feature in Python. They were introduced in
Python 2.2, which is the most recent stable version at the time of
this writing. Therefore, there isn't a lot of tutorial material
available yet. Because generators are the key concept in
\myhdl{}, I include a small tutorial here.

Consider the following nonsensical function:

\begin{verbatim}
def function():
    for i in range(5):
        return i

\end{verbatim}

You can see why it doesn't make a lot of sense. As soon as the first
loop iteration is entered, the function returns:

\begin{verbatim}
>>> function()
0
\end{verbatim}

Returning is fatal for the function call. Further loop iterations
never get a chance, and nothing is left over from the function call
when it returns.

To change the function into a generator function, we replace
\keyword{return} with \keyword{yield}:

\begin{verbatim}
def generator():
    for i in range(5):
        yield i

\end{verbatim}

Now we get:

\begin{verbatim}
>>> generator()
<generator object at 0x815d5a8>

\end{verbatim}

When a generator function is called, it returns a generator object. A
generator object supports the iterator protocol, which is an expensive
way of saying that you can let it generate subsequent values by
calling its \function{next()} method:

\begin{verbatim}
>>> g = generator()
>>> g.next()
0
>>> g.next()
1
>>> g.next()
2
>>> g.next()
3
>>> g.next()
4
>>> g.next()
Traceback (most recent call last):
  File "<stdin>", line 1, in ?
StopIteration

\end{verbatim}

Now we can generate the subsequent values from the for
loop on demand, until they are exhausted. What happens is that the
\keyword{yield} statement is like a
\keyword{return}, except that it is non-fatal: the generator remembers
its state and the point in the code when it yielded. A higher order
agent can decide when to get a further value by calling the
generator's \function{next()} method. We say that generators are
\dfn{resumable functions}.

If you are familiar with hardware description languages, this may
ring a bell. In hardware simulations, there is also a higher order
agent, the Simulator tool, that interacts with such resumable
functions; they are called processes in VHDL and always blocks in
Verilog. Like in those languages, Python generators provide an elegant
and efficient method to model concurrency, without having to resort to
some form of threading.

The use of generators to model concurrency is the first key concept in
\myhdl{}. The second key concept is a related one: in \myhdl{}, the
yielded values are used to define the condition upon which the generator
should be resumed. In other words, the \keyword{yield} statements work
as generalized sensitivity lists. If by now you are still interested,
read on to learn more!

If you want to know more about generators, consult the on-line Python
documentation, e.g. at \url{http://www.python.org/doc/2.2.2/whatsnew}. 

\begin{notice}[warning]
At the beginning of this section I said that generators were
introduced in Python 2.2. This is not entirely correct: in fact,
generators will only be enabled as a standard feature in Python 2.3.
However, a stable version of Python 2.3 has not been released yet at
the time of this writing. So, what to do?

Fortunately, Python lets you import features from its future releases
(provided that the future is not too distant). So, until you use
Python 2.3 or higher, you have to include the following line at the
start of all code that defines generator functions:

\begin{verbatim}
from __future__ import generators

\end{verbatim}

From Python 2.3 on, this line may still be in the code, though it
will not have an effect anymore.
\end{notice}


\chapter{Introduction to \myhdl\ \label{intro}}

\section{A basic \myhdl\ simulation \label{intro-basic}}

We will introduce \myhdl\ with a classic \code{Hello World} style
example. All example code can be found in the distribution directory
under \file{example/manual}.  Here are the contents of a \myhdl\
simulation script called \file{hello1.py}:

\begin{verbatim}
from myhdl import Signal, delay, always, now, Simulation

def HelloWorld():
    
    @always(delay(10))
    def sayHello():
        print "%s Hello World!" % now()

    return sayHello

inst = HelloWorld()
sim = Simulation(inst)
sim.run(30)
\end{verbatim}

When we run this simulation, we get the following output:

\begin{verbatim}
% python hello1.py
10 Hello World!
20 Hello World!
30 Hello World!
_SuspendSimulation: Simulated 30 timesteps
\end{verbatim}

The first line of the script imports a number of objects from the
\code{myhdl} package. In Python we can only use identifiers that are
literally defined in the source file 
\footnote{The exception is the \samp{from module import *} syntax,
that imports all the symbols from a module. Although this is generally
considered bad practice, it can be tolerated for large modules that
export a lot of symbols. One may argue that
\code{myhdl} falls into that category.}.

Then, we define a function called \function{HelloWorld}. In MyHDL,
classic functions are used to model hardware modules. In particular,
the parameter list is used to define the interface. In this first
example, the interface is empty.

Inside the top level function we declared a local function called
\function{sayHello} that defines the desired behavior. This function
is decorated with an \function{@always} decorator that has a delay
object as its parameter.  The meaning is that the function will be
executed whenever the specified delay has expired.

Behind the curtains, the \function{@always} decorator creates a Python
\emph{generator} and reuses the name of the decorated function for
it. Generators are the fundamental objects in MyHDL, and we will say
much more about them further on.

Finally, the top level function returns the local generator. This code
pattern is the simplest incarnation of the basic MyHDL code pattern
to define the contents of a hardware module. We will describe the
general case further on.

In MyHDL, we create an \emph{instance} of a hardware module by calling
the corresponding function. In the example, variable \code{inst} refers
to an instance of \function{HelloWorld}.  To simulate the instance, we
pass it as an argument to a \class{Simulation} object constructor.  We
then run the simulation for the desired amount of timesteps.

\section{Signals, ports, and concurrency \label{intro-conc}}

In the previous section, we simulated a design that consisted
of a single generator. Of course,
real hardware descriptions are not like that: they are
typically massively concurrent. \myhdl\ supports this by allowing an
arbitrary number of concurrent generators. 

With concurrency comes the problem of deterministic
communication. Hardware languages use special objects to
support deterministic communication between concurrent code.
For this purpose \myhdl\
has a \class{Signal} object which is roughly modeled after VHDL
signals.

We will demonstrate the use of signals and the concept of concurrency
by extending and modifying our first example. We define two hardware
modules, one that drives a clock signal, and one that is sensitive
to a positive edge on a clock signal:


\begin{verbatim}
from myhdl import Signal, delay, always, now, Simulation


def ClkDriver(clk):

    @always(delay(10))
    def driveClk():
        clk.next = not clk

    return driveClk


def HelloWorld(clk):
    
    @always(clk.posedge)
    def sayHello():
        print "%s Hello World!" % now()

    return sayHello


clk = Signal(0)
clkdriver_inst = ClkDriver(clk)
hello_inst = HelloWorld(clk)
sim = Simulation(clkdriver_inst, hello_inst)
sim.run(50)
\end{verbatim}

The clock driver function \function{ClkDriver} has a
clock signal as its parameter. This is how a
\emph{port} is modelled in MyHDL. The function
defines a generator
that continuously toggles a clock signal after a certain delay.
A new value of a signal is specified by assigning to its
\code{next} attribute. This is the \myhdl\ equivalent of 
    \index{VHDL!signal assignment}%
the VHDL signal assignment and the 
    \index{Verilog!non-blocking assignment}%
Verilog non-blocking assignment.

The \function{HelloWorld} function is modified from the
first example. It now also takes a clock signal as parameter.
Its generator is made sensitive to a rising
    \index{wait!for a rising edge}%
edge of the clock signal. This is specified by the
\code{posedge} attribute of a signal. The edge
specifier is the argument of the \code{@always}
decorator. As a result, the decorated function
will be executed on every rising clock edge.

The \code{clk} signal is constructed with an initial value
\code{0}. When creating an instance of each to the two
hardware modules, the same clock signal is passed as
the argument. The result is that the two instances
are now connected through the clock signal.
The \class{Simulation} object is constructed with the
two instances.

When we run the simulation, we get:

\begin{verbatim}
% python hello2.py
10 Hello World!
30 Hello World!
50 Hello World!
_SuspendSimulation: Simulated 50 timesteps
\end{verbatim}


\section{Parameters and hierarchy \label{intro-hier}}

We have seen that MyHDL uses functions to model hardware
modules. We have also seen that ports are modeled by using
signals as parameters. To make designs reusable we will also
want to use other objects as parameters. For example, we can
change the clock generator function to make it more general
and reusable, by making the clock period parametrizable, as
follows:

\begin{verbatim}
from myhdl import Signal, delay, instance, always, now, Simulation

def ClkDriver(clk, period=20):
    
    lowTime = int(period/2)
    highTime = period - lowTime

    @instance
    def driveClk():
        while True:
            yield delay(lowTime)
            clk.next = 1
            yield delay(highTime)
            clk.next = 0

    return driveClk
\end{verbatim}

In addition to the clock signal, the clock
\var{period} is a parameter with a default value of \code{20}.

As the low time of the clock may differ from the high time in case of
an odd period, we cannot use the \function{@always} decorator with a
single delay value anymore. Instead, the \function{driveClk} function
is now a generator function with an explicit definition of the desired
behavior. You can see that \function{driveClk} is a generator function (as
opposed to a classic function) because it contains \code{yield}
statements.

When a generator function is called, it returns a generator object. In
fact, this is mainly what the \function{@instance} decorator does. It
is less sophisticated than the \function{@always} decorator,
but it can be used to create a generator from any local generator
function.

The \code{yield} statement is a general Python construct, but MyHDL
uses it in a dedicated way.  In MyHDL, it has a similar meaning as the
wait statement in VHDL: the statement suspends execution of a
generator, and its clauses specify the conditions on which the
generator should wait before resuming. In this case, the generator
waits for a certain delay.

Not that to make sure that the generator runs ``forever'', we wrap its
behavior in a \code{while True} loop.

Likewise, we can define a general \function{Hello} function as follows:

\begin{verbatim}
def Hello(clk, to="World!"):

    @always(clk.posedge)
    def sayHello():
        print "%s Hello %s" % (now(), to)

    return sayHello
\end{verbatim}


We can create any number of instances by calling the functions with
the appropriate parameters. Hierarchy can be modeled by defining the
instances in a higher-level function, and returning them.
This pattern can be repeated for an arbitrary number of
hierarhical levels. Consequently, the general definition
of a \myhdl\ \dfn{instance} is recursive: an instance
   \index{instance!defined}%
is either a sequence of instances, or a generator.

As an example, we will create a higher-level function with
four instances of the lower-level functions, and simulate it:

\begin{verbatim}
def greetings():

    clk1 = Signal(0)
    clk2 = Signal(0)
    
    clkdriver_1 = ClkDriver(clk1) # positional and default association
    clkdriver_2 = ClkDriver(clk=clk2, period=19) # named assocation 
    hello_1 = Hello(clk=clk1) # named and default association
    hello_2 = Hello(to="MyHDL", clk=clk2) # named assocation

    return clkdriver_1, clkdriver_2, hello_1, hello_2


inst = greetings()
sim = Simulation(inst)
sim.run(50)
\end{verbatim}

As in standard Python, positional or named parameter association can
be used in instantiations, or a mix of both\footnote{All positional
parameters have to go before any named parameter.}. All these styles
are demonstrated in the example above. As in hardware description
languages, named association can be very useful if there are a lot of
parameters, as the argument order in the call does not matter in that
case.

The simulation produces the following output:

\begin{verbatim}
% python greetings.py
9 Hello MyHDL
10 Hello World!
28 Hello MyHDL
30 Hello World!
47 Hello MyHDL
50 Hello World!
_SuspendSimulation: Simulated 50 timesteps
\end{verbatim}


\begin{notice}[warning]
Some commonly used terminology has different meanings
in Python versus hardware design. Rather than artificially
changing terminology, I think it's best to keep it
and explicitly describing the differences.

A \dfn{module} in Python refers to all source code
in a particular file. A module can be reused by
other modules by importing. In hardware design,
\index{module!in Python versus hardware design}%
a module is  a reusable block of hardware with
a well defined interface. It can be reused in 
another module by \dfn{instantiating} it.

An \dfn{instance} in Python (and other object-oriented
languages) refers to the object created by a
\index{instance!in Python versus hardware design}%
class constructor. In hardware design, an instance
is a particular incarnation of a hardware module.

Normally, the meaning should be clear from
the context. Occasionally, I may qualify terms 
with the words 'hardware' or '\myhdl{}' to 
avoid ambiguity.
\end{notice}


\section{Bit oriented operations \label{intro-bit}}

Hardware design involves dealing with bits and bit-oriented
operations. The standard Python type \class{int} has most of the
desired features, but lacks support for indexing and slicing. For this
reason, \myhdl\ provides the \class{intbv} class. The name was chosen
to suggest an integer with bit vector flavor.

Class \class{intbv} works transparently as an integer and with other
integer-like types. Like class \class{int}, it provides access to the
underlying two's complement representation for bitwise
operations. In addition, it is a mutable type that provides indexing
and slicing operations, and some additional bit-oriented support such
as concatenation.

\subsection{Bit indexing \label{intro-indexing}}
\index{bit indexing}

As an example, we will consider the design of a Gray encoder. The
following code is a Gray encoder modeled in \myhdl{}:

\begin{verbatim}
from myhdl import Signal, delay, Simulation, always_comb, instance, intbv, bin

def bin2gray(B, G, width):
    """ Gray encoder.

    B -- input intbv signal, binary encoded
    G -- output intbv signal, gray encoded
    width -- bit width
    """
    
    @always_comb
    def logic():
        for i in range(width):
            G.next[i] = B[i+1] ^ B[i]
            
    return logic
\end{verbatim}

This code introduces a few new concepts. The string in triple quotes
at the start of the function is a \dfn{doc string}. This is standard
Python practice for structured documentation of code.

Furthermore, we introduce a third decorator: \function{@always_comb}.
It is used with a classic function and specifies that the 
resulting generator should
    \index{wait!for a signal value change}%
wait for a value change on any input signal. This is typically used to
describe 
    \index{combinatorial logic}%
combinatorial logic. The \function{@always_comb} decorator
automatically infers which signals are used as inputs.

Finally, the code contains bit indexing operations and an exclusive-or
operator as required for a Gray encoder. By convention, the lsb of an
\class{intbv} object has index~\code{0}.

To verify the Gray encoder, we write a test bench that prints input
and output for all possible input values:

\begin{verbatim}
def testBench(width):
    
    B = Signal(intbv(0))
    G = Signal(intbv(0))
    
    dut = traceSignals(bin2gray, B, G, width)

    @instance
    def stimulus():
        for i in range(2**width):
            B.next = intbv(i)
            yield delay(10)
            print "B: " + bin(B, width) + "| G: " + bin(G, width)

    return dut, stimulus
\end{verbatim}

We use the conversion function \code{bin} to get a binary
string representation of the signal values. This function is exported
by the \code{myhdl} package and complements the standard Python
\code{hex} and \code{oct} conversion functions.

To demonstrate, we set up a simulation for a small width: 

\begin{verbatim}
sim = Simulation(testBench(width=3))
sim.run()
\end{verbatim}

The simulation produces the following output:

\begin{verbatim}
% python bin2gray.py
B: 000 | G: 000
B: 001 | G: 001
B: 010 | G: 011
B: 011 | G: 010
B: 100 | G: 110
B: 101 | G: 111
B: 110 | G: 101
B: 111 | G: 100
StopSimulation: No more events
\end{verbatim}

\subsection{Bit slicing \label{intro-slicing}}
\index{bit slicing}

For a change, we will use a plain function as an example to illustrate
slicing.  The following function calculates the HEC byte of an ATM
header.

\begin{verbatim}
from myhdl import intbv, concat

COSET = 0x55

def calculateHec(header):
    """ Return hec for an ATM header, represented as an intbv.

    The hec polynomial is 1 + x + x**2 + x**8.
    """
    hec = intbv(0)
    for bit in header[32:]:
        hec[8:] = concat(hec[7:2],
                         bit ^ hec[1] ^ hec[7],
                         bit ^ hec[0] ^ hec[7],
                         bit ^ hec[7]
                        )
    return hec ^ COSET
\end{verbatim}

The code shows how slicing access and assignment is supported on the
\class{intbv} data type. In accordance with the most common hardware
convention, and unlike standard Python, slicing ranges are
downward. The code also demonstrates concatenation of \class{intbv}
objects.

As in standard Python, the slicing range is half-open: the highest
index bit is not included. Unlike standard Python however, this index
corresponds to the \emph{leftmost} item. Both indices can be omitted
from the slice. If the leftmost index is omitted, the meaning is to
access ``all'' higher order bits.  If the rightmost index is omitted,
it is \code{0} by default.

The half-openness of a slice may seem awkward at first, but it helps
to avoid one-off count issues in practice. For example, the slice
\code{hex[8:]} has exactly \code{8} bits. Likewise, the slice
\code{hex[7:2]} has \code{7-2=5} bits. You can think about it as
follows: for a slice \code{[i:j]}, only bits below index \code{i} are
included, and the bit with index \code{j} is the last bit included.

When an intbv object is sliced, a new intbv object is returned. This
new intbv always has a positive value, even when the original object
was negative.

\section{Summary and perspective}


\section{Some remarks on \myhdl\ and Python \label{intro-python}}

To conclude this introductory chapter, it is useful to stress that
\myhdl\ is not a language in itself. The underlying language is Python, 
and \myhdl\ is implemented as a Python package called \code{myhdl}.
Moreover, it is a design goal to keep the \code{myhdl} package as
minimalistic as possible, so that \myhdl\ descriptions are very much
``pure Python''.

To have Python as the underlying language is significant in several
ways:

\begin{itemize}

\item Python is a very powerful high level language. This translates
into high productivity and elegant solutions to complex problems.

\item  Python is continuously improved by some very clever 
minds, supported by a large and fast growing user base. Python profits
fully from the open source development model.

\item Python comes with an extensive standard library. Some
functionality is likely to be of direct interest to \myhdl\ users:
examples include string handling, regular expressions, random number
generation, unit test support, operating system interfacing and GUI
development. In addition, there are modules for mathematics, database
connections, networking programming, internet data handling, and so
on.

\item Python has a powerful C extension model. All built-in types are
written with the same C API that is available for custom
extensions. To a module user, there is no difference between a
standard Python module and a C extension module --- except
performance. The typical Python development model is to prototype
everything in Python until the application is stable, and (only) then
rewrite performance critical modules in C if necessary.

\end{itemize}





\chapter{Modeling techniques}

\section{RTL modeling}
The present section describes how \myhdl\ supports RTL style modeling
as is typically used for synthesizable models in Verilog or VHDL.

In this domain, \myhdl\ doesn't offer advantages compared to
other solutions. However, as this modeling style is well-known,
this section may be useful for illustrative purposes.

\subsection{Combinatorial logic}

\subsubsection{Template}

Combinatorial logic is described with a generator function code template as
follows: 

\begin{verbatim}
def combinatorialLogic(<arguments>)
    while 1:
        yield <input signal arguments>
        <functional code>

\end{verbatim}

The overall code is wrapped in a \code{while 1} statement to keep the
generator alive. All input signals are clauses in the \code{yield}
statement, so that the generator resumes whenever one of the inputs
changes. 

\subsubsection{Example}

The following is an example of a combinatorial multiplexer:

\begin{verbatim}
def mux(z, a, b, sel):
    """ Multiplexer.
    
    z -- mux output
    a, b -- data inputs
    sel -- control input: select a if asserted, otherwise b
    """
    while 1:
        yield a, b, sel
        if sel == 1:
            z.next = a
        else:
            z.next = b
\end{verbatim}

To verify, let's simulate this logic with some random patterns. The
\code{random} module in Python's standard library comes in handy for
such purposes. The function \code{randrange(\var{n})} returns a random
natural integer smaller than \var{n}. It is used in the test bench
code to produce random input values:

\begin{verbatim}
from random import randrange

(z, a, b, sel) = [Signal(0) for i in range(4)]

MUX_1 = mux(z, a, b, sel)

def test():
    print "z a b sel"
    for i in range(8):
        a.next, b.next, sel.next = randrange(8), randrange(8), randrange(2)
        yield delay(10)
        print "%s %s %s %s" % (z, a, b, sel)
        
Simulation(MUX_1, test()).run() 
   
\end{verbatim}

Because of the randomness, the simulation output varies between runs
\footnote{It also possible to have a reproducible random output, by
explicitly providing a seed value. See the documentation of the
\code{random} module}. One particular run produced the following
output:

\begin{verbatim}
% python mux.py
z a b sel
6 6 1 1
7 7 1 1
7 3 7 0
1 2 1 0
7 7 5 1
4 7 4 0
4 0 4 0
3 3 5 1
StopSimulation: No more events
\end{verbatim}


\subsection{Sequential logic}

\subsubsection{Template}
Sequential RTL models are sensitive to a clock edge. In addition, they
may be sensitive to a reset signal. We will describe one of the most
common patterns: a template with a rising clock edge and an
asynchronous reset signal. Other templates are similar.

\begin{verbatim}
def sequentialLogic(<arguments>, clock, ..., reset, ...)
    while 1:
        yield posedge(clock), negedge(reset)
        if reset == <active level>:
            <reset code>
        else:
            <functional code>

\end{verbatim}


\subsubsection{Example}
The following code is a description of an incrementer with enable, and
an asynchronous power-up reset.

\begin{verbatim}
ACTIVE_LOW, INACTIVE_HIGH = 0, 1

def Inc(count, enable, clock, reset, n):
    """ Incrementer with enable.
    
    count -- output
    enable -- control input, increment when 1
    clock -- clock input
    reset -- asynchronous reset input
    n -- counter max value
    """
    while 1:
        yield posedge(clock), negedge(reset)
        if reset == ACTIVE_LOW:
            count.next = 0
        else:
            if enable:
                count.next = (count + 1) % n

\end{verbatim}

For the test bench, we will use an independent clock generator, stimulus
generator, and monitor. After applying enough stimulus patterns, we
can raise the \code{myhdl.StopSimulation} exception to stop the
simulation run. The test bench for a small incrementer and a small
number of patterns is a follows:

\begin{verbatim}
count, enable, clock, reset = [Signal(intbv(0)) for i in range(4)]

INC_1 = Inc(count, enable, clock, reset, n=4)

def clockGen():
    while 1:
        yield delay(10)
        clock.next = not clock

def stimulus():
    reset.next = ACTIVE_LOW
    yield negedge(clock)
    reset.next = INACTIVE_HIGH
    for i in range(12):
        enable.next = min(1, randrange(3))
        yield negedge(clock)
    raise StopSimulation

def monitor():
    print "enable  count"
    yield posedge(reset)
    while 1:
        yield posedge(clock)
        yield delay(1)
        print "   %s      %s" % (enable, count)
        
Simulation(clockGen(), stimulus(), monitor(), INC_1).run()

\end{verbatim}

The simulation produces the following output:
\begin{verbatim}
% python inc.py
enable  count
   0      0
   1      1
   0      1
   1      2
   1      3
   1      0
   0      0
   1      1
   0      1
   0      1
   0      1
   1      2
StopSimulation

\end{verbatim}


\section{High level modeling}

test

\chapter{Unit testing}

\section{Introduction}

Many aspects in the design flow of modern digital hardware design can
be viewed as a special kind of software development. From that
viewpoint, it is a natural question whether advances in software
design techniques can not also be applied to hardware design.

One software design approach that gets a lot of attention recently is
\emph{Extreme Programming} (XP). It is a fascinating set of techniques and
guidelines that often seems to go against the conventional wisdom. On
other occasions, XP just seems to emphasize the common sense, which
doesn't always coincide with common practice. For example, XP stresses
the importance of normal workweeks, if we are to have the
fresh mind needed for good software development.

It is not my intention nor qualification to present a tutorial on
Extreme Programming. Instead, in this section I will highlight one XP
concept which I think is very relevant to hardware design: the
importance and methodology of unit testing.

\section{The importance of unit tests}

Unit testing is one of the corner stones of Extreme Programming. Other
XP concepts, such as collective ownership of code and continuous
refinement, are only possible by having unit tests. Moreover, XP
emphasizes that writing unit tests should be automated, that they should
test everything in every class, and that they should run perfectly all
the time. 

I believe that these concepts apply directly to hardware design. In
addition, unit tests are a way to manage simulation time. For example,
a state machine that runs very slowly on infrequent events may be
impossible to verify at the system level, even on the fastest
simulator. On the other hand, it may be easy to verify it exhaustively
in a unit test, even on the slowest simulator.

It is clear that unit tests have compelling advantages. On the other
hand, if we need to test everything, we have to write
lots of unit tests. So it should be easy and pleasant
to create, manage and run them. Therefore, XP emphasizes the need for
a unit test framework that supports these tasks. In this chapter,
we will explore the use of the \code{unittest} module from
the standard Python library for creating unit tests for hardware
designs.


\section{Unit test development}

In this section, we will informally explore the application of unit
test techniques to hardware design. We will do so by a (small)
example: testing a binary to Gray encoder as introduced in
section~\ref{gray}. 

\subsection{Defining the requirements}

We start by defining the requirements. For a Gray encoder, we want to
the output to comply with Gray code characteristics. Let's define a
\dfn{code} as a list of \dfn{codewords}, where a codeword is a bit
string. A code of order \code{n} has \code{2**n} codewords.

A well-known characteristic is the one that Gray codes are all about:

\newtheorem{reqGray}{Requirement}
\begin{reqGray} 
Consecutive codewords in a Gray code should differ in a single bit.
\end{reqGray}

Is this sufficient? Not quite: suppose for example that an
implementation returns the lsb of each binary input. This would comply
with the requirement, but is obviously not what we want. Also, we don't
want the bit width of Gray codewords to exceed the bit width of the
binary codewords.

\begin{reqGray} 
Each codeword in a Gray code of order n must occur exactly once in the
binary code of the same order.
\end{reqGray}

With the requirements written down we can proceed.

\subsection{Writing the test first}

A fascinating guideline in the XP world is to write the unit test
first. That is, before implementing something, first write the test
that will verify it. This seems to go against our natural inclination,
and certainly against common practices. Many engineers like to
implement first and think about verification afterwards.

But if you think about it, it makes a lot of sense to deal with
verification first. Verification is about the requirements only --- so
your thoughts are not yet cluttered with implementation details. The
unit tests are an executable description of the requirements, so they
will be better understood and it will be very clear what needs to be
done. Consequently, the implementation should go smoother. Perhaps
most importantly, the test is available when you are done
implementing, and can be run anytime by anybody to verify changes.

Python has a standard \code{unittest} module that facilitates writing,
managing and running unit tests. With \code{unittest}, a test case is 
written by creating a class that inherits from
\code{unittest.TestCase}. Individual tests are created by methods of
that class: all method names that start with \code{test} are
considered to be tests of the test case.

We will define a test case for the Gray code properties, and then
write a test for each of the requirements. The outline of the test
case class is as follows:

\begin{verbatim}
from unittest import TestCase

class TestGrayCodeProperties(TestCase):

    def testSingleBitChange(self):
     """ Check that only one bit changes in successive codewords """
     ....


    def testUniqueCodeWords(self):
        """ Check that all codewords occur exactly once """
    ....
\end{verbatim}

Each method will be a small test bench that tests a single
requirement. To write the tests, we don't need an implementation of
the Gray encoder, but we do need the interface of the design. We can
specify this by a dummy implementation, as follows:

\begin{verbatim}
def bin2gray(B, G, width):
    ### NOT IMPLEMENTED YET! ###
    yield None
\end{verbatim}

For the first requirement, we will write a test bench that applies all
consecutive input numbers, and compares the current output with the
previous one for each input. Then we check that the difference is a
single bit. We will test all Gray codes up to a certain order
\code{MAX_WIDTH}.

\begin{verbatim}
    def testSingleBitChange(self):
        """ Check that only one bit changes in successive codewords """
        
        def test(B, G, width):
            B.next = intbv(0)
            yield delay(10)
            for i in range(1, 2**width):
                G_Z.next = G
                B.next = intbv(i)
                yield delay(10)
                diffcode = bin(G ^ G_Z)
                self.assertEqual(diffcode.count('1'), 1)
        
        for width in range(1, MAX_WIDTH):
            B = Signal(intbv(-1))
            G = Signal(intbv(0))
            G_Z = Signal(intbv(0))
            dut = bin2gray(B, G, width)
            check = test(B, G, width)
            sim = Simulation(dut, check)
            sim.run(quiet=1)
\end{verbatim}

Note how the actual check is performed by a \code{self.assertEqual}
method, defined by the \code{unittest.TestCase} class.

Similarly, we write a test bench for the second requirement. Again, we
simulate all numbers, and put the result in a list. The requirement
implies that if we sort the result list, we should get a range of
numbers:

\begin{verbatim}
    def testUniqueCodeWords(self):
        """ Check that all codewords occur exactly once """

        def test(B, G, width):
            actual = []
            for i in range(2**width):
                B.next = intbv(i)
                yield delay(10)
                actual.append(int(G))
            actual.sort()
            expected = range(2**width)
            self.assertEqual(actual, expected)
       
        for width in range(1, MAX_WIDTH):
            B = Signal(intbv(-1))
            G = Signal(intbv(0))
            dut = bin2gray(B, G, width)
            check = test(B, G, width)
            sim = Simulation(dut, check)
            sim.run(quiet=1)
\end{verbatim}


\subsection{Test-driven implementation}

With the test written, we begin with the implementation. For
illustration purposes, we will intentionally write some incorrect
implementations to see how the test behaves.

The easiest way to run tests defined with the \code{unittest}
framework, is to put a call to its \code{main} method at the end of
the test module:

\begin{verbatim}
unittest.main()
\end{verbatim}

Let's run the test using the dummy Gray encoder shown earlier:

\begin{verbatim}
% python test_gray.py -v
Check that only one bit changes in successive codewords ... FAIL
Check that all codewords occur exactly once ... FAIL
<trace backs not shown>
\end{verbatim}

As expected, this fails completely. Let us try an incorrect
implementation, that puts the lsb of in the input on the output:

\begin{verbatim}
def bin2gray(B, G, width):
    ### INCORRECT - DEMO PURPOSE ONLY! ###
    while 1:
        yield B
        G.next = B[0]
\end{verbatim}


Running the test produces:

\begin{verbatim}
% python test_gray.py -v
Check that only one bit changes in successive codewords ... ok
Check that all codewords occur exactly once ... FAIL

======================================================================
FAIL: Check that all codewords occur exactly once
----------------------------------------------------------------------
Traceback (most recent call last):
  File "test_gray.py", line 109, in testUniqueCodeWords
    sim.run(quiet=1)
...
  File "test_gray.py", line 104, in test
    self.assertEqual(actual, expected)
  File "/usr/local/lib/python2.2/unittest.py", line 286, in failUnlessEqual
    raise self.failureException, \
AssertionError: [0, 0, 1, 1] != [0, 1, 2, 3]

----------------------------------------------------------------------
Ran 2 tests in 0.785s
\end{verbatim}

Now the test passes the first requirement, as expected, but fails the
second one. After the test feedback, a full traceback is shown that
can help to debug the test output.

Finally, if we use the correct implementation as in
section~\ref{gray}, the output is:

\begin{verbatim}
% python test_gray.py -v
Check that only one bit changes in successive codewords ... ok
Check that all codewords occur exactly once ... ok

----------------------------------------------------------------------
Ran 2 tests in 6.364s

OK
\end{verbatim}



\subsection{Changing requirements}

In the previous section, we concentrated on the general requirements
of a Gray code. It is possible to specify these without specifying the
actual code. It is easy to see that there are several codes
that satisfy these requirements. In good XP style, we only tested
the requirements and nothing more.

It may be that more control is needed. For example, the requirement
may be for a particular code, instead of compliance with general
properties. As an illustration, we will show how to test for
\emph{the} original Gray code, which is one specific instance that
satisfies the requirements of the previous section. In this particular
case, this test will actually be easier than the previous one.

We denote the original Gray code of order \code{n} as \code{Ln}. Some
examples: 

\begin{verbatim}
L1 = ['0', '1']
L2 = ['00', '01', '11', '10']
L3 = ['000', '001', '011', '010', '110', '111', '101', 100']
\end{verbatim}

It is possible to specify these codes by a recursive algorithm, as
follows:

\begin{enumerate}
\item L1 = ['0', '1']
\item Ln+1 can be obtained from Ln as follows. Create a new code Ln0 by
prefixing all codewords of Ln with '0'. Create another new code Ln1 by
prefixing all codewords of Ln with '1', and reversing their
order. Ln+1 is the concatenation of Ln0 and Ln1.
\end{enumerate}

Python is well-known  for its elegant algorithmic
descriptions, and this is a good example. We can write the algorithm
in Python as follows:

\begin{verbatim}
def nextLn(Ln):
    """ Return Gray code Ln+1, given Ln. """
    Ln0 = ['0' + codeword for codeword in Ln]
    Ln1 = ['1' + codeword for codeword in Ln]
    Ln1.reverse()
    return Ln0 + Ln1
\end{verbatim}

The code \samp{['0' + codeword for ...]} is called a \dfn{list
comprehension}. It is a concise way to describe lists built by short
computations in a for loop.

The requirement is now that the output code matches the
expected code Ln. We use the \code{nextLn} function to compute the
expected result. The new test case code is as follows:

\begin{verbatim}
class TestOriginalGrayCode(TestCase):

    def testOriginalGrayCode(self):
        """ Check that the code is an original Gray code """

        Rn = []
        
        def stimulus(B, G, n):
            for i in range(2**n):
                B.next = intbv(i)
                yield delay(10)
                Rn.append(bin(G, width=n))
        
        Ln = ['0', '1'] # n == 1
        for n in range(2, MAX_WIDTH):
            Ln = nextLn(Ln)
            del Rn[:]
            B = Signal(intbv(-1))
            G = Signal(intbv(0))
            dut = bin2gray(B, G, n)
            stim = stimulus(B, G, n)
            sim = Simulation(dut, stim)
            sim.run(quiet=1)
            self.assertEqual(Ln, Rn)
\end{verbatim}

As it happens, our implementation is apparently an original Gray code:

\begin{verbatim}
% python test_gray.py -v TestOriginalGrayCode
Check that the code is an original Gray code ... ok

----------------------------------------------------------------------
Ran 1 tests in 3.091s

OK
\end{verbatim}


 


\chapter{MyHDL as a hardware verification language}

\section{Introduction}

One of the most exciting possibilities of \myhdl\
is to use it as a hardware verification language (HVL).
A HVL is a language used to write test benches and
verification environments, and to control simulations.

Nowadays, it is generally acknowledged that HVLs should be equipped
with modern software techniques, such as object orientation. The
reason is that verification it the most complex and time-consuming
task of the design process: consequently every useful technique is
welcome. Moreover, test benches are not required to be
implementable. Therefore, unlike synthesizable code, there
are no constraints on creativity.

Technically, verification of a design implemented in
another language requires cosimulation. \myhdl\ is 
enabled for cosimulation with any HDL simulator that
has a procedural language interface (PLI). The \myhdl\
side is designed to be independent of a particular
simulator, On the other hand, for each HDL simulator a specific
PLI module will have to be written in C. Currently,
the \myhdl\ release contains a PLI module to interface
to the Icarus Verilog simulator. This interface will
be used in the examples.

\section{The HDL side}

To introduce cosimulation, we will continue to use the Gray encoder
example from the previous chapters. Suppose that we want to
synthesize it and write it in Verilog for that purpose. Clearly we would
like to reuse our unit test verification environment. This is exactly
what \myhdl\ offers.

To start, let's recall how the Gray encoder in \myhdl{} looks like:

\begin{verbatim}
def bin2gray(B, G, width):
    """ Gray encoder.

    B -- input intbv signal, binary encoded
    G -- output intbv signal, gray encoded
    width -- bit width
    """
    while 1:
        yield B
        for i in range(width):
            G.next[i] = B[i+1] ^ B[i]

\end{verbatim}

To show the cosimulation flow, we don't need the Verilog
implementation yet, but only the interface.  Our Gray encoder in
Verilog would have the following interface:

\begin{verbatim}
module bin2gray(B, G);

   parameter width = 8;
   input [width-1:0]  B;     
   output [width-1:0] G;
   ....

\end{verbatim}

To write a test bench, one creates a new module that instantiates the
design under test (DUT).  The test bench declares nets and
regs (or signals in VHDL) that are attached to the DUT, and to
stimulus generators and response checkers. In an all-HDL flow, the
generators and checkers are written in the HDL itself, but we will
want to write them in \myhdl{}. To make the connection, we need to
declare which regs \& nets are driven and read by the \myhdl\
simulator. For our example, this is done as follows:

\begin{verbatim}
module dut_bin2gray;

   reg [`width-1:0] B;
   wire [`width-1:0] G;

   initial begin
      $from_myhdl(B);
      $to_myhdl(G);
   end

   bin2gray dut (.B(B), .G(G));
   defparam dut.width = `width;

endmodule

\end{verbatim}

The \code{\$from_myhdl} task call declares which regs are driven by
\myhdl{}, and the \code{\$to_myhdl} task call which regs \& nets are read
by it. These tasks take an arbitrary number of arguments.  They
are defined in a PLI module written in C. They are made available to
the simulation in a simulator-dependent manner.  In Icarus Verilog,
the tasks are defined in a \code{myhdl.vpi} module that is compiled
from C source code.

\section{The \myhdl\ side}

\myhdl\ supports cosimulation by a \code{Cosimulation} object. 
A \code{Cosimulation} object must know how to run a HDL cosimulation.
Therefore, the first argument to its constructor is a command string
to execute a simulation. The way to generate and run an
simulation executable is simulator dependent.
For example, in Icarus Verilog, a simulation executable for our
example can be obtained obtained by running the \code{iverilog}
compiler as follows:

\begin{verbatim}
% iverilog -o bin2gray -Dwidth=4 bin2gray.v dut_bin2gray.v

\end{verbatim}

This generates a \code{bin2gray} executable for a parameter \code{width}
of 4, by compiling the contributing verilog files.

The simulation itself is run by the \code{vvp} command:

\begin{verbatim}
% vvp -m ./myhdl.vpi bin2gray

\end{verbatim}

This runs the \code{bin2gray} simulation, and specifies to use the
\code{myhdl.vpi} PLI module present in the current directory. (This is 
just a command line usage example; actually simulating with the
\code{myhdl.vpi} module is only meaningful from a
\code{Cosimulation} object.)

We can use a \code{Cosimulation} object to provide a HDL cosimulation
version of a design to the \myhdl\ simulator. Instead of a generator
function, we write a function that returns a \code{Cosimulation}
object. For our example and the Icarus Verilog simulator, this is done
as follows:

\begin{verbatim}
import os

from myhdl import Cosimulation

cmd = "iverilog -o bin2gray -Dwidth=%s bin2gray.v dut_bin2gray.v"
      
def bin2gray(B, G, width):
    os.system(cmd % width)
    return Cosimulation("vvp -m ./myhdl.vpi bin2gray", B=B, G=G)

\end{verbatim}

After the executable command argument, the \code{Cosimulation}
constructor takes an arbitrary number of keyword arguments. Those
arguments make the link between \myhdl\ Signals and HDL nets, regs, or
signals, by named association. The keyword is the name of the argument
in a \code{\$to_myhdl} or \code{\$from_myhdl} call; the argument is
the \myhdl\ Signal.

With all this in place, we can now use the existing unit test
to verify the Verilog implementation. Note that we kept the
same name and parameters for the the \code{bin2gray} function:
all we need to do is to provide this alternative definition
to the existing unit test.

Let's quickly try it just to be sure:

\begin{verbatim}
module bin2gray(B, G);

   parameter width = 8;
   input [width-1:0]  B;
   output [width-1:0] G;
   reg [width-1:0] G;
   integer i;

   always @(B) begin
      for (i=0; i < width-1; i=i+1)
        G[i] <= B[i+1] ^ B[i];
   end

endmodule

\end{verbatim}

If we run our unit test we get:

\begin{verbatim}

% python test_bin2gray.py   
Check that only one bit changes in successive codewords ... ERROR
Check that all codewords occur exactly once ... FAIL
Check that the code is an original Gray code ... ERROR
...

\end{verbatim}

Oops! It seems we still have a bug! Oh yes, but of course, 
we need to zero-extend the input to get the msb output bit
correctly:

\begin{verbatim}
module bin2gray(B, G);

   parameter width = 8;
   input [width-1:0]  B;
   output [width-1:0] G;
   reg [width-1:0] G;
   integer i;
   wire [width:0] extB;

   assign extB = {1'b0, B};

   always @(extB) begin
      for (i=0; i < width; i=i+1)
        G[i] <= extB[i+1] ^ extB[i];
   end

endmodule

\end{verbatim}

And now:

\begin{verbatim}
% python test_bin2gray.py 
Check that only one bit changes in successive codewords ... ok
Check that all codewords occur exactly once ... ok
Check that the code is an original Gray code ... ok

----------------------------------------------------------------------
Ran 3 tests in 2.729s

OK

\end{verbatim}


\section{Restrictions}

In the ideal case, it should be possible to simulate
any HDL description seamlessly with \myhdl{}. Moreover
the communicating signals at each side should act
transparently as a single one, enabling fully race free
operation.

For various reasons, it may not be possible or desirable
to achieve full generality. As anyone that has developed
applications with the Verilog PLI can testify, the
restrictions in a particular simulator, and the 
differences over various simulators, can be quite 
frustrating. Moreover, full generality may require
a disproportionate amount of development work compared
to a slightly less general solution that may
be sufficient for the target application.

Consequently, I have tried to achieve a solution
which is simple enough so that one can reasonably 
expect that any PLI-enabled simulator can support it,
and that is relatively easy to verify and maintain.
At the same time, the solution is sufficiently general 
to cover the target application space.

The result is a compromise that places certain restrictions
on the HDL code. In this section, these restrictions 
are presented.

\subsection{Only passive HDL can be cosimulated}

The most important restriction of the \myhdl\ cosimulation solution is
that only ``passive'' HDL can be cosimulated.  This means that the HDL
code should not contain any statements with time delays. In other
words, the \myhdl\ simulator should be the master of time; in
particular, any clock signal should be generated at the \myhdl\ side.

At first this may seem like an important restriction, but if one
considers the target application for cosimulation, it probably
isn't. 

\myhdl\ support cosimulations so that test benches for HDL
designs can be written in Python.
Let's consider the nature of the target HDL designs. For high-level,
behavioral models that are not intended for implementation, it should
come as no surprise that I would recommend to write them in \myhdl\
directly; that is exactly the target of the \myhdl\ effort. Likewise,
gate level designs with annotated timing are not the target
application: static timing analysis is a much better verification
method for such designs.

Rather, the targeted HDL designs are naturally models that are
intended for implementation.  Most likely, this will be through
synthesis. As time delays are meaningless in synthesizable code, the
restriction is compatible with the target application.

\subsection{Race sensitivity issues}

In a typical TTL code, some events cause other events to occur in the
same time step. For example, when a clock signal triggers some signals
may change in the same time step. For race-free operation, an HDL
must differentiate between such events within a time step. This is done
by the concept of ``delta'' cycles. In a fully general, race free
cosimulation, the cosimulators would communicate at the level of delta
cycles. However, in \myhdl\ cosimulation, this is not entirely the
case.

Delta cycles from the \myhdl\ simulator toward the HDL cosimulator are
preserved. However, in the opposite direction, they are not. The
signals changes are only returned to the \myhdl\ simulator after all delta
cycles have been performed in the HDL cosimulator.

What does this mean? Let's start with the good news. As explained in
the previous section, the logic of the \myhdl\ cosimulation implies
that clocks are generated at the \myhdl\ side.  \emph{When using a
\myhdl\ clock and its corresponding HDL signal directly as a clock,
cosimulation operation is race free.} In other words, the case
that most closely reflects the \myhdl\ cosimulation approach, is race free.

The situation is different when you want to use a signal driven by the
HDL (and the corresponding MyHDL signal) as a clock. 
Communication triggered by such a clock is not race free. The solution
is to treat such an interface as a chip interface instead of an RTL
interface.  For example, when data is triggered at positive clock
edges, it can safely be sampled at negative clock edges.
Alternatively, the \myhdl\ data signals can be declared with a delay
value, so that they are guaranteed to change after the clock
edge.


\section{Implementation notes}

\begin{quote}
\em
This section requires some knowledge of PLI terminology.
\end{quote}

Enabling a simulator for cosimulation requires a PLI module
written in C. In Verilog, the PLI is part of the ``standard''.
However, different simulators implement different versions 
and portions of the standard. Worse yet, the behavior of
certain PLI callbacks is not defined on some essential points. 
As a result, one should plan to write a specific PLI module
for any simulator.

The present release contains a PLI module for the 
open source Icarus simulator. I would like to add
modules for any popular simulator in the future,
either from external contributions, or by getting
access to them myself. The same holds for VHDL
simulators: it would be great to have an interface
to the Modelsim VHDL simulator.

This section documents
the current approach and status of the PLI module
implementation in Icarus, and some reflections
on future implementations in other simulators.

\subsection{Icarus Verilog}

To make cosimulation work, a specific type of PLI callback is
needed. The callback should be run when all pending events have been
processed, while allowing the creation of new events in the current
time step (e.g. by the \myhdl\ simulator).  In some Verilog simulators,
the \code{cbReadWriteSync} callback does exactly that. However,
in others, including Icarus, it does not. The callback's behavior is
not fully standardized; some simulators run the callback before
non-blocking assignment events have been processed.

Consequently, I had to look for a workaround. One half of the solution
is to use the \code{cbReadOnlySync} callback.  This callback runs
after all pending events have been processed.  However, it does not
permit to create new events in the current time step.  The second half
of the solution is to map \myhdl\ delta cycles onto Verilog time steps.
Note that there is some freedom here because of the restriction that
only passive HDL code can be cosimulated.

I chose to make the time granularity in the Verilog simulator a 1000
times finer than in the \myhdl{} simulator. For each \myhdl\ time step,
1000 Verilog time steps are available for  \myhdl\ delta cycles. In practice,
only a few delta cycles per time step should be needed. More than 1000
almost certainly indicates an error. This limit is checked at
run-time. The factor 1000 also makes it easy to distinguish ``real''
time from delta cycle time when printing out the Verilog time.

\subsection{Other Verilog simulators}

The Icarus module is written with VPI calls, which are provided by the
most recent generation of the Verilog PLI. Some simulators may only
support TF/ACC calls, requiring a complete redesign of the interface
module.

If the simulator supports VPI, the Icarus module should be reusable to
a large extent. However, it may be possible to improve on it.  The
workaround described in the previous section may not be necessary. In
some simulators, the \code{cbReadWriteSync} callback occurs after all
events (including non-blocking assignments) have been processed. In
that case, the functionality can be supported without a finer time
granularity in the Verilog simulator.

There are also Verilog standardization efforts underway to resolve the
ambiguity of the \code{cbReadWriteSync} callback. The solution will be
to introduce new, well defined callbacks. From reading some proposals,
I conclude that the \code{cbEndOfSimTime} callback would provide the
required functionality.

\subsection{VHDL}

It would be great to have an interface to the Modelsim VHDL
simulator. This will require a redesign from scratch with the
appropriate PLI.  One feature which I would like to keep if possible
is the way to declare the communicating signals.  In the Verilog
solution, it is not necessary to define and instantiate any special
entity (module). Rather, the participating signals can be declared
directly in the \code{to_myhdl} and \code{from_myhdl} task calls.


\chapter{Conversion to Verilog\label{conv}}
\section{Introduction\label{conv-intro}}

MyHDL 0.4 provides a path to automatic implementation, by converting
a subset of MyHDL code into synthesizable Verilog code.

MyHDL aims to be a complete design language, for high level modeling,
verification, but also for implementation. However, prior to \myhdl\
0.4 a \myhdl\ user had to translate synthesizable code manually to
Verilog or VHDL. Needless to say, this is inconvenient. With \myhdl\
0.4, this manual step should no longer be necessary.  The automatic
conversion provides a direct path from Python to an FPGA or ASIC
implementation.

\section{Solution description\label{conv-solution}}

The solution works as follows. The hardware description should be
modeled in \myhdl\ style, and satisfy certain constraints that are
typical for implementation-oriented hardware modeling.  Subsequently,
such a design is converted to an equivalent model in the Verilog
language, using the function \function{toVerilog} from the \myhdl\
library. Finally, a third-party \emph{synthesis tool} is used to
convert the Verilog design into a gate implementation for an ASIC or
FPGA. There are a number of Verilog synthesis tools available, varying
in price, capabilities, and target implementation space.

The conversion does not start from source files, but from a design
that has been "elaborated" by the Python interpreter.  This has
important advantages. First, there are no restrictions on how to
describe structure, as all "structural" constructs and parameters are
processed before the conversion starts. Second, the work of the Python
interpreter is "reused". The converter uses the Python profiler to
track the interpreter's operation and to infer the design structure
and name spaces. It then selectively compiles pieces of source code
for additional analysis and for conversion. This is done using the
Python compiler package.

\section{Features\label{conv-features}}

\subsection{The design is converted after elaboration\label{conv-features-elab}}
\emph{Elaboration} refers to the initial processing of a hardware
description to achieve a representation of a design instance that is
ready for simulation or synthesis. In particular, structural
parameters and constructs are processed in this step. In \myhdl{}, the
Python interpreter itself is used for elaboration.  A
\class{Simulation} object is constructed with elaborated design
instances as arguments.  Likewise, the Verilog conversion works on an
elaborated design instance. The Python interpreter is thus used as
much as possible.

\subsection{The structural description can be arbitrarily complex and hierarchical\label{conv-features-struc}}
As the conversion works on an elaborated design instance, any modeling
constraints only apply to the leaf elements of the design structure,
that is, the co-operating generators. In other words, there are no
restrictions on the description of the design structure: Python's full
power can be used for that purpose. Also, the design hierarchy can be
arbitrarily deep.

\subsection{Generators are mapped to Verilog always or initial blocks\label{conv-features-gen}}
The converter analyzes the code of each generator and maps it
into a Verilog \code{always} blocks if possible, and to 
an \code{initial} block otherwise.
The converted Verilog design will be a flat
"net list of blocks".

\subsection{The Verilog interface is inferred from signal usage\label{conv-features-intf}}
In \myhdl{}, the input or output direction of interface signals
is not explicitly declared. The converter investigates signal usage
in the design hierarchy to infer whether a signal is used as an
input, output, or an internal signal. Internal signals are
given a hierarchical name in the Verilog output.

\subsection{Function calls are mapped to a unique Verilog function or task\label{conv-features-func}}
The converter analyzes function calls and function code to see if they
should be mapped to Verilog functions or to tasks. Python functions
are much more powerful than Verilog subprograms; for example, they are
inherently generic, and they can be called with named association.  To
support this power in Verilog, a unique Verilog function or task is
generated per Python function call.

\subsection{If-then-else structures may be mapped to Verilog case statements\label{conv-features-if}}
Python does not provide a case statement. However, 
the converter recognizes if-then-else structures in which a variable is
sequentially compared to items of an enumeration type, and maps
such a structure to a Verilog case statement with the appropriate
synthesis attributes.

\subsection{Choice of encoding schemes for enumeration types\label{conv-features-enum}}
The \function{enum} function in \myhdl\ returns an enumeration type. This
function takes an additional parameter \var{encoding} that specifies the
desired encoding in the implementation: binary, one hot, or one cold.
The Verilog converter generates the appropriate code.


\section{The convertible subset\label{conv-subset}}

\subsection{Introduction\label{conv-subset-intro}}

Unsurprisingly, not all Python code can be converted into Verilog. In
fact, there are very important restrictions.  As the goal of the
conversion functionality is implementation, this should not be a big
issue: anyone familiar with synthesis is used to similar restrictions
in the \emph{synthesizable subset} of Verilog and VHDL. The converter
attempts to issue clear error messages when it encounters a construct
that cannot be converted. 

In practice, the synthesizable subset usually refers to RTL synthesis,
which is by far the most popular type of synthesis today. There are
industry standards that define the RTL synthesis subset.  However,
those were not used as a model for the restrictions of the MyHDL
converter, but as a minimal starting point.  On that basis, whenever
it was judged easy or useful to support an additional feature, this
was done. For example, it is actually easier to convert while loops
than for loops even though they are not RTL-synthesizable.  As another
example, \keyword{print} is supported because it's so useful for
debugging, even though it's not synthesizable.  In summary, the
convertible subset is a superset of the standard RTL synthesis subset,
and supports synthesis tools with more advanced capabilities, such as
behavioral synthesis.

Recall that any restrictions only apply to the design post
elaboration.  In practice, this means that they apply only to the code
of the generators, that are the "leaf" functional blocks in a MyHDL
design.

\subsection{Coding style\label{conv-subset-style}}

A natural restriction on convertible code is that it should be
written in MyHDL style: cooperating generators, communicating through
signals, and with \code{yield} statements specifying wait points and resume
conditions.  Supported resume conditions are a signal edge, a signal
change, or a tuple of such conditions.

\subsection{Supported types\label{conv-subset-types}}

The most important restriction regards object types. Verilog is an
almost typeless language, while Python is strongly (albeit
dynamically) typed. The converter needs to infer the types of
variables and map them to Verilog types. Therefore, it does type
inferencing of object constructors and expressions.

Only a limited amount of types can be converted.
Python \class{int} and \class{long} objects are mapped to Verilog
integers. All other supported types are mapped to Verilog regs (or
wires), and therefore need to have a defined bit width. The supported
types are the Python \class{bool} type, the MyHDL \class{intbv} type,
and MyHDL enumeration types returned by function \function{enum}. The
latter objects can also be used as the base object of a
\class{Signal}. 

\class{intbv} objects need to be constructed so that a bit
width can be inferred. This can be done by specifying minimum
and maximum values, e.g. as follows:

\begin{verbatim}
index = intbv(0, min=0, max=2**N)
\end{verbatim}

Alternatively, a slice can be taken from an \class{intbv} object
as follows:

\begin{verbatim}
index = intbv(0)[N:]
\end{verbatim}

Such as slice returns a new \class{intbv} object, with minimum
value \code{0} , and maximum value \code{2**N}.


\subsection{Supported statements\label{conv-subset-statements}}

The following is a list of the statements that are supported by the
Verilog converter, possibly qualified with restrictions
or usage notes. Recall that
this list only applies to the design post elaboration: in practice,
this means it applies to the code of the generators that are the leaf
blocks in a design.

\begin{description}

\item[The \keyword{break} statement.]

\item[The \keyword{continue} statement.]

\item[The \keyword{def} statement.]

\item[The \keyword{for} statement.]
The only supported iteration scheme is iterating through sequences of
integers returned by built-in function \function{range} or \myhdl\
function \function{downrange}.  The optional \keyword{else} clause is
not supported.

\item[The \keyword{if} statement.]
\keyword{if}, \keyword{elif}, and \keyword{else} clauses
are fully supported.

\item[The \keyword{pass} statement.]

\item[The \keyword{print} statement.]
The only supported expression for printing is
a single literal string.
The string can be interpolated, but the format specifiers
are copied verbatim to the Verilog output.
Print to a file (with syntax \code{'>>'}) is not supported.

\item[The \keyword{raise} statement.]
This statement is mapped to Verilog statements
that end the simulation with an error message.

\item[The \keyword{return} statement.]

\item[The \keyword{yield} statement.] 
The yielded expression can be a signal, a signal edge
as specified by \myhdl\ functions \function{posedge}
or \function{negedge}, or a tuple of signals and
edge specifications.

\item[The \keyword{while} statement.]
The optional \keyword{else}
clause is not supported.

\end{description}

\section{Methodology notes\label{conv-meth}}

\subsection{Simulation\label{conv-meth-sim}}

In the Python philosophy, the run-time rules. The Python compiler
doesn't attempt to detect a lot of errors beyond syntax errors, which
given Python's ultra-dynamic nature would be an almost impossible task
anyway. To verify a Python program, one should run it, preferably
using unit testing to verify each feature.

The same philosophy should be used when converting a MyHDL description
to Verilog: make sure the simulation runs fine first. Although the
converter checks many things and attempts to issue clear error
messages, there is no guarantee that it does a meaningful job unless
the simulation runs fine.

\subsection{Conversion output verification\label{conv-meth-conv}}
It is always prudent to verify the converted Verilog output.
To make this task easier, the converter also generates a
test bench that makes it possible to simulate the Verilog
design using the Verilog co-simulation interface. This 
permits to verify the Verilog code with the same test
bench used for the \myhdl\ code. This is also how
the Verilog converter development is being verified.

\subsection{Assignment issues\label{conv-meth-assign}}

\subsubsection{Name assignment in Python\label{conv-meth-assign-python}}

Name assignment in Python is a different concept than in
many other languages. This point is very important for
effective modeling in Python, and even more so
for synthesizable \myhdl\ code. Therefore, the issues are
discussed here.

Consider the following name assignments:

\begin{verbatim}
a = 4
a = ``a string''
a = False
\end{verbatim}

In many languages, the meaning would be that an
existing variable \var{a} gets a number of different values.
In Python, such a concept of a variable doesn't exist. Instead,
assignment merely creates a new binding of a name to a
certain object, that replaces any previous binding.
So in the example, the name \var{a} is bound a 
number of different objects in sequence.

The Verilog converter has to investigate name
assignment and usage in \myhdl\ code, and to map
names to Verilog variables. To achieve that,
it tries to infer the type and possibly the
bit width of each expression that is assigned
to a name.

Multiple assignments to the same name can be supported if it can be
determined that a consistent type and bit width is being used in the
assignments. This can be done for boolean expressions, numeric
expressions, and enumeration type literals. In Verilog, the
corresponding name is mapped to a single bit \code{reg}, an
\code{integer} or a \code{reg} of the appropriate width, respectively.

In other cases, a single assignment should be used when an object is
created. Subsequent value changes are then achieved by modification of
an existing object.  This technique should be used for \class{Signal}
and \class{intbv} objects.

\subsubsection{Signal assignment\label{conv-meth-assign-signal}}

Signal assignment in \myhdl\ is implemented using attribute assignment
to attribute \code{next}.  Value changes are thus modeled by
modification of the existing object. The converter investigates the
\class{Signal} object to infer the type and bit width of the
corresponding Verilog variable.

\subsubsection{\class{intbv} objects\label{conv-meth-assign-intbv}}

Type \class{intbv} is likely to be the workhorse for synthesizable
modeling in \myhdl{}. An \class{intbv} instance behaves like a
(mutable) integer whose individual bits can be accessed and
modified. Also, it is possible to constrain its set of values. In
addition to error checking, this makes it possible to infer a bit
width, which is required for implementation.

In Verilog, an \class{intbv} instance will be mapped to a \code{reg}
with an appropriate width. As noted before, it is not possible
to modify its value using name assignment. In the following, we
will show how it can be done instead. Consider:

\begin{verbatim}
a = intbv(0)[8:]
\end{verbatim}

This is an \class{intbv} object with initial value \code{0} and
bit width 8. The change its value to \code{5}, we can use
slice assignment:

\begin{verbatim}
a[8:] = 5
\end{verbatim}

The same can be achieved by leaving the bit width unspecified, 
which has the meaning to change ``all'' bits:

\begin{verbatim}
a[:] = 5
\end{verbatim}

Often the new value will depend on the old one. For example,
to increment an \class{intbv} with the technique above:

\begin{verbatim}
a[:] = a + 1
\end{verbatim}

Python also provides \emph{augmented} assignment operators,
which can be used to implement in-place operations. These are supported
on \class{intbv} objects and by the converter, so that the increment
can also be done as follows:

\begin{verbatim}
a += 1
\end{verbatim}

\section{Converter usage\label{conv-usage}}

We will demonstrate the conversion process by showing some examples.

\subsection{A small design with a single generator\label{conv-usage-small}}

Consider the following MyHDL code for an incrementer module:

\begin{verbatim}
def inc(count, enable, clock, reset, n):
    """ Incrementer with enable.
    
    count -- output
    enable -- control input, increment when 1
    clock -- clock input
    reset -- asynchronous reset input
    n -- counter max value
    """
    def incProcess():
        while 1:
            yield posedge(clock), negedge(reset)
            if reset == ACTIVE_LOW:
                count.next = 0
            else:
                if enable:
                    count.next = (count + 1) % n
    return incProcess()
\end{verbatim}

In Verilog terminology, function \function{inc} corresponds to a
module, while generator function \function{incProcess}
roughly corresponds to an always block.

Normally, to simulate the design, we would "elaborate" an instance
as follows:

\begin{verbatim}
m = 8
n = 2 ** m
 
count = Signal(intbv(0)[m:])
enable = Signal(bool(0))
clock, reset = [Signal(bool()) for i in range(2)]

inc_inst = inc(count, enable, clock, reset, n=n)
\end{verbatim}

\code{inc_inst} is an elaborated design instance that can be simulated. To
convert it to Verilog, we change the last line as follows:

\begin{verbatim}
inc_inst = toVerilog(inc, count, enable, clock, reset, n=n)
\end{verbatim}

Again, this creates an instance that can be simulated, but as a side
effect, it also generates a Verilog module in file \file{inc_inst.v},
that is supposed to have identical behavior. The Verilog code
is as follows:

\begin{verbatim}
module inc_inst (
    count,
    enable,
    clock,
    reset
);

output [7:0] count;
reg [7:0] count;
input enable;
input clock;
input reset;


always @(posedge clock or negedge reset) begin: _MYHDL1_BLOCK
    if ((reset == 0)) begin
        count <= 0;
    end
    else begin
        if (enable) begin
            count <= ((count + 1) % 256);
        end
    end
end

endmodule
\end{verbatim}

You can see the module interface and the always block, as expected
from the MyHDL design. 

\subsection{Converting a generator directly\label{conv-usage-gen}}

It is also possible to convert a generator
directly. For example, consider the following generator function:

\begin{verbatim}
def bin2gray(B, G, width):
    """ Gray encoder.

    B -- input intbv signal, binary encoded
    G -- output intbv signal, gray encoded
    width -- bit width
    """
    Bext = intbv(0)[width+1:]
    while 1:
        yield B
        Bext[:] = B
        for i in range(width):
            G.next[i] = Bext[i+1] ^ Bext[i]
\end{verbatim}

As before, you can create an instance and convert to
Verilog as follows:

\begin{verbatim}
width = 8

B = Signal(intbv(0)[width:])
G = Signal(intbv(0)[width:])

bin2gray_inst = toVerilog(bin2gray, B, G, width)
 \end{verbatim}

The generate Verilog module is as follows:

\begin{verbatim}
module bin2gray_inst (
    B,
    G
);

input [7:0] B;
output [7:0] G;
reg [7:0] G;

always @(B) begin: _MYHDL1_BLOCK
    integer i;
    reg [9-1:0] Bext;
    Bext[9-1:0] = B;
    for (i=0; i<8; i=i+1) begin
        G[i] <= (Bext[(i + 1)] ^ Bext[i]);
    end
end

endmodule
\end{verbatim}

\subsection{A hierarchical design\label{conv-usage-hier}}
The hierarchy of convertible designs can be
arbitrarily deep.

For example, suppose we want to design an
incrementer with Gray code output. Using the
designs from previous sections, we can proceed
as follows:

\begin{verbatim}
def GrayInc(graycnt, enable, clock, reset, width):
    
    bincnt = Signal(intbv()[width:])
    
    INC_1 = inc(bincnt, enable, clock, reset, n=2**width)
    BIN2GRAY_1 = bin2gray(B=bincnt, G=graycnt, width=width)
    
    return INC_1, BIN2GRAY_1
\end{verbatim}

According to Gray code properties, only a single bit
will change in consecutive values. However, as the
\code{bin2gray} module is combinatorial, the output bits
may have transient glitches, which may not be desirable.
To solve this, let's create an additional level of
hierarchy an add an output register to the design.
(This will create an additional latency of a clock
cycle, which may not be acceptable, but we will
ignore that here.)

\begin{verbatim}
def GrayIncReg(graycnt, enable, clock, reset, width):
    
    graycnt_comb = Signal(intbv()[width:])
    
    GRAY_INC_1 = GrayInc(graycnt_comb, enable, clock, reset, width)
    
    def reg():
        while 1:
            yield posedge(clock)
            graycnt.next = graycnt_comb
    REG_1 = reg()
    
    return GRAY_INC_1, REG_1
\end{verbatim}

We can convert this hierarchical design as before:

\begin{verbatim}
width = 8
graycnt = Signal(intbv()[width:])
enable, clock, reset = [Signal(bool()) for i in range(3)]

GRAY_INC_REG_1 = toVerilog(GrayIncReg, graycnt, enable, clock, reset, width)
\end{verbatim}

The Verilog output module looks as follows:

\begin{verbatim}
module GRAY_INC_REG_1 (
    graycnt,
    enable,
    clock,
    reset
);

output [7:0] graycnt;
reg [7:0] graycnt;
input enable;
input clock;
input reset;

reg [7:0] graycnt_comb;
reg [7:0] _GRAY_INC_1_bincnt;

always @(posedge clock or negedge reset) begin: _MYHDL1_BLOCK
    if ((reset == 0)) begin
        _GRAY_INC_1_bincnt <= 0;
    end
    else begin
        if (enable) begin
            _GRAY_INC_1_bincnt <= ((_GRAY_INC_1_bincnt + 1) % 256);
        end
    end
end

always @(_GRAY_INC_1_bincnt) begin: _MYHDL4_BLOCK
    integer i;
    reg [9-1:0] Bext;
    Bext[9-1:0] = _GRAY_INC_1_bincnt;
    for (i=0; i<8; i=i+1) begin
        graycnt_comb[i] <= (Bext[(i + 1)] ^ Bext[i]);
    end
end

always @(posedge clock) begin: _MYHDL9_BLOCK
    graycnt <= graycnt_comb;
end

endmodule
\end{verbatim}

Note that the output is a flat ``net list of blocks'', and
that hierarchical signal names are generated as necessary.

\subsection{Optimizations for finite state machines\label{conv-usage-fsm}}
As often in hardware design, finite state machines deserve special attention.

In Verilog and VHDL, finite state machines are typically described
using case statements.  Python doesn't have a case statement, but the
converter recognizes particular if-then-else structures and maps them
to case statements. This optimization occurs when a variable whose
type is an enumerated type is sequentially tested against enumeration
items in an if-then-else structure. Also, the appropriate synthesis
pragmas for efficient synthesis are generated in the Verilog code.

As a further optimization, function \function{enum} was enhanced to support
alternative encoding schemes elegantly, using an additional parameter
'encoding'. For example:

\begin{verbatim}
t_State = enum('SEARCH', 'CONFIRM', 'SYNC', encoding="one_hot")
\end{verbatim}

The default encoding is \code{binary}; the other possibilities are \code{one_hot} and
\code{one_cold}. This parameter only affects the conversion output, not the
behavior of the type. Verilog case statements are optimized for an
efficient implementation according to the encoding. Note that in
contrast, a Verilog designer needs to make nontrivial code changes to
implement a different encoding scheme.

As an example, consider the following finite state machine, whose
state variable used the enumeration type defined above:

\begin{verbatim}
FRAME_SIZE = 8

def FramerCtrl(SOF, state, syncFlag, clk, reset_n):
    
    """ Framing control FSM.

    SOF -- start-of-frame output bit
    state -- FramerState output
    syncFlag -- sync pattern found indication input
    clk -- clock input
    reset_n -- active low reset
    
    """
    
    index = intbv(0, min=0, max=8) # position in frame
    while 1:
        yield posedge(clk), negedge(reset_n)
        if reset_n == ACTIVE_LOW:
            SOF.next = 0
            index[:] = 0
            state.next = t_State.SEARCH
        else:
            SOF.next = 0
            if state == t_State.SEARCH:
                index[:] = 0
                if syncFlag:
                    state.next = t_State.CONFIRM
            elif state == t_State.CONFIRM:
                if index == 0:
                    if syncFlag:
                        state.next = t_State.SYNC
                    else:
                        state.next = t_State.SEARCH
            elif state == t_State.SYNC:
                if index == 0:
                    if not syncFlag:
                        state.next = t_State.SEARCH
                SOF.next = (index == FRAME_SIZE-1)
            else:
                raise ValueError("Undefined state")
            index[:]= (index + 1) % FRAME_SIZE

\end{verbatim}

The conversion is done as before:

\begin{verbatim}
SOF = Signal(bool(0))
syncFlag = Signal(bool(0))
clk = Signal(bool(0))
reset_n = Signal(bool(1))
state = Signal(t_State.SEARCH)
framerctrl_inst = toVerilog(FramerCtrl, SOF, state, syncFlag, clk, reset_n)
\end{verbatim}

The Verilog output looks as follows:

\begin{verbatim}
module framerctrl_inst (
    SOF,
    state,
    syncFlag,
    clk,
    reset_n
);
output SOF;
reg SOF;
output [2:0] state;
reg [2:0] state;
input syncFlag;
input clk;
input reset_n;

always @(posedge clk or negedge reset_n) begin: _MYHDL1_BLOCK
    reg [3-1:0] index;
    if ((reset_n == 0)) begin
        SOF <= 0;
        index[3-1:0] = 0;
        state <= 3'b001;
    end
    else begin
        SOF <= 0;
        // synthesis parallel_case full_case
        casez (state)
            3'b??1: begin
                index[3-1:0] = 0;
                if (syncFlag) begin
                    state <= 3'b010;
                end
            end
            3'b?1?: begin
                if ((index == 0)) begin
                    if (syncFlag) begin
                        state <= 3'b100;
                    end
                    else begin
                        state <= 3'b001;
                    end
                end
            end
            3'b1??: begin
                if ((index == 0)) begin
                    if ((!syncFlag)) begin
                        state <= 3'b001;
                    end
                end
                SOF <= (index == (8 - 1));
            end
            default: begin
                $display("Verilog: ValueError(Undefined state)");
                $finish;
            end
        endcase
        index[3-1:0] = ((index + 1) % 8);
    end
end
endmodule
\end{verbatim}


\section{Known issues\label{conv-issues}}
\begin{description}

\item[Negative values of \class{intbv} instances are not supported.]
The \class{intbv} class is quite capable of representing negative
values. However, the \code{signed} type support in Verilog is
relatively recent and mapping to it may be tricky. In my judgment,
this is not the most urgent requirement, so
I decided to leave this for later.

\item[Verilog integers are 32 bit wide]
Usually, Verilog integers are 32 bit wide. In contrast, Python is
moving toward integers with undefined width. Python \class{int} 
and \class{long} variables are mapped to Verilog integers; so for values
larger than 32 bit this mapping is incorrect.

\item[Synthesis pragmas are specified as Verilog comments.] The recommended
way to specify synthesis pragmas in Verilog is through attribute
lists. However, my Verilog simulator (Icarus) doesn't support them
for \code{case} statements (to specify \code{parallel_case} and
\code{full_case} pragmas). Therefore, I still used the old
but deprecated method of synthesis pragmas in Verilog comments.

\item[Inconsistent place of the sensitivity list inferred from \code{always_comb}.]
The semantics of \code{always_comb}, both in Verilog and \myhdl{}, is to
have an implicit sensitivity list at the end of the code. However, this
may not be synthesizable. Therefore, the inferred sensitivity list is
put at the top of the corresponding \code{always} block.
This may cause inconsistent behavior at the start of the
simulation. The workaround is to create events at time 0.

\item[Non-blocking assignments to task arguments don't work.] 
I didn't get non-blocking (signal) assignments to task arguments to
work.  I don't know yet whether the issue is my own, a Verilog issue,
or an issue with my Verilog simulator Icarus. I'll need to check this
further.

\end{description}


\chapter{Reference manual}


\myhdl\ is implemented as a Python package called \code{myhdl}. This
chapter describes the objects that are exported by this package.

\section{The \class{Simulation} class}
\begin{classdesc}{Simulation}{arg \optional{, arg \moreargs}}
Class to construct a new simulation. Each argument is either be a
\myhdl\ generator, or a nested sequence of such generators. (A nested
sequence is defined as a sequence in which each item may itself be a
nested sequence.) See section~\ref{myhdl-generators} for the
definition of \myhdl\ generators and their interaction with a
\class{Simulation} object.
\end{classdesc}

A \class{Simulation} object has the following method:

\begin{methoddesc}[Simulation]{run}{\optional{duration}}
Run the simulation forever (by default) or for a specified duration.
\end{methoddesc}

\section{The \class{Signal} class}
\label{signal}
\begin{classdesc}{Signal}{val \optional{, delay}}
This class is used to construct a new signal and to initialize its
value to \var{val}. Optionally, a delay can be specified.
\end{classdesc}

A \class{Signal} object has the following attributes:

\begin{memberdesc}[Signal]{next}
Read-write attribute that represents the next value of the signal.
\end{memberdesc}

\begin{memberdesc}[Signal]{val}
Read-only attribute that represents the current value of the signal.

This attribute is always available to access the current value;
however in many practical case it will not be needed. Whenever there
is no ambiguity, the Signal object's current value is used
implicitly. In particular, all Python's standard numeric, bit-wise,
logical and comparison operators are implemented on a Signal object by
delegating to its current value. The exception is augmented
assignment. These operators are not implemented as they would break
the rule that the current value should be a read-only attribute. In
addition, when a Signal object is directly assigned to the \code{next}
attribute of another Signal object, its current value is assigned
instead.
\end{memberdesc}



\section{\myhdl\ generators and trigger objects}
\label{myhdl-generators}
\myhdl\ generators are standard Python generators with specialized
\keyword{yield} statements. In hardware description languages, the equivalent
statements are called \emph{sensitivity lists}. The general format
of \keyword{yield} statements in in \myhdl\ generators is:

\hspace{\leftmargin}\keyword{yield} \var{clause \optional{, clause ...}}

After a simulation object executes a \keyword{yield} statement, it
suspends execution of the generator. At the same time, each
\var{clause} is a \emph{trigger object} which defines the condition
upon which the generator should be resumed. However, per invocation of a
\keyword{yield} statement, the generator is resumed exactly once,
regardless of the number of clauses. This happens as soon as one
of the objects triggers; subsequent triggers are
neglected. (However, as a result of the resumption, it is possible
that the same \keyword{yield} statement is invoked again, and that a
subsequent trigger still triggers the generator.)

In this section, the trigger objects and their functionality will be
described. 

\begin{funcdesc}{posedge}{signal}
Return a trigger object that specifies that the generator should
resume on a rising edge on the signal. A rising edge means a change
from false to true.
\end{funcdesc}

\begin{funcdesc}{negedge}{signal}
Return a trigger object that specifies that the generator should
resume on a falling edge on the signal. A falling edge means a change
from true to false.
\end{funcdesc}

\begin{funcdesc}{delay}{t}
Return a trigger object that specifies that the generator should
resume after a delay \var{t}.
\end{funcdesc}

\begin{funcdesc}{join}{arg \optional{, arg \moreargs}}
Join a number of trigger objects together and return a joined
trigger object.  The effect is that the joined trigger object will
trigger when \emph{all} of its arguments have triggered.
\end{funcdesc}

In addition, some objects can directly function as trigger
objects. These are the objects of the following types:

\begin{datadesc}{Signal}
For the full description of the \class{Signal} class, see
section~\ref{signal}.

A signal is a trigger object. Whenever a signal changes value, the
generator is triggered.
\end{datadesc}

\begin{datadesc}{GeneratorType}
\myhdl\ generators can itself be used as trigger objects. 
This corresponds to spawning a new generator, while the original
generator waits for it to complete.  In other words, the original
generator is triggered when the spawned generator completes.
\end{datadesc}

In addition, as a special case, the Python \code{None} object can be
present in  a \code{yield} statement:

\begin{datadesc}{None}
This is the do-nothing trigger object. The generator immediately
resumes, as if no \code{yield} statement were present. This can be
useful if the \code{yield} statement also has generator clauses: those
generators are spawned, while the original generator resumes
immediately.

\end{datadesc}



\section{Miscellaneous objects}

The following objects can be convenient in \myhdl\ modeling.

\begin{excclassdesc}{StopSimulation}{}
Base exception that is caught by the \code{Simulation.run} method to
stop a simulation. Can be subclassed and raised in generator code.
\end{excclassdesc}

\begin{funcdesc}{now}{}
Return the current simulation time.
\end{funcdesc}

\begin{funcdesc}{downrange}{high \optional{, low=0}}
Generates a downward range list of integers. Modeled after the
standard \code{range} function, but works in the downward
direction. The returned interval is half-open, with the \var{high}
index not included. \var{low} is optional and defaults to zero.

This function is especially useful with the \class{intbv} class, that
also works with downward indexing.
\end{funcdesc}

\begin{funcdesc}{bin}{num \optional{, width}}
Return a representation as a bit string.  If \var{width} is provided,
and if it is larger than the width of the default representation, the
bit string is padded with the sign bit.

This function complements the standard Python conversion functions
\code{hex} and \code{oct}. A binary string representation is often
needed in hardware design.
\end{funcdesc}


\section{The \class{intbv} class}

\begin{classdesc}{intbv}{arg}
This class represents \class{int}-like objects with some additional
features that make it suitable for hardware design. The constructor
argument can be an \class{int}, a \class{long}, an \class{intbv} or a
bit string (a string with only '0's or '1's). For a bit string
argument, the value is calculated as \code{int(\var{bitstring}, 2)}. 
\end{classdesc}

Unlike \class{int} objects, \class{intbv} objects are mutable; this is
also the reason for their existence. Mutability is needed to support
assignment to indexes and slices, as is common in hardware design. For
the same reason, \class{intbv} is not a subclass from \class{int},
even though \class{int} provides most of the desired
functionality. (It is not possible to derive a mutable subtype from
an immutable base type.)

An \class{intbv} object supports the same comparison, numeric,
bitwise, logical, and conversion operations as \class{int} objects. See
\url{http://www.python.org/doc/current/lib/typesnumeric.html} for more
information on such operations. In all binary operations,
\class{intbv} objects can work together with \class{int} objects; in
those cases the return type is an \class{intbv} object.

In addition, \class{intbv} objects support indexing and slicing
operations:

\begin{tableiii}{clc}{code}{Operation}{Result}{Notes}
  \lineiii{\var{bv}[\var{i}]}
	  {item \var{i} of \var{bv}}
	  {(1)}
  \lineiii{\var{bv}[\var{i}] = \var{x}}  
	  {item \var{i} of \var{bv} is replaced by \var{x}} 
          {(1)}
  \lineiii{\var{bv}[\var{i}:\var{j}]} 
          {slice of \var{bv} from \var{i} downto \var{j}} 
          {(2)(3)}
  \lineiii{\var{bv}[\var{i}:\var{j}] = \var{t}} 
  	  {slice of \var{bv} from \var{i} downto \var{j} is replaced
          by \var{t}} 
          {(2)(4)}
\end{tableiii}

\begin{description}
\item[(1)] Indexing follows the most common hardware design
	  conventions: the lsb bit is the rightmost bit, and it has
	  index 0. This has the following desirable property: if the
	  \class{intbv} value is decomposed as a sum of powers of 2,
	  the bit with index \var{i} corresponds to the term
	  \code{2**i}.

\item[(2)] It follows from the indexing convention that slicing ranges
	  are downward, in contrast to standard Python. However, the
	  Python convention of half-open ranges is followed. In
	  accordance with standard Python, the high index is not
	  included. However, it is the \emph{leftmost} index in this
	  case. As in standard Python, this takes care of one-off
	  issues in many practical cases: in particular,
	  \code{bv[\var{i}:]} returns \var{i} bits;
	  \code{bv[\var{i}:\var{j}]} has \code{\var{i}-\var{j}}
	  bits. As \class{intbv} objects have no explicitly defined
	  bit width, the high index \var{j} has no default value and
	  cannot be omitted, while the low index \var{j} defaults to
	  \code{0}.

\item[(3)] The value returned from a slicing operation is always
	  positive; higher order bits are implicitly assumed to be
	  zero. The bit width is implicitly stored in the returned bit
	  width, so that the returned object can be used in
	  concatenations and as an iterator.

\item[(4)] In setting a slice, it is checked whether the slice is wide
	  enough to accept all significant bits of the value.
\end{description}

In addition, \class{intbv} objects support a concatenation method:

\begin{methoddesc}[intbv]{concat}{\optional{arg \moreargs}}
Concatenate the arguments to an \class{intbv} object. Naturally, the
concatenation arguments need to have a defined bit width. Therefore,
if they are \class{intbv} objects, they have to be the return values
of a slicing operation. Alternatively, they may be bit strings.

In contrast to all other arguments, the implicit \var{self} argument
doesn't need to have a defined bit with. This is due to the fact that
concatenation occurs at the lsb (rightmost) side.

It may be clearer to call this method as an unbound method with an
explicit first \class{intbv} argument.
\end{methoddesc}

In addition, an \class{intbv} object supports the iterator protocol. This
makes it possible to iterate over all its bits, from the high index to
index 0. This is only possible for \class{intbv} objects with a
defined bit width.













\documentclass{manual}
\usepackage{palatino}
\renewcommand{\ttdefault}{cmtt}
\renewcommand{\sfdefault}{cmss}
\newcommand{\myhdl}{\protect \mbox{MyHDL}}
\usepackage{graphicx}

\title{The \myhdl\ manual}

\author{Jan Decaluwe}
\authoraddress{
Email: \email{jan@jandecaluwe.com}
}

\date{December 19, 2005}	% XXX update before release!
\release{0.5b1}  		% software release, not documentation
\setreleaseinfo{}		% empty for final release
\setshortversion{0.5b1}		% major.minor only for software


\makeindex

\begin{document}

\maketitle

Copyright \copyright{} 2001 Python Software Foundation.
All rights reserved.

Copyright \copyright{} 2000 BeOpen.com.
All rights reserved.

Copyright \copyright{} 1995-2000 Corporation for National Research Initiatives.
All rights reserved.

Copyright \copyright{} 1991-1995 Stichting Mathematisch Centrum.
All rights reserved.

See the end of this document for complete license and permissions
information.


\begin{abstract}

\noindent

The goal of the \myhdl{} project is to empower hardware designers with
the elegance and simplicity of the Python language.

\myhdl{} is a free, open-source (LGPL) package for using Python as a
hardware description and verification language. Python is a very high
level language, and hardware designers can use its full power to model
and simulate their designs. Moreover, \myhdl{} can convert a design to
Verilog. In combination with an external synthesis tool, it provides a
complete path from Python to a silicon implementation.

\emph{Modeling}


Python's power and clarity make \myhdl{} an ideal solution for high level
modeling. Python is famous for enabling elegant solutions to complex
modeling problems. Moreover, Python is outstanding for rapid
application development and experimentation.

The key idea behind \myhdl{} is the use of Python generators to model
hardware concurrency. Generators are best described as resumable
functions. In \myhdl{}, generators are used in a specific way so that
they become similar to always blocks in Verilog or processes in VHDL.

A hardware module is modeled as a function that returns any number of
generators. This approach makes it straightforward to support features
such as arbitrary hierarchy, named port association, arrays of
instances, and conditional instantiation.

Furthermore, \myhdl{} provides classes that implement traditional
hardware description concepts. It provides a signal class to support
communication between generators, a class to support bit oriented
operations, and a class for enumeration types.

\emph{Simulation and Verification}

The built-in simulator runs on top of the Python interpreter. It
supports waveform viewing by tracing signal changes in a VCD file.

With \myhdl{}, the Python unit test framework can be used on hardware
designs. Although unit testing is a popular modern software
verification technique, it is not yet common in the hardware design
world, making it one more area in which \myhdl{} innovates.

\myhdl{} can also be used as hardware verification language for VHDL and
Verilog designs, by co-simulation with traditional HDL simulators.

\emph{Conversion to Verilog}

The converter to Verilog works on an instantiated design that has been
fully elaborated. Consequently, the original design structure can be
arbitrarily complex.

The converter automates certain tasks that are tedious or hard in
Verilog directly. Notable features are the possibility to choose
between various FSM state encodings based on a single attribute, the
mapping of certain high-level objects to RAM and ROM descriptions, and
the automated handling of signed arithmetic issues.



\end{abstract}

\tableofcontents

\chapter{Background information}

\section{Prerequisites}

You need a basic understanding of Python to use \myhdl\.
If you don't know Python, you will take comfort in knowing
that it is probably one of the easiest programming languages to
learn \footnote{You must be bored by such claims, but in Python's
case it's true.}. Learning Python is also one of the better time
investments that engineering professionals can make \footnote{I am not
biased.}.

For beginners, \url{http://www.python.org/doc/current/tut/tut.html} is
probably the best choice for an on-line tutorial. For alternatives,
see \url{http://www.python.org/doc/Newbies.html}.

A working knowledge of a hardware description language such as Verilog
or VHDL is helpful. Chances are that you know one of those anyway, if
you are interested in \myhdl{}.

\section{A small tutorial on generators}

Generators are a recent feature in Python. They were introduced in
Python 2.2, which is the most recent stable version at the time of
this writing. Therefore, there isn't a lot of tutorial material
available yet. Because generators are the key concept in
\myhdl{}, I include a small tutorial here.

Consider the following nonsensical function:

\begin{verbatim}
def function():
    for i in range(5):
        return i

\end{verbatim}

You can see why it doesn't make a lot of sense. As soon as the first
loop iteration is entered, the function returns:

\begin{verbatim}
>>> function()
0
\end{verbatim}

Returning is fatal for the function call. Further loop iterations
never get a chance, and nothing is left over from the function call
when it returns.

To change the function into a generator function, we replace
\keyword{return} with \keyword{yield}:

\begin{verbatim}
def generator():
    for i in range(5):
        yield i

\end{verbatim}

Now we get:

\begin{verbatim}
>>> generator()
<generator object at 0x815d5a8>

\end{verbatim}

When a generator function is called, it returns a generator object. A
generator object supports the iterator protocol, which is an expensive
way of saying that you can let it generate subsequent values by
calling its \function{next()} method:

\begin{verbatim}
>>> g = generator()
>>> g.next()
0
>>> g.next()
1
>>> g.next()
2
>>> g.next()
3
>>> g.next()
4
>>> g.next()
Traceback (most recent call last):
  File "<stdin>", line 1, in ?
StopIteration

\end{verbatim}

Now we can generate the subsequent values from the for
loop on demand, until they are exhausted. What happens is that the
\keyword{yield} statement is like a
\keyword{return}, except that it is non-fatal: the generator remembers
its state and the point in the code when it yielded. A higher order
agent can decide when to get a further value by calling the
generator's \function{next()} method. We say that generators are
\dfn{resumable functions}.

If you are familiar with hardware description languages, this may
ring a bell. In hardware simulations, there is also a higher order
agent, the Simulator tool, that interacts with such resumable
functions; they are called processes in VHDL and always blocks in
Verilog. Like in those languages, Python generators provide an elegant
and efficient method to model concurrency, without having to resort to
some form of threading.

The use of generators to model concurrency is the first key concept in
\myhdl{}. The second key concept is a related one: in \myhdl{}, the
yielded values are used to define the condition upon which the generator
should be resumed. In other words, the \keyword{yield} statements work
as generalized sensitivity lists. If by now you are still interested,
read on to learn more!

If you want to know more about generators, consult the on-line Python
documentation, e.g. at \url{http://www.python.org/doc/2.2.2/whatsnew}. 

\begin{notice}[warning]
At the beginning of this section I said that generators were
introduced in Python 2.2. This is not entirely correct: in fact,
generators will only be enabled as a standard feature in Python 2.3.
However, a stable version of Python 2.3 has not been released yet at
the time of this writing. So, what to do?

Fortunately, Python lets you import features from its future releases
(provided that the future is not too distant). So, until you use
Python 2.3 or higher, you have to include the following line at the
start of all code that defines generator functions:

\begin{verbatim}
from __future__ import generators

\end{verbatim}

From Python 2.3 on, this line may still be in the code, though it
will not have an effect anymore.
\end{notice}


\chapter{Introduction to \myhdl\ \label{intro}}

\section{A basic \myhdl\ simulation \label{intro-basic}}

We will introduce \myhdl\ with a classic \code{Hello World} style
example. All example code can be found in the distribution directory
under \file{example/manual}.  Here are the contents of a \myhdl\
simulation script called \file{hello1.py}:

\begin{verbatim}
from myhdl import Signal, delay, always, now, Simulation

def HelloWorld():
    
    @always(delay(10))
    def sayHello():
        print "%s Hello World!" % now()

    return sayHello

inst = HelloWorld()
sim = Simulation(inst)
sim.run(30)
\end{verbatim}

When we run this simulation, we get the following output:

\begin{verbatim}
% python hello1.py
10 Hello World!
20 Hello World!
30 Hello World!
_SuspendSimulation: Simulated 30 timesteps
\end{verbatim}

The first line of the script imports a number of objects from the
\code{myhdl} package. In Python we can only use identifiers that are
literally defined in the source file 
\footnote{The exception is the \samp{from module import *} syntax,
that imports all the symbols from a module. Although this is generally
considered bad practice, it can be tolerated for large modules that
export a lot of symbols. One may argue that
\code{myhdl} falls into that category.}.

Then, we define a function called \function{HelloWorld}. In MyHDL,
classic functions are used to model hardware modules. In particular,
the parameter list is used to define the interface. In this first
example, the interface is empty.

Inside the top level function we declared a local function called
\function{sayHello} that defines the desired behavior. This function
is decorated with an \function{@always} decorator that has a delay
object as its parameter.  The meaning is that the function will be
executed whenever the specified delay has expired.

Behind the curtains, the \function{@always} decorator creates a Python
\emph{generator} and reuses the name of the decorated function for
it. Generators are the fundamental objects in MyHDL, and we will say
much more about them further on.

Finally, the top level function returns the local generator. This code
pattern is the simplest incarnation of the basic MyHDL code pattern
to define the contents of a hardware module. We will describe the
general case further on.

In MyHDL, we create an \emph{instance} of a hardware module by calling
the corresponding function. In the example, variable \code{inst} refers
to an instance of \function{HelloWorld}.  To simulate the instance, we
pass it as an argument to a \class{Simulation} object constructor.  We
then run the simulation for the desired amount of timesteps.

\section{Signals, ports, and concurrency \label{intro-conc}}

In the previous section, we simulated a design that consisted
of a single generator. Of course,
real hardware descriptions are not like that: they are
typically massively concurrent. \myhdl\ supports this by allowing an
arbitrary number of concurrent generators. 

With concurrency comes the problem of deterministic
communication. Hardware languages use special objects to
support deterministic communication between concurrent code.
For this purpose \myhdl\
has a \class{Signal} object which is roughly modeled after VHDL
signals.

We will demonstrate the use of signals and the concept of concurrency
by extending and modifying our first example. We define two hardware
modules, one that drives a clock signal, and one that is sensitive
to a positive edge on a clock signal:


\begin{verbatim}
from myhdl import Signal, delay, always, now, Simulation


def ClkDriver(clk):

    @always(delay(10))
    def driveClk():
        clk.next = not clk

    return driveClk


def HelloWorld(clk):
    
    @always(clk.posedge)
    def sayHello():
        print "%s Hello World!" % now()

    return sayHello


clk = Signal(0)
clkdriver_inst = ClkDriver(clk)
hello_inst = HelloWorld(clk)
sim = Simulation(clkdriver_inst, hello_inst)
sim.run(50)
\end{verbatim}

The clock driver function \function{ClkDriver} has a
clock signal as its parameter. This is how a
\emph{port} is modelled in MyHDL. The function
defines a generator
that continuously toggles a clock signal after a certain delay.
A new value of a signal is specified by assigning to its
\code{next} attribute. This is the \myhdl\ equivalent of 
    \index{VHDL!signal assignment}%
the VHDL signal assignment and the 
    \index{Verilog!non-blocking assignment}%
Verilog non-blocking assignment.

The \function{HelloWorld} function is modified from the
first example. It now also takes a clock signal as parameter.
Its generator is made sensitive to a rising
    \index{wait!for a rising edge}%
edge of the clock signal. This is specified by the
\code{posedge} attribute of a signal. The edge
specifier is the argument of the \code{@always}
decorator. As a result, the decorated function
will be executed on every rising clock edge.

The \code{clk} signal is constructed with an initial value
\code{0}. When creating an instance of each to the two
hardware modules, the same clock signal is passed as
the argument. The result is that the two instances
are now connected through the clock signal.
The \class{Simulation} object is constructed with the
two instances.

When we run the simulation, we get:

\begin{verbatim}
% python hello2.py
10 Hello World!
30 Hello World!
50 Hello World!
_SuspendSimulation: Simulated 50 timesteps
\end{verbatim}


\section{Parameters and hierarchy \label{intro-hier}}

We have seen that MyHDL uses functions to model hardware
modules. We have also seen that ports are modeled by using
signals as parameters. To make designs reusable we will also
want to use other objects as parameters. For example, we can
change the clock generator function to make it more general
and reusable, by making the clock period parametrizable, as
follows:

\begin{verbatim}
from myhdl import Signal, delay, instance, always, now, Simulation

def ClkDriver(clk, period=20):
    
    lowTime = int(period/2)
    highTime = period - lowTime

    @instance
    def driveClk():
        while True:
            yield delay(lowTime)
            clk.next = 1
            yield delay(highTime)
            clk.next = 0

    return driveClk
\end{verbatim}

In addition to the clock signal, the clock
\var{period} is a parameter with a default value of \code{20}.

As the low time of the clock may differ from the high time in case of
an odd period, we cannot use the \function{@always} decorator with a
single delay value anymore. Instead, the \function{driveClk} function
is now a generator function with an explicit definition of the desired
behavior. You can see that \function{driveClk} is a generator function (as
opposed to a classic function) because it contains \code{yield}
statements.

When a generator function is called, it returns a generator object. In
fact, this is mainly what the \function{@instance} decorator does. It
is less sophisticated than the \function{@always} decorator,
but it can be used to create a generator from any local generator
function.

The \code{yield} statement is a general Python construct, but MyHDL
uses it in a dedicated way.  In MyHDL, it has a similar meaning as the
wait statement in VHDL: the statement suspends execution of a
generator, and its clauses specify the conditions on which the
generator should wait before resuming. In this case, the generator
waits for a certain delay.

Not that to make sure that the generator runs ``forever'', we wrap its
behavior in a \code{while True} loop.

Likewise, we can define a general \function{Hello} function as follows:

\begin{verbatim}
def Hello(clk, to="World!"):

    @always(clk.posedge)
    def sayHello():
        print "%s Hello %s" % (now(), to)

    return sayHello
\end{verbatim}


We can create any number of instances by calling the functions with
the appropriate parameters. Hierarchy can be modeled by defining the
instances in a higher-level function, and returning them.
This pattern can be repeated for an arbitrary number of
hierarhical levels. Consequently, the general definition
of a \myhdl\ \dfn{instance} is recursive: an instance
   \index{instance!defined}%
is either a sequence of instances, or a generator.

As an example, we will create a higher-level function with
four instances of the lower-level functions, and simulate it:

\begin{verbatim}
def greetings():

    clk1 = Signal(0)
    clk2 = Signal(0)
    
    clkdriver_1 = ClkDriver(clk1) # positional and default association
    clkdriver_2 = ClkDriver(clk=clk2, period=19) # named assocation 
    hello_1 = Hello(clk=clk1) # named and default association
    hello_2 = Hello(to="MyHDL", clk=clk2) # named assocation

    return clkdriver_1, clkdriver_2, hello_1, hello_2


inst = greetings()
sim = Simulation(inst)
sim.run(50)
\end{verbatim}

As in standard Python, positional or named parameter association can
be used in instantiations, or a mix of both\footnote{All positional
parameters have to go before any named parameter.}. All these styles
are demonstrated in the example above. As in hardware description
languages, named association can be very useful if there are a lot of
parameters, as the argument order in the call does not matter in that
case.

The simulation produces the following output:

\begin{verbatim}
% python greetings.py
9 Hello MyHDL
10 Hello World!
28 Hello MyHDL
30 Hello World!
47 Hello MyHDL
50 Hello World!
_SuspendSimulation: Simulated 50 timesteps
\end{verbatim}


\begin{notice}[warning]
Some commonly used terminology has different meanings
in Python versus hardware design. Rather than artificially
changing terminology, I think it's best to keep it
and explicitly describing the differences.

A \dfn{module} in Python refers to all source code
in a particular file. A module can be reused by
other modules by importing. In hardware design,
\index{module!in Python versus hardware design}%
a module is  a reusable block of hardware with
a well defined interface. It can be reused in 
another module by \dfn{instantiating} it.

An \dfn{instance} in Python (and other object-oriented
languages) refers to the object created by a
\index{instance!in Python versus hardware design}%
class constructor. In hardware design, an instance
is a particular incarnation of a hardware module.

Normally, the meaning should be clear from
the context. Occasionally, I may qualify terms 
with the words 'hardware' or '\myhdl{}' to 
avoid ambiguity.
\end{notice}


\section{Bit oriented operations \label{intro-bit}}

Hardware design involves dealing with bits and bit-oriented
operations. The standard Python type \class{int} has most of the
desired features, but lacks support for indexing and slicing. For this
reason, \myhdl\ provides the \class{intbv} class. The name was chosen
to suggest an integer with bit vector flavor.

Class \class{intbv} works transparently as an integer and with other
integer-like types. Like class \class{int}, it provides access to the
underlying two's complement representation for bitwise
operations. In addition, it is a mutable type that provides indexing
and slicing operations, and some additional bit-oriented support such
as concatenation.

\subsection{Bit indexing \label{intro-indexing}}
\index{bit indexing}

As an example, we will consider the design of a Gray encoder. The
following code is a Gray encoder modeled in \myhdl{}:

\begin{verbatim}
from myhdl import Signal, delay, Simulation, always_comb, instance, intbv, bin

def bin2gray(B, G, width):
    """ Gray encoder.

    B -- input intbv signal, binary encoded
    G -- output intbv signal, gray encoded
    width -- bit width
    """
    
    @always_comb
    def logic():
        for i in range(width):
            G.next[i] = B[i+1] ^ B[i]
            
    return logic
\end{verbatim}

This code introduces a few new concepts. The string in triple quotes
at the start of the function is a \dfn{doc string}. This is standard
Python practice for structured documentation of code.

Furthermore, we introduce a third decorator: \function{@always_comb}.
It is used with a classic function and specifies that the 
resulting generator should
    \index{wait!for a signal value change}%
wait for a value change on any input signal. This is typically used to
describe 
    \index{combinatorial logic}%
combinatorial logic. The \function{@always_comb} decorator
automatically infers which signals are used as inputs.

Finally, the code contains bit indexing operations and an exclusive-or
operator as required for a Gray encoder. By convention, the lsb of an
\class{intbv} object has index~\code{0}.

To verify the Gray encoder, we write a test bench that prints input
and output for all possible input values:

\begin{verbatim}
def testBench(width):
    
    B = Signal(intbv(0))
    G = Signal(intbv(0))
    
    dut = traceSignals(bin2gray, B, G, width)

    @instance
    def stimulus():
        for i in range(2**width):
            B.next = intbv(i)
            yield delay(10)
            print "B: " + bin(B, width) + "| G: " + bin(G, width)

    return dut, stimulus
\end{verbatim}

We use the conversion function \code{bin} to get a binary
string representation of the signal values. This function is exported
by the \code{myhdl} package and complements the standard Python
\code{hex} and \code{oct} conversion functions.

To demonstrate, we set up a simulation for a small width: 

\begin{verbatim}
sim = Simulation(testBench(width=3))
sim.run()
\end{verbatim}

The simulation produces the following output:

\begin{verbatim}
% python bin2gray.py
B: 000 | G: 000
B: 001 | G: 001
B: 010 | G: 011
B: 011 | G: 010
B: 100 | G: 110
B: 101 | G: 111
B: 110 | G: 101
B: 111 | G: 100
StopSimulation: No more events
\end{verbatim}

\subsection{Bit slicing \label{intro-slicing}}
\index{bit slicing}

For a change, we will use a plain function as an example to illustrate
slicing.  The following function calculates the HEC byte of an ATM
header.

\begin{verbatim}
from myhdl import intbv, concat

COSET = 0x55

def calculateHec(header):
    """ Return hec for an ATM header, represented as an intbv.

    The hec polynomial is 1 + x + x**2 + x**8.
    """
    hec = intbv(0)
    for bit in header[32:]:
        hec[8:] = concat(hec[7:2],
                         bit ^ hec[1] ^ hec[7],
                         bit ^ hec[0] ^ hec[7],
                         bit ^ hec[7]
                        )
    return hec ^ COSET
\end{verbatim}

The code shows how slicing access and assignment is supported on the
\class{intbv} data type. In accordance with the most common hardware
convention, and unlike standard Python, slicing ranges are
downward. The code also demonstrates concatenation of \class{intbv}
objects.

As in standard Python, the slicing range is half-open: the highest
index bit is not included. Unlike standard Python however, this index
corresponds to the \emph{leftmost} item. Both indices can be omitted
from the slice. If the leftmost index is omitted, the meaning is to
access ``all'' higher order bits.  If the rightmost index is omitted,
it is \code{0} by default.

The half-openness of a slice may seem awkward at first, but it helps
to avoid one-off count issues in practice. For example, the slice
\code{hex[8:]} has exactly \code{8} bits. Likewise, the slice
\code{hex[7:2]} has \code{7-2=5} bits. You can think about it as
follows: for a slice \code{[i:j]}, only bits below index \code{i} are
included, and the bit with index \code{j} is the last bit included.

When an intbv object is sliced, a new intbv object is returned. This
new intbv always has a positive value, even when the original object
was negative.

\section{Summary and perspective}


\section{Some remarks on \myhdl\ and Python \label{intro-python}}

To conclude this introductory chapter, it is useful to stress that
\myhdl\ is not a language in itself. The underlying language is Python, 
and \myhdl\ is implemented as a Python package called \code{myhdl}.
Moreover, it is a design goal to keep the \code{myhdl} package as
minimalistic as possible, so that \myhdl\ descriptions are very much
``pure Python''.

To have Python as the underlying language is significant in several
ways:

\begin{itemize}

\item Python is a very powerful high level language. This translates
into high productivity and elegant solutions to complex problems.

\item  Python is continuously improved by some very clever 
minds, supported by a large and fast growing user base. Python profits
fully from the open source development model.

\item Python comes with an extensive standard library. Some
functionality is likely to be of direct interest to \myhdl\ users:
examples include string handling, regular expressions, random number
generation, unit test support, operating system interfacing and GUI
development. In addition, there are modules for mathematics, database
connections, networking programming, internet data handling, and so
on.

\item Python has a powerful C extension model. All built-in types are
written with the same C API that is available for custom
extensions. To a module user, there is no difference between a
standard Python module and a C extension module --- except
performance. The typical Python development model is to prototype
everything in Python until the application is stable, and (only) then
rewrite performance critical modules in C if necessary.

\end{itemize}





\chapter{Modeling techniques}

\section{RTL modeling}
The present section describes how \myhdl\ supports RTL style modeling
as is typically used for synthesizable models in Verilog or VHDL.

In this domain, \myhdl\ doesn't offer advantages compared to
other solutions. However, as this modeling style is well-known,
this section may be useful for illustrative purposes.

\subsection{Combinatorial logic}

\subsubsection{Template}

Combinatorial logic is described with a generator function code template as
follows: 

\begin{verbatim}
def combinatorialLogic(<arguments>)
    while 1:
        yield <input signal arguments>
        <functional code>

\end{verbatim}

The overall code is wrapped in a \code{while 1} statement to keep the
generator alive. All input signals are clauses in the \code{yield}
statement, so that the generator resumes whenever one of the inputs
changes. 

\subsubsection{Example}

The following is an example of a combinatorial multiplexer:

\begin{verbatim}
def mux(z, a, b, sel):
    """ Multiplexer.
    
    z -- mux output
    a, b -- data inputs
    sel -- control input: select a if asserted, otherwise b
    """
    while 1:
        yield a, b, sel
        if sel == 1:
            z.next = a
        else:
            z.next = b
\end{verbatim}

To verify, let's simulate this logic with some random patterns. The
\code{random} module in Python's standard library comes in handy for
such purposes. The function \code{randrange(\var{n})} returns a random
natural integer smaller than \var{n}. It is used in the test bench
code to produce random input values:

\begin{verbatim}
from random import randrange

(z, a, b, sel) = [Signal(0) for i in range(4)]

MUX_1 = mux(z, a, b, sel)

def test():
    print "z a b sel"
    for i in range(8):
        a.next, b.next, sel.next = randrange(8), randrange(8), randrange(2)
        yield delay(10)
        print "%s %s %s %s" % (z, a, b, sel)
        
Simulation(MUX_1, test()).run() 
   
\end{verbatim}

Because of the randomness, the simulation output varies between runs
\footnote{It also possible to have a reproducible random output, by
explicitly providing a seed value. See the documentation of the
\code{random} module}. One particular run produced the following
output:

\begin{verbatim}
% python mux.py
z a b sel
6 6 1 1
7 7 1 1
7 3 7 0
1 2 1 0
7 7 5 1
4 7 4 0
4 0 4 0
3 3 5 1
StopSimulation: No more events
\end{verbatim}


\subsection{Sequential logic}

\subsubsection{Template}
Sequential RTL models are sensitive to a clock edge. In addition, they
may be sensitive to a reset signal. We will describe one of the most
common patterns: a template with a rising clock edge and an
asynchronous reset signal. Other templates are similar.

\begin{verbatim}
def sequentialLogic(<arguments>, clock, ..., reset, ...)
    while 1:
        yield posedge(clock), negedge(reset)
        if reset == <active level>:
            <reset code>
        else:
            <functional code>

\end{verbatim}


\subsubsection{Example}
The following code is a description of an incrementer with enable, and
an asynchronous power-up reset.

\begin{verbatim}
ACTIVE_LOW, INACTIVE_HIGH = 0, 1

def Inc(count, enable, clock, reset, n):
    """ Incrementer with enable.
    
    count -- output
    enable -- control input, increment when 1
    clock -- clock input
    reset -- asynchronous reset input
    n -- counter max value
    """
    while 1:
        yield posedge(clock), negedge(reset)
        if reset == ACTIVE_LOW:
            count.next = 0
        else:
            if enable:
                count.next = (count + 1) % n

\end{verbatim}

For the test bench, we will use an independent clock generator, stimulus
generator, and monitor. After applying enough stimulus patterns, we
can raise the \code{myhdl.StopSimulation} exception to stop the
simulation run. The test bench for a small incrementer and a small
number of patterns is a follows:

\begin{verbatim}
count, enable, clock, reset = [Signal(intbv(0)) for i in range(4)]

INC_1 = Inc(count, enable, clock, reset, n=4)

def clockGen():
    while 1:
        yield delay(10)
        clock.next = not clock

def stimulus():
    reset.next = ACTIVE_LOW
    yield negedge(clock)
    reset.next = INACTIVE_HIGH
    for i in range(12):
        enable.next = min(1, randrange(3))
        yield negedge(clock)
    raise StopSimulation

def monitor():
    print "enable  count"
    yield posedge(reset)
    while 1:
        yield posedge(clock)
        yield delay(1)
        print "   %s      %s" % (enable, count)
        
Simulation(clockGen(), stimulus(), monitor(), INC_1).run()

\end{verbatim}

The simulation produces the following output:
\begin{verbatim}
% python inc.py
enable  count
   0      0
   1      1
   0      1
   1      2
   1      3
   1      0
   0      0
   1      1
   0      1
   0      1
   0      1
   1      2
StopSimulation

\end{verbatim}


\section{High level modeling}

test

\chapter{Unit testing}

\section{Introduction}

Many aspects in the design flow of modern digital hardware design can
be viewed as a special kind of software development. From that
viewpoint, it is a natural question whether advances in software
design techniques can not also be applied to hardware design.

One software design approach that gets a lot of attention recently is
\emph{Extreme Programming} (XP). It is a fascinating set of techniques and
guidelines that often seems to go against the conventional wisdom. On
other occasions, XP just seems to emphasize the common sense, which
doesn't always coincide with common practice. For example, XP stresses
the importance of normal workweeks, if we are to have the
fresh mind needed for good software development.

It is not my intention nor qualification to present a tutorial on
Extreme Programming. Instead, in this section I will highlight one XP
concept which I think is very relevant to hardware design: the
importance and methodology of unit testing.

\section{The importance of unit tests}

Unit testing is one of the corner stones of Extreme Programming. Other
XP concepts, such as collective ownership of code and continuous
refinement, are only possible by having unit tests. Moreover, XP
emphasizes that writing unit tests should be automated, that they should
test everything in every class, and that they should run perfectly all
the time. 

I believe that these concepts apply directly to hardware design. In
addition, unit tests are a way to manage simulation time. For example,
a state machine that runs very slowly on infrequent events may be
impossible to verify at the system level, even on the fastest
simulator. On the other hand, it may be easy to verify it exhaustively
in a unit test, even on the slowest simulator.

It is clear that unit tests have compelling advantages. On the other
hand, if we need to test everything, we have to write
lots of unit tests. So it should be easy and pleasant
to create, manage and run them. Therefore, XP emphasizes the need for
a unit test framework that supports these tasks. In this chapter,
we will explore the use of the \code{unittest} module from
the standard Python library for creating unit tests for hardware
designs.


\section{Unit test development}

In this section, we will informally explore the application of unit
test techniques to hardware design. We will do so by a (small)
example: testing a binary to Gray encoder as introduced in
section~\ref{gray}. 

\subsection{Defining the requirements}

We start by defining the requirements. For a Gray encoder, we want to
the output to comply with Gray code characteristics. Let's define a
\dfn{code} as a list of \dfn{codewords}, where a codeword is a bit
string. A code of order \code{n} has \code{2**n} codewords.

A well-known characteristic is the one that Gray codes are all about:

\newtheorem{reqGray}{Requirement}
\begin{reqGray} 
Consecutive codewords in a Gray code should differ in a single bit.
\end{reqGray}

Is this sufficient? Not quite: suppose for example that an
implementation returns the lsb of each binary input. This would comply
with the requirement, but is obviously not what we want. Also, we don't
want the bit width of Gray codewords to exceed the bit width of the
binary codewords.

\begin{reqGray} 
Each codeword in a Gray code of order n must occur exactly once in the
binary code of the same order.
\end{reqGray}

With the requirements written down we can proceed.

\subsection{Writing the test first}

A fascinating guideline in the XP world is to write the unit test
first. That is, before implementing something, first write the test
that will verify it. This seems to go against our natural inclination,
and certainly against common practices. Many engineers like to
implement first and think about verification afterwards.

But if you think about it, it makes a lot of sense to deal with
verification first. Verification is about the requirements only --- so
your thoughts are not yet cluttered with implementation details. The
unit tests are an executable description of the requirements, so they
will be better understood and it will be very clear what needs to be
done. Consequently, the implementation should go smoother. Perhaps
most importantly, the test is available when you are done
implementing, and can be run anytime by anybody to verify changes.

Python has a standard \code{unittest} module that facilitates writing,
managing and running unit tests. With \code{unittest}, a test case is 
written by creating a class that inherits from
\code{unittest.TestCase}. Individual tests are created by methods of
that class: all method names that start with \code{test} are
considered to be tests of the test case.

We will define a test case for the Gray code properties, and then
write a test for each of the requirements. The outline of the test
case class is as follows:

\begin{verbatim}
from unittest import TestCase

class TestGrayCodeProperties(TestCase):

    def testSingleBitChange(self):
     """ Check that only one bit changes in successive codewords """
     ....


    def testUniqueCodeWords(self):
        """ Check that all codewords occur exactly once """
    ....
\end{verbatim}

Each method will be a small test bench that tests a single
requirement. To write the tests, we don't need an implementation of
the Gray encoder, but we do need the interface of the design. We can
specify this by a dummy implementation, as follows:

\begin{verbatim}
def bin2gray(B, G, width):
    ### NOT IMPLEMENTED YET! ###
    yield None
\end{verbatim}

For the first requirement, we will write a test bench that applies all
consecutive input numbers, and compares the current output with the
previous one for each input. Then we check that the difference is a
single bit. We will test all Gray codes up to a certain order
\code{MAX_WIDTH}.

\begin{verbatim}
    def testSingleBitChange(self):
        """ Check that only one bit changes in successive codewords """
        
        def test(B, G, width):
            B.next = intbv(0)
            yield delay(10)
            for i in range(1, 2**width):
                G_Z.next = G
                B.next = intbv(i)
                yield delay(10)
                diffcode = bin(G ^ G_Z)
                self.assertEqual(diffcode.count('1'), 1)
        
        for width in range(1, MAX_WIDTH):
            B = Signal(intbv(-1))
            G = Signal(intbv(0))
            G_Z = Signal(intbv(0))
            dut = bin2gray(B, G, width)
            check = test(B, G, width)
            sim = Simulation(dut, check)
            sim.run(quiet=1)
\end{verbatim}

Note how the actual check is performed by a \code{self.assertEqual}
method, defined by the \code{unittest.TestCase} class.

Similarly, we write a test bench for the second requirement. Again, we
simulate all numbers, and put the result in a list. The requirement
implies that if we sort the result list, we should get a range of
numbers:

\begin{verbatim}
    def testUniqueCodeWords(self):
        """ Check that all codewords occur exactly once """

        def test(B, G, width):
            actual = []
            for i in range(2**width):
                B.next = intbv(i)
                yield delay(10)
                actual.append(int(G))
            actual.sort()
            expected = range(2**width)
            self.assertEqual(actual, expected)
       
        for width in range(1, MAX_WIDTH):
            B = Signal(intbv(-1))
            G = Signal(intbv(0))
            dut = bin2gray(B, G, width)
            check = test(B, G, width)
            sim = Simulation(dut, check)
            sim.run(quiet=1)
\end{verbatim}


\subsection{Test-driven implementation}

With the test written, we begin with the implementation. For
illustration purposes, we will intentionally write some incorrect
implementations to see how the test behaves.

The easiest way to run tests defined with the \code{unittest}
framework, is to put a call to its \code{main} method at the end of
the test module:

\begin{verbatim}
unittest.main()
\end{verbatim}

Let's run the test using the dummy Gray encoder shown earlier:

\begin{verbatim}
% python test_gray.py -v
Check that only one bit changes in successive codewords ... FAIL
Check that all codewords occur exactly once ... FAIL
<trace backs not shown>
\end{verbatim}

As expected, this fails completely. Let us try an incorrect
implementation, that puts the lsb of in the input on the output:

\begin{verbatim}
def bin2gray(B, G, width):
    ### INCORRECT - DEMO PURPOSE ONLY! ###
    while 1:
        yield B
        G.next = B[0]
\end{verbatim}


Running the test produces:

\begin{verbatim}
% python test_gray.py -v
Check that only one bit changes in successive codewords ... ok
Check that all codewords occur exactly once ... FAIL

======================================================================
FAIL: Check that all codewords occur exactly once
----------------------------------------------------------------------
Traceback (most recent call last):
  File "test_gray.py", line 109, in testUniqueCodeWords
    sim.run(quiet=1)
...
  File "test_gray.py", line 104, in test
    self.assertEqual(actual, expected)
  File "/usr/local/lib/python2.2/unittest.py", line 286, in failUnlessEqual
    raise self.failureException, \
AssertionError: [0, 0, 1, 1] != [0, 1, 2, 3]

----------------------------------------------------------------------
Ran 2 tests in 0.785s
\end{verbatim}

Now the test passes the first requirement, as expected, but fails the
second one. After the test feedback, a full traceback is shown that
can help to debug the test output.

Finally, if we use the correct implementation as in
section~\ref{gray}, the output is:

\begin{verbatim}
% python test_gray.py -v
Check that only one bit changes in successive codewords ... ok
Check that all codewords occur exactly once ... ok

----------------------------------------------------------------------
Ran 2 tests in 6.364s

OK
\end{verbatim}



\subsection{Changing requirements}

In the previous section, we concentrated on the general requirements
of a Gray code. It is possible to specify these without specifying the
actual code. It is easy to see that there are several codes
that satisfy these requirements. In good XP style, we only tested
the requirements and nothing more.

It may be that more control is needed. For example, the requirement
may be for a particular code, instead of compliance with general
properties. As an illustration, we will show how to test for
\emph{the} original Gray code, which is one specific instance that
satisfies the requirements of the previous section. In this particular
case, this test will actually be easier than the previous one.

We denote the original Gray code of order \code{n} as \code{Ln}. Some
examples: 

\begin{verbatim}
L1 = ['0', '1']
L2 = ['00', '01', '11', '10']
L3 = ['000', '001', '011', '010', '110', '111', '101', 100']
\end{verbatim}

It is possible to specify these codes by a recursive algorithm, as
follows:

\begin{enumerate}
\item L1 = ['0', '1']
\item Ln+1 can be obtained from Ln as follows. Create a new code Ln0 by
prefixing all codewords of Ln with '0'. Create another new code Ln1 by
prefixing all codewords of Ln with '1', and reversing their
order. Ln+1 is the concatenation of Ln0 and Ln1.
\end{enumerate}

Python is well-known  for its elegant algorithmic
descriptions, and this is a good example. We can write the algorithm
in Python as follows:

\begin{verbatim}
def nextLn(Ln):
    """ Return Gray code Ln+1, given Ln. """
    Ln0 = ['0' + codeword for codeword in Ln]
    Ln1 = ['1' + codeword for codeword in Ln]
    Ln1.reverse()
    return Ln0 + Ln1
\end{verbatim}

The code \samp{['0' + codeword for ...]} is called a \dfn{list
comprehension}. It is a concise way to describe lists built by short
computations in a for loop.

The requirement is now that the output code matches the
expected code Ln. We use the \code{nextLn} function to compute the
expected result. The new test case code is as follows:

\begin{verbatim}
class TestOriginalGrayCode(TestCase):

    def testOriginalGrayCode(self):
        """ Check that the code is an original Gray code """

        Rn = []
        
        def stimulus(B, G, n):
            for i in range(2**n):
                B.next = intbv(i)
                yield delay(10)
                Rn.append(bin(G, width=n))
        
        Ln = ['0', '1'] # n == 1
        for n in range(2, MAX_WIDTH):
            Ln = nextLn(Ln)
            del Rn[:]
            B = Signal(intbv(-1))
            G = Signal(intbv(0))
            dut = bin2gray(B, G, n)
            stim = stimulus(B, G, n)
            sim = Simulation(dut, stim)
            sim.run(quiet=1)
            self.assertEqual(Ln, Rn)
\end{verbatim}

As it happens, our implementation is apparently an original Gray code:

\begin{verbatim}
% python test_gray.py -v TestOriginalGrayCode
Check that the code is an original Gray code ... ok

----------------------------------------------------------------------
Ran 1 tests in 3.091s

OK
\end{verbatim}


 


\chapter{MyHDL as a hardware verification language}

\section{Introduction}

One of the most exciting possibilities of \myhdl\
is to use it as a hardware verification language (HVL).
A HVL is a language used to write test benches and
verification environments, and to control simulations.

Nowadays, it is generally acknowledged that HVLs should be equipped
with modern software techniques, such as object orientation. The
reason is that verification it the most complex and time-consuming
task of the design process: consequently every useful technique is
welcome. Moreover, test benches are not required to be
implementable. Therefore, unlike synthesizable code, there
are no constraints on creativity.

Technically, verification of a design implemented in
another language requires cosimulation. \myhdl\ is 
enabled for cosimulation with any HDL simulator that
has a procedural language interface (PLI). The \myhdl\
side is designed to be independent of a particular
simulator, On the other hand, for each HDL simulator a specific
PLI module will have to be written in C. Currently,
the \myhdl\ release contains a PLI module to interface
to the Icarus Verilog simulator. This interface will
be used in the examples.

\section{The HDL side}

To introduce cosimulation, we will continue to use the Gray encoder
example from the previous chapters. Suppose that we want to
synthesize it and write it in Verilog for that purpose. Clearly we would
like to reuse our unit test verification environment. This is exactly
what \myhdl\ offers.

To start, let's recall how the Gray encoder in \myhdl{} looks like:

\begin{verbatim}
def bin2gray(B, G, width):
    """ Gray encoder.

    B -- input intbv signal, binary encoded
    G -- output intbv signal, gray encoded
    width -- bit width
    """
    while 1:
        yield B
        for i in range(width):
            G.next[i] = B[i+1] ^ B[i]

\end{verbatim}

To show the cosimulation flow, we don't need the Verilog
implementation yet, but only the interface.  Our Gray encoder in
Verilog would have the following interface:

\begin{verbatim}
module bin2gray(B, G);

   parameter width = 8;
   input [width-1:0]  B;     
   output [width-1:0] G;
   ....

\end{verbatim}

To write a test bench, one creates a new module that instantiates the
design under test (DUT).  The test bench declares nets and
regs (or signals in VHDL) that are attached to the DUT, and to
stimulus generators and response checkers. In an all-HDL flow, the
generators and checkers are written in the HDL itself, but we will
want to write them in \myhdl{}. To make the connection, we need to
declare which regs \& nets are driven and read by the \myhdl\
simulator. For our example, this is done as follows:

\begin{verbatim}
module dut_bin2gray;

   reg [`width-1:0] B;
   wire [`width-1:0] G;

   initial begin
      $from_myhdl(B);
      $to_myhdl(G);
   end

   bin2gray dut (.B(B), .G(G));
   defparam dut.width = `width;

endmodule

\end{verbatim}

The \code{\$from_myhdl} task call declares which regs are driven by
\myhdl{}, and the \code{\$to_myhdl} task call which regs \& nets are read
by it. These tasks take an arbitrary number of arguments.  They
are defined in a PLI module written in C. They are made available to
the simulation in a simulator-dependent manner.  In Icarus Verilog,
the tasks are defined in a \code{myhdl.vpi} module that is compiled
from C source code.

\section{The \myhdl\ side}

\myhdl\ supports cosimulation by a \code{Cosimulation} object. 
A \code{Cosimulation} object must know how to run a HDL cosimulation.
Therefore, the first argument to its constructor is a command string
to execute a simulation. The way to generate and run an
simulation executable is simulator dependent.
For example, in Icarus Verilog, a simulation executable for our
example can be obtained obtained by running the \code{iverilog}
compiler as follows:

\begin{verbatim}
% iverilog -o bin2gray -Dwidth=4 bin2gray.v dut_bin2gray.v

\end{verbatim}

This generates a \code{bin2gray} executable for a parameter \code{width}
of 4, by compiling the contributing verilog files.

The simulation itself is run by the \code{vvp} command:

\begin{verbatim}
% vvp -m ./myhdl.vpi bin2gray

\end{verbatim}

This runs the \code{bin2gray} simulation, and specifies to use the
\code{myhdl.vpi} PLI module present in the current directory. (This is 
just a command line usage example; actually simulating with the
\code{myhdl.vpi} module is only meaningful from a
\code{Cosimulation} object.)

We can use a \code{Cosimulation} object to provide a HDL cosimulation
version of a design to the \myhdl\ simulator. Instead of a generator
function, we write a function that returns a \code{Cosimulation}
object. For our example and the Icarus Verilog simulator, this is done
as follows:

\begin{verbatim}
import os

from myhdl import Cosimulation

cmd = "iverilog -o bin2gray -Dwidth=%s bin2gray.v dut_bin2gray.v"
      
def bin2gray(B, G, width):
    os.system(cmd % width)
    return Cosimulation("vvp -m ./myhdl.vpi bin2gray", B=B, G=G)

\end{verbatim}

After the executable command argument, the \code{Cosimulation}
constructor takes an arbitrary number of keyword arguments. Those
arguments make the link between \myhdl\ Signals and HDL nets, regs, or
signals, by named association. The keyword is the name of the argument
in a \code{\$to_myhdl} or \code{\$from_myhdl} call; the argument is
the \myhdl\ Signal.

With all this in place, we can now use the existing unit test
to verify the Verilog implementation. Note that we kept the
same name and parameters for the the \code{bin2gray} function:
all we need to do is to provide this alternative definition
to the existing unit test.

Let's quickly try it just to be sure:

\begin{verbatim}
module bin2gray(B, G);

   parameter width = 8;
   input [width-1:0]  B;
   output [width-1:0] G;
   reg [width-1:0] G;
   integer i;

   always @(B) begin
      for (i=0; i < width-1; i=i+1)
        G[i] <= B[i+1] ^ B[i];
   end

endmodule

\end{verbatim}

If we run our unit test we get:

\begin{verbatim}

% python test_bin2gray.py   
Check that only one bit changes in successive codewords ... ERROR
Check that all codewords occur exactly once ... FAIL
Check that the code is an original Gray code ... ERROR
...

\end{verbatim}

Oops! It seems we still have a bug! Oh yes, but of course, 
we need to zero-extend the input to get the msb output bit
correctly:

\begin{verbatim}
module bin2gray(B, G);

   parameter width = 8;
   input [width-1:0]  B;
   output [width-1:0] G;
   reg [width-1:0] G;
   integer i;
   wire [width:0] extB;

   assign extB = {1'b0, B};

   always @(extB) begin
      for (i=0; i < width; i=i+1)
        G[i] <= extB[i+1] ^ extB[i];
   end

endmodule

\end{verbatim}

And now:

\begin{verbatim}
% python test_bin2gray.py 
Check that only one bit changes in successive codewords ... ok
Check that all codewords occur exactly once ... ok
Check that the code is an original Gray code ... ok

----------------------------------------------------------------------
Ran 3 tests in 2.729s

OK

\end{verbatim}


\section{Restrictions}

In the ideal case, it should be possible to simulate
any HDL description seamlessly with \myhdl{}. Moreover
the communicating signals at each side should act
transparently as a single one, enabling fully race free
operation.

For various reasons, it may not be possible or desirable
to achieve full generality. As anyone that has developed
applications with the Verilog PLI can testify, the
restrictions in a particular simulator, and the 
differences over various simulators, can be quite 
frustrating. Moreover, full generality may require
a disproportionate amount of development work compared
to a slightly less general solution that may
be sufficient for the target application.

Consequently, I have tried to achieve a solution
which is simple enough so that one can reasonably 
expect that any PLI-enabled simulator can support it,
and that is relatively easy to verify and maintain.
At the same time, the solution is sufficiently general 
to cover the target application space.

The result is a compromise that places certain restrictions
on the HDL code. In this section, these restrictions 
are presented.

\subsection{Only passive HDL can be cosimulated}

The most important restriction of the \myhdl\ cosimulation solution is
that only ``passive'' HDL can be cosimulated.  This means that the HDL
code should not contain any statements with time delays. In other
words, the \myhdl\ simulator should be the master of time; in
particular, any clock signal should be generated at the \myhdl\ side.

At first this may seem like an important restriction, but if one
considers the target application for cosimulation, it probably
isn't. 

\myhdl\ support cosimulations so that test benches for HDL
designs can be written in Python.
Let's consider the nature of the target HDL designs. For high-level,
behavioral models that are not intended for implementation, it should
come as no surprise that I would recommend to write them in \myhdl\
directly; that is exactly the target of the \myhdl\ effort. Likewise,
gate level designs with annotated timing are not the target
application: static timing analysis is a much better verification
method for such designs.

Rather, the targeted HDL designs are naturally models that are
intended for implementation.  Most likely, this will be through
synthesis. As time delays are meaningless in synthesizable code, the
restriction is compatible with the target application.

\subsection{Race sensitivity issues}

In a typical TTL code, some events cause other events to occur in the
same time step. For example, when a clock signal triggers some signals
may change in the same time step. For race-free operation, an HDL
must differentiate between such events within a time step. This is done
by the concept of ``delta'' cycles. In a fully general, race free
cosimulation, the cosimulators would communicate at the level of delta
cycles. However, in \myhdl\ cosimulation, this is not entirely the
case.

Delta cycles from the \myhdl\ simulator toward the HDL cosimulator are
preserved. However, in the opposite direction, they are not. The
signals changes are only returned to the \myhdl\ simulator after all delta
cycles have been performed in the HDL cosimulator.

What does this mean? Let's start with the good news. As explained in
the previous section, the logic of the \myhdl\ cosimulation implies
that clocks are generated at the \myhdl\ side.  \emph{When using a
\myhdl\ clock and its corresponding HDL signal directly as a clock,
cosimulation operation is race free.} In other words, the case
that most closely reflects the \myhdl\ cosimulation approach, is race free.

The situation is different when you want to use a signal driven by the
HDL (and the corresponding MyHDL signal) as a clock. 
Communication triggered by such a clock is not race free. The solution
is to treat such an interface as a chip interface instead of an RTL
interface.  For example, when data is triggered at positive clock
edges, it can safely be sampled at negative clock edges.
Alternatively, the \myhdl\ data signals can be declared with a delay
value, so that they are guaranteed to change after the clock
edge.


\section{Implementation notes}

\begin{quote}
\em
This section requires some knowledge of PLI terminology.
\end{quote}

Enabling a simulator for cosimulation requires a PLI module
written in C. In Verilog, the PLI is part of the ``standard''.
However, different simulators implement different versions 
and portions of the standard. Worse yet, the behavior of
certain PLI callbacks is not defined on some essential points. 
As a result, one should plan to write a specific PLI module
for any simulator.

The present release contains a PLI module for the 
open source Icarus simulator. I would like to add
modules for any popular simulator in the future,
either from external contributions, or by getting
access to them myself. The same holds for VHDL
simulators: it would be great to have an interface
to the Modelsim VHDL simulator.

This section documents
the current approach and status of the PLI module
implementation in Icarus, and some reflections
on future implementations in other simulators.

\subsection{Icarus Verilog}

To make cosimulation work, a specific type of PLI callback is
needed. The callback should be run when all pending events have been
processed, while allowing the creation of new events in the current
time step (e.g. by the \myhdl\ simulator).  In some Verilog simulators,
the \code{cbReadWriteSync} callback does exactly that. However,
in others, including Icarus, it does not. The callback's behavior is
not fully standardized; some simulators run the callback before
non-blocking assignment events have been processed.

Consequently, I had to look for a workaround. One half of the solution
is to use the \code{cbReadOnlySync} callback.  This callback runs
after all pending events have been processed.  However, it does not
permit to create new events in the current time step.  The second half
of the solution is to map \myhdl\ delta cycles onto Verilog time steps.
Note that there is some freedom here because of the restriction that
only passive HDL code can be cosimulated.

I chose to make the time granularity in the Verilog simulator a 1000
times finer than in the \myhdl{} simulator. For each \myhdl\ time step,
1000 Verilog time steps are available for  \myhdl\ delta cycles. In practice,
only a few delta cycles per time step should be needed. More than 1000
almost certainly indicates an error. This limit is checked at
run-time. The factor 1000 also makes it easy to distinguish ``real''
time from delta cycle time when printing out the Verilog time.

\subsection{Other Verilog simulators}

The Icarus module is written with VPI calls, which are provided by the
most recent generation of the Verilog PLI. Some simulators may only
support TF/ACC calls, requiring a complete redesign of the interface
module.

If the simulator supports VPI, the Icarus module should be reusable to
a large extent. However, it may be possible to improve on it.  The
workaround described in the previous section may not be necessary. In
some simulators, the \code{cbReadWriteSync} callback occurs after all
events (including non-blocking assignments) have been processed. In
that case, the functionality can be supported without a finer time
granularity in the Verilog simulator.

There are also Verilog standardization efforts underway to resolve the
ambiguity of the \code{cbReadWriteSync} callback. The solution will be
to introduce new, well defined callbacks. From reading some proposals,
I conclude that the \code{cbEndOfSimTime} callback would provide the
required functionality.

\subsection{VHDL}

It would be great to have an interface to the Modelsim VHDL
simulator. This will require a redesign from scratch with the
appropriate PLI.  One feature which I would like to keep if possible
is the way to declare the communicating signals.  In the Verilog
solution, it is not necessary to define and instantiate any special
entity (module). Rather, the participating signals can be declared
directly in the \code{to_myhdl} and \code{from_myhdl} task calls.


\chapter{Conversion to Verilog\label{conv}}
\section{Introduction\label{conv-intro}}

MyHDL 0.4 provides a path to automatic implementation, by converting
a subset of MyHDL code into synthesizable Verilog code.

MyHDL aims to be a complete design language, for high level modeling,
verification, but also for implementation. However, prior to \myhdl\
0.4 a \myhdl\ user had to translate synthesizable code manually to
Verilog or VHDL. Needless to say, this is inconvenient. With \myhdl\
0.4, this manual step should no longer be necessary.  The automatic
conversion provides a direct path from Python to an FPGA or ASIC
implementation.

\section{Solution description\label{conv-solution}}

The solution works as follows. The hardware description should be
modeled in \myhdl\ style, and satisfy certain constraints that are
typical for implementation-oriented hardware modeling.  Subsequently,
such a design is converted to an equivalent model in the Verilog
language, using the function \function{toVerilog} from the \myhdl\
library. Finally, a third-party \emph{synthesis tool} is used to
convert the Verilog design into a gate implementation for an ASIC or
FPGA. There are a number of Verilog synthesis tools available, varying
in price, capabilities, and target implementation space.

The conversion does not start from source files, but from a design
that has been "elaborated" by the Python interpreter.  This has
important advantages. First, there are no restrictions on how to
describe structure, as all "structural" constructs and parameters are
processed before the conversion starts. Second, the work of the Python
interpreter is "reused". The converter uses the Python profiler to
track the interpreter's operation and to infer the design structure
and name spaces. It then selectively compiles pieces of source code
for additional analysis and for conversion. This is done using the
Python compiler package.

\section{Features\label{conv-features}}

\subsection{The design is converted after elaboration\label{conv-features-elab}}
\emph{Elaboration} refers to the initial processing of a hardware
description to achieve a representation of a design instance that is
ready for simulation or synthesis. In particular, structural
parameters and constructs are processed in this step. In \myhdl{}, the
Python interpreter itself is used for elaboration.  A
\class{Simulation} object is constructed with elaborated design
instances as arguments.  Likewise, the Verilog conversion works on an
elaborated design instance. The Python interpreter is thus used as
much as possible.

\subsection{The structural description can be arbitrarily complex and hierarchical\label{conv-features-struc}}
As the conversion works on an elaborated design instance, any modeling
constraints only apply to the leaf elements of the design structure,
that is, the co-operating generators. In other words, there are no
restrictions on the description of the design structure: Python's full
power can be used for that purpose. Also, the design hierarchy can be
arbitrarily deep.

\subsection{Generators are mapped to Verilog always or initial blocks\label{conv-features-gen}}
The converter analyzes the code of each generator and maps it
into a Verilog \code{always} blocks if possible, and to 
an \code{initial} block otherwise.
The converted Verilog design will be a flat
"net list of blocks".

\subsection{The Verilog interface is inferred from signal usage\label{conv-features-intf}}
In \myhdl{}, the input or output direction of interface signals
is not explicitly declared. The converter investigates signal usage
in the design hierarchy to infer whether a signal is used as an
input, output, or an internal signal. Internal signals are
given a hierarchical name in the Verilog output.

\subsection{Function calls are mapped to a unique Verilog function or task\label{conv-features-func}}
The converter analyzes function calls and function code to see if they
should be mapped to Verilog functions or to tasks. Python functions
are much more powerful than Verilog subprograms; for example, they are
inherently generic, and they can be called with named association.  To
support this power in Verilog, a unique Verilog function or task is
generated per Python function call.

\subsection{If-then-else structures may be mapped to Verilog case statements\label{conv-features-if}}
Python does not provide a case statement. However, 
the converter recognizes if-then-else structures in which a variable is
sequentially compared to items of an enumeration type, and maps
such a structure to a Verilog case statement with the appropriate
synthesis attributes.

\subsection{Choice of encoding schemes for enumeration types\label{conv-features-enum}}
The \function{enum} function in \myhdl\ returns an enumeration type. This
function takes an additional parameter \var{encoding} that specifies the
desired encoding in the implementation: binary, one hot, or one cold.
The Verilog converter generates the appropriate code.


\section{The convertible subset\label{conv-subset}}

\subsection{Introduction\label{conv-subset-intro}}

Unsurprisingly, not all Python code can be converted into Verilog. In
fact, there are very important restrictions.  As the goal of the
conversion functionality is implementation, this should not be a big
issue: anyone familiar with synthesis is used to similar restrictions
in the \emph{synthesizable subset} of Verilog and VHDL. The converter
attempts to issue clear error messages when it encounters a construct
that cannot be converted. 

In practice, the synthesizable subset usually refers to RTL synthesis,
which is by far the most popular type of synthesis today. There are
industry standards that define the RTL synthesis subset.  However,
those were not used as a model for the restrictions of the MyHDL
converter, but as a minimal starting point.  On that basis, whenever
it was judged easy or useful to support an additional feature, this
was done. For example, it is actually easier to convert while loops
than for loops even though they are not RTL-synthesizable.  As another
example, \keyword{print} is supported because it's so useful for
debugging, even though it's not synthesizable.  In summary, the
convertible subset is a superset of the standard RTL synthesis subset,
and supports synthesis tools with more advanced capabilities, such as
behavioral synthesis.

Recall that any restrictions only apply to the design post
elaboration.  In practice, this means that they apply only to the code
of the generators, that are the "leaf" functional blocks in a MyHDL
design.

\subsection{Coding style\label{conv-subset-style}}

A natural restriction on convertible code is that it should be
written in MyHDL style: cooperating generators, communicating through
signals, and with \code{yield} statements specifying wait points and resume
conditions.  Supported resume conditions are a signal edge, a signal
change, or a tuple of such conditions.

\subsection{Supported types\label{conv-subset-types}}

The most important restriction regards object types. Verilog is an
almost typeless language, while Python is strongly (albeit
dynamically) typed. The converter needs to infer the types of
variables and map them to Verilog types. Therefore, it does type
inferencing of object constructors and expressions.

Only a limited amount of types can be converted.
Python \class{int} and \class{long} objects are mapped to Verilog
integers. All other supported types are mapped to Verilog regs (or
wires), and therefore need to have a defined bit width. The supported
types are the Python \class{bool} type, the MyHDL \class{intbv} type,
and MyHDL enumeration types returned by function \function{enum}. The
latter objects can also be used as the base object of a
\class{Signal}. 

\class{intbv} objects need to be constructed so that a bit
width can be inferred. This can be done by specifying minimum
and maximum values, e.g. as follows:

\begin{verbatim}
index = intbv(0, min=0, max=2**N)
\end{verbatim}

Alternatively, a slice can be taken from an \class{intbv} object
as follows:

\begin{verbatim}
index = intbv(0)[N:]
\end{verbatim}

Such as slice returns a new \class{intbv} object, with minimum
value \code{0} , and maximum value \code{2**N}.


\subsection{Supported statements\label{conv-subset-statements}}

The following is a list of the statements that are supported by the
Verilog converter, possibly qualified with restrictions
or usage notes. Recall that
this list only applies to the design post elaboration: in practice,
this means it applies to the code of the generators that are the leaf
blocks in a design.

\begin{description}

\item[The \keyword{break} statement.]

\item[The \keyword{continue} statement.]

\item[The \keyword{def} statement.]

\item[The \keyword{for} statement.]
The only supported iteration scheme is iterating through sequences of
integers returned by built-in function \function{range} or \myhdl\
function \function{downrange}.  The optional \keyword{else} clause is
not supported.

\item[The \keyword{if} statement.]
\keyword{if}, \keyword{elif}, and \keyword{else} clauses
are fully supported.

\item[The \keyword{pass} statement.]

\item[The \keyword{print} statement.]
The only supported expression for printing is
a single literal string.
The string can be interpolated, but the format specifiers
are copied verbatim to the Verilog output.
Print to a file (with syntax \code{'>>'}) is not supported.

\item[The \keyword{raise} statement.]
This statement is mapped to Verilog statements
that end the simulation with an error message.

\item[The \keyword{return} statement.]

\item[The \keyword{yield} statement.] 
The yielded expression can be a signal, a signal edge
as specified by \myhdl\ functions \function{posedge}
or \function{negedge}, or a tuple of signals and
edge specifications.

\item[The \keyword{while} statement.]
The optional \keyword{else}
clause is not supported.

\end{description}

\section{Methodology notes\label{conv-meth}}

\subsection{Simulation\label{conv-meth-sim}}

In the Python philosophy, the run-time rules. The Python compiler
doesn't attempt to detect a lot of errors beyond syntax errors, which
given Python's ultra-dynamic nature would be an almost impossible task
anyway. To verify a Python program, one should run it, preferably
using unit testing to verify each feature.

The same philosophy should be used when converting a MyHDL description
to Verilog: make sure the simulation runs fine first. Although the
converter checks many things and attempts to issue clear error
messages, there is no guarantee that it does a meaningful job unless
the simulation runs fine.

\subsection{Conversion output verification\label{conv-meth-conv}}
It is always prudent to verify the converted Verilog output.
To make this task easier, the converter also generates a
test bench that makes it possible to simulate the Verilog
design using the Verilog co-simulation interface. This 
permits to verify the Verilog code with the same test
bench used for the \myhdl\ code. This is also how
the Verilog converter development is being verified.

\subsection{Assignment issues\label{conv-meth-assign}}

\subsubsection{Name assignment in Python\label{conv-meth-assign-python}}

Name assignment in Python is a different concept than in
many other languages. This point is very important for
effective modeling in Python, and even more so
for synthesizable \myhdl\ code. Therefore, the issues are
discussed here.

Consider the following name assignments:

\begin{verbatim}
a = 4
a = ``a string''
a = False
\end{verbatim}

In many languages, the meaning would be that an
existing variable \var{a} gets a number of different values.
In Python, such a concept of a variable doesn't exist. Instead,
assignment merely creates a new binding of a name to a
certain object, that replaces any previous binding.
So in the example, the name \var{a} is bound a 
number of different objects in sequence.

The Verilog converter has to investigate name
assignment and usage in \myhdl\ code, and to map
names to Verilog variables. To achieve that,
it tries to infer the type and possibly the
bit width of each expression that is assigned
to a name.

Multiple assignments to the same name can be supported if it can be
determined that a consistent type and bit width is being used in the
assignments. This can be done for boolean expressions, numeric
expressions, and enumeration type literals. In Verilog, the
corresponding name is mapped to a single bit \code{reg}, an
\code{integer} or a \code{reg} of the appropriate width, respectively.

In other cases, a single assignment should be used when an object is
created. Subsequent value changes are then achieved by modification of
an existing object.  This technique should be used for \class{Signal}
and \class{intbv} objects.

\subsubsection{Signal assignment\label{conv-meth-assign-signal}}

Signal assignment in \myhdl\ is implemented using attribute assignment
to attribute \code{next}.  Value changes are thus modeled by
modification of the existing object. The converter investigates the
\class{Signal} object to infer the type and bit width of the
corresponding Verilog variable.

\subsubsection{\class{intbv} objects\label{conv-meth-assign-intbv}}

Type \class{intbv} is likely to be the workhorse for synthesizable
modeling in \myhdl{}. An \class{intbv} instance behaves like a
(mutable) integer whose individual bits can be accessed and
modified. Also, it is possible to constrain its set of values. In
addition to error checking, this makes it possible to infer a bit
width, which is required for implementation.

In Verilog, an \class{intbv} instance will be mapped to a \code{reg}
with an appropriate width. As noted before, it is not possible
to modify its value using name assignment. In the following, we
will show how it can be done instead. Consider:

\begin{verbatim}
a = intbv(0)[8:]
\end{verbatim}

This is an \class{intbv} object with initial value \code{0} and
bit width 8. The change its value to \code{5}, we can use
slice assignment:

\begin{verbatim}
a[8:] = 5
\end{verbatim}

The same can be achieved by leaving the bit width unspecified, 
which has the meaning to change ``all'' bits:

\begin{verbatim}
a[:] = 5
\end{verbatim}

Often the new value will depend on the old one. For example,
to increment an \class{intbv} with the technique above:

\begin{verbatim}
a[:] = a + 1
\end{verbatim}

Python also provides \emph{augmented} assignment operators,
which can be used to implement in-place operations. These are supported
on \class{intbv} objects and by the converter, so that the increment
can also be done as follows:

\begin{verbatim}
a += 1
\end{verbatim}

\section{Converter usage\label{conv-usage}}

We will demonstrate the conversion process by showing some examples.

\subsection{A small design with a single generator\label{conv-usage-small}}

Consider the following MyHDL code for an incrementer module:

\begin{verbatim}
def inc(count, enable, clock, reset, n):
    """ Incrementer with enable.
    
    count -- output
    enable -- control input, increment when 1
    clock -- clock input
    reset -- asynchronous reset input
    n -- counter max value
    """
    def incProcess():
        while 1:
            yield posedge(clock), negedge(reset)
            if reset == ACTIVE_LOW:
                count.next = 0
            else:
                if enable:
                    count.next = (count + 1) % n
    return incProcess()
\end{verbatim}

In Verilog terminology, function \function{inc} corresponds to a
module, while generator function \function{incProcess}
roughly corresponds to an always block.

Normally, to simulate the design, we would "elaborate" an instance
as follows:

\begin{verbatim}
m = 8
n = 2 ** m
 
count = Signal(intbv(0)[m:])
enable = Signal(bool(0))
clock, reset = [Signal(bool()) for i in range(2)]

inc_inst = inc(count, enable, clock, reset, n=n)
\end{verbatim}

\code{inc_inst} is an elaborated design instance that can be simulated. To
convert it to Verilog, we change the last line as follows:

\begin{verbatim}
inc_inst = toVerilog(inc, count, enable, clock, reset, n=n)
\end{verbatim}

Again, this creates an instance that can be simulated, but as a side
effect, it also generates a Verilog module in file \file{inc_inst.v},
that is supposed to have identical behavior. The Verilog code
is as follows:

\begin{verbatim}
module inc_inst (
    count,
    enable,
    clock,
    reset
);

output [7:0] count;
reg [7:0] count;
input enable;
input clock;
input reset;


always @(posedge clock or negedge reset) begin: _MYHDL1_BLOCK
    if ((reset == 0)) begin
        count <= 0;
    end
    else begin
        if (enable) begin
            count <= ((count + 1) % 256);
        end
    end
end

endmodule
\end{verbatim}

You can see the module interface and the always block, as expected
from the MyHDL design. 

\subsection{Converting a generator directly\label{conv-usage-gen}}

It is also possible to convert a generator
directly. For example, consider the following generator function:

\begin{verbatim}
def bin2gray(B, G, width):
    """ Gray encoder.

    B -- input intbv signal, binary encoded
    G -- output intbv signal, gray encoded
    width -- bit width
    """
    Bext = intbv(0)[width+1:]
    while 1:
        yield B
        Bext[:] = B
        for i in range(width):
            G.next[i] = Bext[i+1] ^ Bext[i]
\end{verbatim}

As before, you can create an instance and convert to
Verilog as follows:

\begin{verbatim}
width = 8

B = Signal(intbv(0)[width:])
G = Signal(intbv(0)[width:])

bin2gray_inst = toVerilog(bin2gray, B, G, width)
 \end{verbatim}

The generate Verilog module is as follows:

\begin{verbatim}
module bin2gray_inst (
    B,
    G
);

input [7:0] B;
output [7:0] G;
reg [7:0] G;

always @(B) begin: _MYHDL1_BLOCK
    integer i;
    reg [9-1:0] Bext;
    Bext[9-1:0] = B;
    for (i=0; i<8; i=i+1) begin
        G[i] <= (Bext[(i + 1)] ^ Bext[i]);
    end
end

endmodule
\end{verbatim}

\subsection{A hierarchical design\label{conv-usage-hier}}
The hierarchy of convertible designs can be
arbitrarily deep.

For example, suppose we want to design an
incrementer with Gray code output. Using the
designs from previous sections, we can proceed
as follows:

\begin{verbatim}
def GrayInc(graycnt, enable, clock, reset, width):
    
    bincnt = Signal(intbv()[width:])
    
    INC_1 = inc(bincnt, enable, clock, reset, n=2**width)
    BIN2GRAY_1 = bin2gray(B=bincnt, G=graycnt, width=width)
    
    return INC_1, BIN2GRAY_1
\end{verbatim}

According to Gray code properties, only a single bit
will change in consecutive values. However, as the
\code{bin2gray} module is combinatorial, the output bits
may have transient glitches, which may not be desirable.
To solve this, let's create an additional level of
hierarchy an add an output register to the design.
(This will create an additional latency of a clock
cycle, which may not be acceptable, but we will
ignore that here.)

\begin{verbatim}
def GrayIncReg(graycnt, enable, clock, reset, width):
    
    graycnt_comb = Signal(intbv()[width:])
    
    GRAY_INC_1 = GrayInc(graycnt_comb, enable, clock, reset, width)
    
    def reg():
        while 1:
            yield posedge(clock)
            graycnt.next = graycnt_comb
    REG_1 = reg()
    
    return GRAY_INC_1, REG_1
\end{verbatim}

We can convert this hierarchical design as before:

\begin{verbatim}
width = 8
graycnt = Signal(intbv()[width:])
enable, clock, reset = [Signal(bool()) for i in range(3)]

GRAY_INC_REG_1 = toVerilog(GrayIncReg, graycnt, enable, clock, reset, width)
\end{verbatim}

The Verilog output module looks as follows:

\begin{verbatim}
module GRAY_INC_REG_1 (
    graycnt,
    enable,
    clock,
    reset
);

output [7:0] graycnt;
reg [7:0] graycnt;
input enable;
input clock;
input reset;

reg [7:0] graycnt_comb;
reg [7:0] _GRAY_INC_1_bincnt;

always @(posedge clock or negedge reset) begin: _MYHDL1_BLOCK
    if ((reset == 0)) begin
        _GRAY_INC_1_bincnt <= 0;
    end
    else begin
        if (enable) begin
            _GRAY_INC_1_bincnt <= ((_GRAY_INC_1_bincnt + 1) % 256);
        end
    end
end

always @(_GRAY_INC_1_bincnt) begin: _MYHDL4_BLOCK
    integer i;
    reg [9-1:0] Bext;
    Bext[9-1:0] = _GRAY_INC_1_bincnt;
    for (i=0; i<8; i=i+1) begin
        graycnt_comb[i] <= (Bext[(i + 1)] ^ Bext[i]);
    end
end

always @(posedge clock) begin: _MYHDL9_BLOCK
    graycnt <= graycnt_comb;
end

endmodule
\end{verbatim}

Note that the output is a flat ``net list of blocks'', and
that hierarchical signal names are generated as necessary.

\subsection{Optimizations for finite state machines\label{conv-usage-fsm}}
As often in hardware design, finite state machines deserve special attention.

In Verilog and VHDL, finite state machines are typically described
using case statements.  Python doesn't have a case statement, but the
converter recognizes particular if-then-else structures and maps them
to case statements. This optimization occurs when a variable whose
type is an enumerated type is sequentially tested against enumeration
items in an if-then-else structure. Also, the appropriate synthesis
pragmas for efficient synthesis are generated in the Verilog code.

As a further optimization, function \function{enum} was enhanced to support
alternative encoding schemes elegantly, using an additional parameter
'encoding'. For example:

\begin{verbatim}
t_State = enum('SEARCH', 'CONFIRM', 'SYNC', encoding="one_hot")
\end{verbatim}

The default encoding is \code{binary}; the other possibilities are \code{one_hot} and
\code{one_cold}. This parameter only affects the conversion output, not the
behavior of the type. Verilog case statements are optimized for an
efficient implementation according to the encoding. Note that in
contrast, a Verilog designer needs to make nontrivial code changes to
implement a different encoding scheme.

As an example, consider the following finite state machine, whose
state variable used the enumeration type defined above:

\begin{verbatim}
FRAME_SIZE = 8

def FramerCtrl(SOF, state, syncFlag, clk, reset_n):
    
    """ Framing control FSM.

    SOF -- start-of-frame output bit
    state -- FramerState output
    syncFlag -- sync pattern found indication input
    clk -- clock input
    reset_n -- active low reset
    
    """
    
    index = intbv(0, min=0, max=8) # position in frame
    while 1:
        yield posedge(clk), negedge(reset_n)
        if reset_n == ACTIVE_LOW:
            SOF.next = 0
            index[:] = 0
            state.next = t_State.SEARCH
        else:
            SOF.next = 0
            if state == t_State.SEARCH:
                index[:] = 0
                if syncFlag:
                    state.next = t_State.CONFIRM
            elif state == t_State.CONFIRM:
                if index == 0:
                    if syncFlag:
                        state.next = t_State.SYNC
                    else:
                        state.next = t_State.SEARCH
            elif state == t_State.SYNC:
                if index == 0:
                    if not syncFlag:
                        state.next = t_State.SEARCH
                SOF.next = (index == FRAME_SIZE-1)
            else:
                raise ValueError("Undefined state")
            index[:]= (index + 1) % FRAME_SIZE

\end{verbatim}

The conversion is done as before:

\begin{verbatim}
SOF = Signal(bool(0))
syncFlag = Signal(bool(0))
clk = Signal(bool(0))
reset_n = Signal(bool(1))
state = Signal(t_State.SEARCH)
framerctrl_inst = toVerilog(FramerCtrl, SOF, state, syncFlag, clk, reset_n)
\end{verbatim}

The Verilog output looks as follows:

\begin{verbatim}
module framerctrl_inst (
    SOF,
    state,
    syncFlag,
    clk,
    reset_n
);
output SOF;
reg SOF;
output [2:0] state;
reg [2:0] state;
input syncFlag;
input clk;
input reset_n;

always @(posedge clk or negedge reset_n) begin: _MYHDL1_BLOCK
    reg [3-1:0] index;
    if ((reset_n == 0)) begin
        SOF <= 0;
        index[3-1:0] = 0;
        state <= 3'b001;
    end
    else begin
        SOF <= 0;
        // synthesis parallel_case full_case
        casez (state)
            3'b??1: begin
                index[3-1:0] = 0;
                if (syncFlag) begin
                    state <= 3'b010;
                end
            end
            3'b?1?: begin
                if ((index == 0)) begin
                    if (syncFlag) begin
                        state <= 3'b100;
                    end
                    else begin
                        state <= 3'b001;
                    end
                end
            end
            3'b1??: begin
                if ((index == 0)) begin
                    if ((!syncFlag)) begin
                        state <= 3'b001;
                    end
                end
                SOF <= (index == (8 - 1));
            end
            default: begin
                $display("Verilog: ValueError(Undefined state)");
                $finish;
            end
        endcase
        index[3-1:0] = ((index + 1) % 8);
    end
end
endmodule
\end{verbatim}


\section{Known issues\label{conv-issues}}
\begin{description}

\item[Negative values of \class{intbv} instances are not supported.]
The \class{intbv} class is quite capable of representing negative
values. However, the \code{signed} type support in Verilog is
relatively recent and mapping to it may be tricky. In my judgment,
this is not the most urgent requirement, so
I decided to leave this for later.

\item[Verilog integers are 32 bit wide]
Usually, Verilog integers are 32 bit wide. In contrast, Python is
moving toward integers with undefined width. Python \class{int} 
and \class{long} variables are mapped to Verilog integers; so for values
larger than 32 bit this mapping is incorrect.

\item[Synthesis pragmas are specified as Verilog comments.] The recommended
way to specify synthesis pragmas in Verilog is through attribute
lists. However, my Verilog simulator (Icarus) doesn't support them
for \code{case} statements (to specify \code{parallel_case} and
\code{full_case} pragmas). Therefore, I still used the old
but deprecated method of synthesis pragmas in Verilog comments.

\item[Inconsistent place of the sensitivity list inferred from \code{always_comb}.]
The semantics of \code{always_comb}, both in Verilog and \myhdl{}, is to
have an implicit sensitivity list at the end of the code. However, this
may not be synthesizable. Therefore, the inferred sensitivity list is
put at the top of the corresponding \code{always} block.
This may cause inconsistent behavior at the start of the
simulation. The workaround is to create events at time 0.

\item[Non-blocking assignments to task arguments don't work.] 
I didn't get non-blocking (signal) assignments to task arguments to
work.  I don't know yet whether the issue is my own, a Verilog issue,
or an issue with my Verilog simulator Icarus. I'll need to check this
further.

\end{description}


\chapter{Reference manual}


\myhdl\ is implemented as a Python package called \code{myhdl}. This
chapter describes the objects that are exported by this package.

\section{The \class{Simulation} class}
\begin{classdesc}{Simulation}{arg \optional{, arg \moreargs}}
Class to construct a new simulation. Each argument is either be a
\myhdl\ generator, or a nested sequence of such generators. (A nested
sequence is defined as a sequence in which each item may itself be a
nested sequence.) See section~\ref{myhdl-generators} for the
definition of \myhdl\ generators and their interaction with a
\class{Simulation} object.
\end{classdesc}

A \class{Simulation} object has the following method:

\begin{methoddesc}[Simulation]{run}{\optional{duration}}
Run the simulation forever (by default) or for a specified duration.
\end{methoddesc}

\section{The \class{Signal} class}
\label{signal}
\begin{classdesc}{Signal}{val \optional{, delay}}
This class is used to construct a new signal and to initialize its
value to \var{val}. Optionally, a delay can be specified.
\end{classdesc}

A \class{Signal} object has the following attributes:

\begin{memberdesc}[Signal]{next}
Read-write attribute that represents the next value of the signal.
\end{memberdesc}

\begin{memberdesc}[Signal]{val}
Read-only attribute that represents the current value of the signal.

This attribute is always available to access the current value;
however in many practical case it will not be needed. Whenever there
is no ambiguity, the Signal object's current value is used
implicitly. In particular, all Python's standard numeric, bit-wise,
logical and comparison operators are implemented on a Signal object by
delegating to its current value. The exception is augmented
assignment. These operators are not implemented as they would break
the rule that the current value should be a read-only attribute. In
addition, when a Signal object is directly assigned to the \code{next}
attribute of another Signal object, its current value is assigned
instead.
\end{memberdesc}



\section{\myhdl\ generators and trigger objects}
\label{myhdl-generators}
\myhdl\ generators are standard Python generators with specialized
\keyword{yield} statements. In hardware description languages, the equivalent
statements are called \emph{sensitivity lists}. The general format
of \keyword{yield} statements in in \myhdl\ generators is:

\hspace{\leftmargin}\keyword{yield} \var{clause \optional{, clause ...}}

After a simulation object executes a \keyword{yield} statement, it
suspends execution of the generator. At the same time, each
\var{clause} is a \emph{trigger object} which defines the condition
upon which the generator should be resumed. However, per invocation of a
\keyword{yield} statement, the generator is resumed exactly once,
regardless of the number of clauses. This happens as soon as one
of the objects triggers; subsequent triggers are
neglected. (However, as a result of the resumption, it is possible
that the same \keyword{yield} statement is invoked again, and that a
subsequent trigger still triggers the generator.)

In this section, the trigger objects and their functionality will be
described. 

\begin{funcdesc}{posedge}{signal}
Return a trigger object that specifies that the generator should
resume on a rising edge on the signal. A rising edge means a change
from false to true.
\end{funcdesc}

\begin{funcdesc}{negedge}{signal}
Return a trigger object that specifies that the generator should
resume on a falling edge on the signal. A falling edge means a change
from true to false.
\end{funcdesc}

\begin{funcdesc}{delay}{t}
Return a trigger object that specifies that the generator should
resume after a delay \var{t}.
\end{funcdesc}

\begin{funcdesc}{join}{arg \optional{, arg \moreargs}}
Join a number of trigger objects together and return a joined
trigger object.  The effect is that the joined trigger object will
trigger when \emph{all} of its arguments have triggered.
\end{funcdesc}

In addition, some objects can directly function as trigger
objects. These are the objects of the following types:

\begin{datadesc}{Signal}
For the full description of the \class{Signal} class, see
section~\ref{signal}.

A signal is a trigger object. Whenever a signal changes value, the
generator is triggered.
\end{datadesc}

\begin{datadesc}{GeneratorType}
\myhdl\ generators can itself be used as trigger objects. 
This corresponds to spawning a new generator, while the original
generator waits for it to complete.  In other words, the original
generator is triggered when the spawned generator completes.
\end{datadesc}

In addition, as a special case, the Python \code{None} object can be
present in  a \code{yield} statement:

\begin{datadesc}{None}
This is the do-nothing trigger object. The generator immediately
resumes, as if no \code{yield} statement were present. This can be
useful if the \code{yield} statement also has generator clauses: those
generators are spawned, while the original generator resumes
immediately.

\end{datadesc}



\section{Miscellaneous objects}

The following objects can be convenient in \myhdl\ modeling.

\begin{excclassdesc}{StopSimulation}{}
Base exception that is caught by the \code{Simulation.run} method to
stop a simulation. Can be subclassed and raised in generator code.
\end{excclassdesc}

\begin{funcdesc}{now}{}
Return the current simulation time.
\end{funcdesc}

\begin{funcdesc}{downrange}{high \optional{, low=0}}
Generates a downward range list of integers. Modeled after the
standard \code{range} function, but works in the downward
direction. The returned interval is half-open, with the \var{high}
index not included. \var{low} is optional and defaults to zero.

This function is especially useful with the \class{intbv} class, that
also works with downward indexing.
\end{funcdesc}

\begin{funcdesc}{bin}{num \optional{, width}}
Return a representation as a bit string.  If \var{width} is provided,
and if it is larger than the width of the default representation, the
bit string is padded with the sign bit.

This function complements the standard Python conversion functions
\code{hex} and \code{oct}. A binary string representation is often
needed in hardware design.
\end{funcdesc}


\section{The \class{intbv} class}

\begin{classdesc}{intbv}{arg}
This class represents \class{int}-like objects with some additional
features that make it suitable for hardware design. The constructor
argument can be an \class{int}, a \class{long}, an \class{intbv} or a
bit string (a string with only '0's or '1's). For a bit string
argument, the value is calculated as \code{int(\var{bitstring}, 2)}. 
\end{classdesc}

Unlike \class{int} objects, \class{intbv} objects are mutable; this is
also the reason for their existence. Mutability is needed to support
assignment to indexes and slices, as is common in hardware design. For
the same reason, \class{intbv} is not a subclass from \class{int},
even though \class{int} provides most of the desired
functionality. (It is not possible to derive a mutable subtype from
an immutable base type.)

An \class{intbv} object supports the same comparison, numeric,
bitwise, logical, and conversion operations as \class{int} objects. See
\url{http://www.python.org/doc/current/lib/typesnumeric.html} for more
information on such operations. In all binary operations,
\class{intbv} objects can work together with \class{int} objects; in
those cases the return type is an \class{intbv} object.

In addition, \class{intbv} objects support indexing and slicing
operations:

\begin{tableiii}{clc}{code}{Operation}{Result}{Notes}
  \lineiii{\var{bv}[\var{i}]}
	  {item \var{i} of \var{bv}}
	  {(1)}
  \lineiii{\var{bv}[\var{i}] = \var{x}}  
	  {item \var{i} of \var{bv} is replaced by \var{x}} 
          {(1)}
  \lineiii{\var{bv}[\var{i}:\var{j}]} 
          {slice of \var{bv} from \var{i} downto \var{j}} 
          {(2)(3)}
  \lineiii{\var{bv}[\var{i}:\var{j}] = \var{t}} 
  	  {slice of \var{bv} from \var{i} downto \var{j} is replaced
          by \var{t}} 
          {(2)(4)}
\end{tableiii}

\begin{description}
\item[(1)] Indexing follows the most common hardware design
	  conventions: the lsb bit is the rightmost bit, and it has
	  index 0. This has the following desirable property: if the
	  \class{intbv} value is decomposed as a sum of powers of 2,
	  the bit with index \var{i} corresponds to the term
	  \code{2**i}.

\item[(2)] It follows from the indexing convention that slicing ranges
	  are downward, in contrast to standard Python. However, the
	  Python convention of half-open ranges is followed. In
	  accordance with standard Python, the high index is not
	  included. However, it is the \emph{leftmost} index in this
	  case. As in standard Python, this takes care of one-off
	  issues in many practical cases: in particular,
	  \code{bv[\var{i}:]} returns \var{i} bits;
	  \code{bv[\var{i}:\var{j}]} has \code{\var{i}-\var{j}}
	  bits. As \class{intbv} objects have no explicitly defined
	  bit width, the high index \var{j} has no default value and
	  cannot be omitted, while the low index \var{j} defaults to
	  \code{0}.

\item[(3)] The value returned from a slicing operation is always
	  positive; higher order bits are implicitly assumed to be
	  zero. The bit width is implicitly stored in the returned bit
	  width, so that the returned object can be used in
	  concatenations and as an iterator.

\item[(4)] In setting a slice, it is checked whether the slice is wide
	  enough to accept all significant bits of the value.
\end{description}

In addition, \class{intbv} objects support a concatenation method:

\begin{methoddesc}[intbv]{concat}{\optional{arg \moreargs}}
Concatenate the arguments to an \class{intbv} object. Naturally, the
concatenation arguments need to have a defined bit width. Therefore,
if they are \class{intbv} objects, they have to be the return values
of a slicing operation. Alternatively, they may be bit strings.

In contrast to all other arguments, the implicit \var{self} argument
doesn't need to have a defined bit with. This is due to the fact that
concatenation occurs at the lsb (rightmost) side.

It may be clearer to call this method as an unbound method with an
explicit first \class{intbv} argument.
\end{methoddesc}

In addition, an \class{intbv} object supports the iterator protocol. This
makes it possible to iterate over all its bits, from the high index to
index 0. This is only possible for \class{intbv} objects with a
defined bit width.













\documentclass{manual}
\usepackage{palatino}
\renewcommand{\ttdefault}{cmtt}
\renewcommand{\sfdefault}{cmss}
\newcommand{\myhdl}{\protect \mbox{MyHDL}}
\usepackage{graphicx}

\title{The \myhdl\ manual}

\input{boilerplate}

\makeindex

\begin{document}

\maketitle

\input{copyright}

\begin{abstract}

\noindent

The goal of the \myhdl{} project is to empower hardware designers with
the elegance and simplicity of the Python language.

\myhdl{} is a free, open-source (LGPL) package for using Python as a
hardware description and verification language. Python is a very high
level language, and hardware designers can use its full power to model
and simulate their designs. Moreover, \myhdl{} can convert a design to
Verilog. In combination with an external synthesis tool, it provides a
complete path from Python to a silicon implementation.

\emph{Modeling}


Python's power and clarity make \myhdl{} an ideal solution for high level
modeling. Python is famous for enabling elegant solutions to complex
modeling problems. Moreover, Python is outstanding for rapid
application development and experimentation.

The key idea behind \myhdl{} is the use of Python generators to model
hardware concurrency. Generators are best described as resumable
functions. In \myhdl{}, generators are used in a specific way so that
they become similar to always blocks in Verilog or processes in VHDL.

A hardware module is modeled as a function that returns any number of
generators. This approach makes it straightforward to support features
such as arbitrary hierarchy, named port association, arrays of
instances, and conditional instantiation.

Furthermore, \myhdl{} provides classes that implement traditional
hardware description concepts. It provides a signal class to support
communication between generators, a class to support bit oriented
operations, and a class for enumeration types.

\emph{Simulation and Verification}

The built-in simulator runs on top of the Python interpreter. It
supports waveform viewing by tracing signal changes in a VCD file.

With \myhdl{}, the Python unit test framework can be used on hardware
designs. Although unit testing is a popular modern software
verification technique, it is not yet common in the hardware design
world, making it one more area in which \myhdl{} innovates.

\myhdl{} can also be used as hardware verification language for VHDL and
Verilog designs, by co-simulation with traditional HDL simulators.

\emph{Conversion to Verilog}

The converter to Verilog works on an instantiated design that has been
fully elaborated. Consequently, the original design structure can be
arbitrarily complex.

The converter automates certain tasks that are tedious or hard in
Verilog directly. Notable features are the possibility to choose
between various FSM state encodings based on a single attribute, the
mapping of certain high-level objects to RAM and ROM descriptions, and
the automated handling of signed arithmetic issues.



\end{abstract}

\tableofcontents

\input{background.tex}
\input{intro.tex}
\input{modeling.tex}
\input{unittest.tex}
\input{cosimulation.tex}

\chapter{Conversion to Verilog\label{conv}}
\input{conversion.tex}

\input{reference.tex}

\input{MyHDL.ind}

\end{document}


\end{document}


\end{document}


\end{document}
