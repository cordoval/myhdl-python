\chapter{Co-simulation with Verilog and VHDL \label{cosim}}

\section{Introduction \label{cosim-intro}}

One of the most exciting possibilities of \myhdl\
is to use it as a hardware verification language (HVL).
A HVL is a language used to write test benches and
verification environments, and to control simulations.

Nowadays, it is generally acknowledged that HVLs should be equipped
with modern software techniques, such as object orientation. The
reason is that verification it the most complex and time-consuming
task of the design process: consequently every useful technique is
welcome. Moreover, test benches are not required to be
implementable. Therefore, unlike synthesizable code, there
are no constraints on creativity.

Technically, verification of a design implemented in
another language requires co-simulation. \myhdl\ is 
enabled for co-simulation with any HDL simulator that
has a procedural language interface (PLI). The \myhdl\
side is designed to be independent of a particular
simulator, On the other hand, for each HDL simulator a specific
PLI module will have to be written in C. Currently,
the \myhdl\ release contains a PLI module to interface
to the Icarus Verilog simulator. This interface will
be used in the examples.

\section{The HDL side \label{cosim-hdl}}

To introduce co-simulation, we will continue to use the Gray encoder
example from the previous chapters. Suppose that we want to
synthesize it and write it in Verilog for that purpose. Clearly we would
like to reuse our unit test verification environment. 

To start, let's recall how the Gray encoder in \myhdl{} looks like:

\begin{verbatim}
def bin2gray(B, G, width):
    """ Gray encoder.

    B -- input intbv signal, binary encoded
    G -- output intbv signal, gray encoded
    width -- bit width
    """
    while 1:
        yield B
        for i in range(width):
            G.next[i] = B[i+1] ^ B[i]
\end{verbatim}

To show the co-simulation flow, we don't need the Verilog
implementation yet, but only the interface.  Our Gray encoder in
Verilog would have the following interface:

\begin{verbatim}
module bin2gray(B, G);

   parameter width = 8;
   input [width-1:0]  B;     
   output [width-1:0] G;
   ....
\end{verbatim}

To write a test bench, one creates a new module that instantiates the
design under test (DUT).  The test bench declares nets and
regs (or signals in VHDL) that are attached to the DUT, and to
stimulus generators and response checkers. In an all-HDL flow, the
generators and checkers are written in the HDL itself, but we will
want to write them in \myhdl{}. To make the connection, we need to
declare which regs \& nets are driven and read by the \myhdl\
simulator. For our example, this is done as follows:

\begin{verbatim}
module dut_bin2gray;

   reg [`width-1:0] B;
   wire [`width-1:0] G;

   initial begin
      $from_myhdl(B);
      $to_myhdl(G);
   end

   bin2gray dut (.B(B), .G(G));
   defparam dut.width = `width;

endmodule
\end{verbatim}

The \code{\$from_myhdl} task call declares which regs are driven by
\myhdl{}, and the \code{\$to_myhdl} task call which regs \& nets are read
by it. These tasks take an arbitrary number of arguments.  They are
defined in a PLI module written in C and made available in a
simulator-dependent manner.  In Icarus Verilog, the tasks are defined
in a \code{myhdl.vpi} module that is compiled from C source code.

\section{The \myhdl\ side \label{cosim-myhdl}}

\myhdl\ supports co-simulation by a \code{Cosimulation} object. 
A \code{Cosimulation} object must know how to run a HDL simulation.
Therefore, the first argument to its constructor is a command string
to execute a simulation.

The way to generate and run an simulation executable is simulator
dependent.  For example, in Icarus Verilog, a simulation executable
for our example can be obtained obtained by running the
\code{iverilog} compiler as follows:

\begin{verbatim}
% iverilog -o bin2gray -Dwidth=4 bin2gray.v dut_bin2gray.v
\end{verbatim}

This generates a \code{bin2gray} executable for a parameter \code{width}
of 4, by compiling the contributing verilog files.

The simulation itself is run by the \code{vvp} command:

\begin{verbatim}
% vvp -m ./myhdl.vpi bin2gray
\end{verbatim}

This runs the \code{bin2gray} simulation, and specifies to use the
\code{myhdl.vpi} PLI module present in the current directory. (This is 
just a command line usage example; actually simulating with the
\code{myhdl.vpi} module is only meaningful from a
\code{Cosimulation} object.)

We can use a \code{Cosimulation} object to provide a HDL
version of a design to the \myhdl\ simulator. Instead of a generator
function, we write a function that returns a \code{Cosimulation}
object. For our example and the Icarus Verilog simulator, this is done
as follows:

\begin{verbatim}
import os

from myhdl import Cosimulation

cmd = "iverilog -o bin2gray -Dwidth=%s bin2gray.v dut_bin2gray.v"
      
def bin2gray(B, G, width):
    os.system(cmd % width)
    return Cosimulation("vvp -m ./myhdl.vpi bin2gray", B=B, G=G)
\end{verbatim}

After the executable command argument, the \code{Cosimulation}
constructor takes an arbitrary number of keyword arguments. Those
arguments make the link between \myhdl\ Signals and HDL nets, regs, or
signals, by named association. The keyword is the name of an argument
in a \code{\$to_myhdl} or \code{\$from_myhdl} call; the argument is
a \myhdl\ Signal.

With all this in place, we can now use the existing unit test
to verify the Verilog implementation. Note that we kept the
same name and parameters for the the \code{bin2gray} function:
all we need to do is to provide this alternative definition
to the existing unit test.

Let's try it on the Verilog design:

\begin{verbatim}
module bin2gray(B, G);

   parameter width = 8;
   input [width-1:0]  B;
   output [width-1:0] G;
   reg [width-1:0] G;
   integer i;
   wire [width:0] extB;

   assign extB = {1'b0, B}; // zero-extend input

   always @(extB) begin
      for (i=0; i < width; i=i+1)
        G[i] <= extB[i+1] ^ extB[i];
   end

endmodule
\end{verbatim}

When we run our unit test, we get:

\begin{verbatim}
% python test_bin2gray.py 
Check that only one bit changes in successive codewords ... ok
Check that all codewords occur exactly once ... ok
Check that the code is an original Gray code ... ok

----------------------------------------------------------------------
Ran 3 tests in 2.729s

OK
\end{verbatim}


\section{Restrictions \label{cosim-restr}}

In the ideal case, it should be possible to simulate
any HDL description seamlessly with \myhdl{}. Moreover
the communicating signals at each side should act
transparently as a single one, enabling fully race free
operation.

For various reasons, it may not be possible or desirable
to achieve full generality. As anyone that has developed
applications with the Verilog PLI can testify, the
restrictions in a particular simulator, and the 
differences over various simulators, can be quite 
frustrating. Moreover, full generality may require
a disproportionate amount of development work compared
to a slightly less general solution that may
be sufficient for the target application.

Consequently, I have tried to achieve a solution
which is simple enough so that one can reasonably 
expect that any PLI-enabled simulator can support it,
and that is relatively easy to verify and maintain.
At the same time, the solution is sufficiently general 
to cover the target application space.

The result is a compromise that places certain restrictions
on the HDL code. In this section, these restrictions 
are presented.

\subsection{Only passive HDL can be co-simulated \label{cosim-pass}}

The most important restriction of the \myhdl\ co-simulation solution is
that only ``passive'' HDL can be co-simulated.  This means that the HDL
code should not contain any statements with time delays. In other
words, the \myhdl\ simulator should be the master of time; in
particular, any clock signal should be generated at the \myhdl\ side.

At first this may seem like an important restriction, but if one
considers the target application for co-simulation, it probably
isn't. 

\myhdl\ supports co-simulation so that test benches for HDL
designs can be written in Python.  Let's consider the nature of the
target HDL designs. For high-level, behavioral models that are not
intended for implementation, it should come as no surprise that I
would recommend to write them in \myhdl\ directly; that is one of the
goals of the \myhdl\ effort. Likewise, gate level designs with
annotated timing are not the target application: static timing
analysis is a much better verification method for such designs.

Rather, the targeted HDL designs are naturally models that are
intended for implementation, most likely through synthesis. As time
delays are meaningless in synthesizable code, the restriction is
compatible with the target application.

\subsection{Race sensitivity issues \label{cosim-race}}

In a typical RTL code, some events cause other events to occur in the
same time step. For example, when a clock signal triggers some signals
may change in the same time step. For race-free operation, an HDL
must differentiate between such events within a time step. This is done
by the concept of ``delta'' cycles. In a fully general, race free
co-simulation, the co-simulators would communicate at the level of delta
cycles. However, in \myhdl\ co-simulation, this is not entirely the
case.

Delta cycles from the \myhdl\ simulator toward the HDL co-simulator are
preserved. However, in the opposite direction, they are not. The
signals changes are only returned to the \myhdl\ simulator after all delta
cycles have been performed in the HDL co-simulator.

What does this mean? Let's start with the good news. As explained in
the previous section, the concept behind \myhdl\ co-simulation implies
that clocks are generated at the \myhdl\ side.  \emph{When using a
\myhdl\ clock and its corresponding HDL signal directly as a clock,
co-simulation is race free.} In other words, the case
that most closely reflects the \myhdl\ co-simulation approach, is race free.

The situation is different when you want to use a signal driven by the
HDL (and the corresponding MyHDL signal) as a clock. 
Communication triggered by such a clock is not race free. The solution
is to treat such an interface as a chip interface instead of an RTL
interface.  For example, when data is triggered at positive clock
edges, it can safely be sampled at negative clock edges.
Alternatively, the \myhdl\ data signals can be declared with a delay
value, so that they are guaranteed to change after the clock
edge.


\section{Implementation notes \label{cosim-impl}}

\begin{quote}
\em
This section requires some knowledge of PLI terminology.
\end{quote}

Enabling a simulator for co-simulation requires a PLI module written
in C. In Verilog, the PLI is part of the ``standard''.  However,
different simulators implement different versions and portions of the
standard. Worse yet, the behavior of certain PLI callbacks is not
defined on some essential points.  As a result, one should plan to
write a specific PLI module for any simulator.

The present release contains a PLI module for the open source Icarus
simulator. I would like to add modules for other simulators in the
future, either from external contributions, or by getting access to
them myself.

This section documents the current approach and status of the PLI
module implementation in Icarus, and some reflections on future
implementations in other simulators.

\subsection{Icarus Verilog \label{cosim-icarus}}

\subsubsection{Delta cycle implementation \label{cosim-icarus-delta}}

To make co-simulation work, a specific type of PLI callback is
needed. The callback should be run when all pending events have been
processed, while allowing the creation of new events in the current
time step (e.g. by the \myhdl\ simulator).  In some Verilog
simulators, the \code{cbReadWriteSync} callback does exactly
that. However, in others, including Icarus, it does not. The
callback's behavior is not fully standardized; some simulators run the
callback before non-blocking assignment events have been processed.

Consequently, I had to look for a workaround. One half of the solution
is to use the \code{cbReadOnlySync} callback.  This callback runs
after all pending events have been processed.  However, it does not
permit to create new events in the current time step.  The second half
of the solution is to map \myhdl\ delta cycles onto real Verilog time
steps.  Note that fortunately I had some freedom here because of the
restriction that only passive HDL code can be co-simulated.

I chose to make the time granularity in the Verilog simulator a 1000
times finer than in the \myhdl{} simulator. For each \myhdl\ time
step, 1000 Verilog time steps are available for \myhdl\ delta
cycles. In practice, only a few delta cycles per time step should be
needed. Exceeding this limit almost certainly indicates a design error;
the limit is checked at run-time. The factor 1000 also makes it
easy to distinguish ``real'' time from delta cycle time when printing
out the Verilog time.

\subsubsection{Passive Verilog check \label{cosim-icarus-pass}}

As explained before, co-simulated Verilog should not contain delay
statements. Ideally, there should be a run-time check to flag
non-compliant code. However, there is currently no such check in the
Icarus module.

The check can be written using the \code{cbNextSimTime} VPI callback
in Verilog. However, Icarus 0.7 doesn't support this callback. In the
meantime, support for it has been added to the Icarus development
branch.  When Icarus 0.8 is released, a check will be added.

In the mean time, just don't do this. It may appear to ``work'' but it
really won't as events will be missed over the co-simulation
interface.

\subsubsection{Interrupted system calls \label{cosim-impl-syscalls}}

The PLI module uses \code{read} and \code{write} system calls to
communicate between the co-simulators. The implementation assumes that
these calls are restarted automatically by the operating system when
interrupted. This is apparently what happens on my Linux box and on
other operating systems I have used before.

I know how to handle non-restarted interrupted system calls, but I
cannot test the code! Also, I don't know whether this is still a
relevant issue with modern operating systems. So I left it
at that for the moment, and added assertions that should trigger
when this situation occurs.

Whenever an assertion fires in the PLI module, let me
know. The same holds for Python exceptions that you cannot
easily explain.

\subsection{Other Verilog simulators \label{cosim-impl-verilog}}

The Icarus module is written with VPI calls, which are provided by the
most recent generation of the Verilog PLI. Some simulators may only
support TF/ACC calls, requiring a complete redesign of the interface
module.

If the simulator supports VPI, the Icarus module should be reusable to
a large extent. However, it may be possible to improve on it.  The
workaround to support delta cycles described in
Section~\ref{cosim-icarus-delta} may not be necessary. In some
simulators, the \code{cbReadWriteSync} callback occurs after all
events (including non-blocking assignments) have been processed. In
that case, the functionality can be supported without a finer time
granularity in the Verilog simulator.

There are also Verilog standardization efforts underway to resolve the
ambiguity of the \code{cbReadWriteSync} callback. The solution will be
to introduce new, well defined callbacks. From reading some proposals,
I conclude that the \code{cbEndOfSimTime} callback would provide the
required functionality.

\subsection{VHDL \label{cosim-impl-vhdl}}

It would be great to have an interface to VHDL simulators such as the
Modelsim VHDL simulator. This will require a PLI module using the
PLI of the VHDL simulator. One feature which I would
like to keep if possible is the way to declare the communicating
signals.  In the Verilog solution, it is not necessary to define and
instantiate a special HDL module (entity). Rather, the participating
signals can be declared directly in the \code{\$to_myhdl} and
\code{\$from_myhdl} task calls.
