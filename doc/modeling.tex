\chapter{Modeling techniques}

\section{RTL modeling}

The present section describes how \myhdl\ supports RTL style modeling
as is typically used for synthesizable models in Verilog or VHDL.

\subsection{Combinatorial logic}

\subsubsection{Template}

Combinatorial logic is described with a generator function code template as
follows: 

\begin{verbatim}
def combinatorialLogic(<arguments>)
    while 1:
        yield <input signal arguments>
        <functional code>

\end{verbatim}

The overall code is wrapped in a \code{while 1} statement to keep the
generator alive. All input signals are clauses in the \code{yield}
statement, so that the generator resumes whenever one of the inputs
changes. 

\subsubsection{Example}

The following is an example of a combinatorial multiplexer:

\begin{verbatim}
def mux(z, a, b, sel):
    """ Multiplexer.
    
    z -- mux output
    a, b -- data inputs
    sel -- control input: select a if asserted, otherwise b
    """
    while 1:
        yield a, b, sel
        if sel == 1:
            z.next = a
        else:
            z.next = b
\end{verbatim}

To verify, let's simulate this logic with some random patterns. The
\code{random} module in Python's standard library comes in handy for
such purposes. The function \code{randrange(\var{n})} returns a random
natural integer smaller than \var{n}. It is used in the test bench
code to produce random input values:

\begin{verbatim}
from random import randrange

(z, a, b, sel) = [Signal(0) for i in range(4)]

MUX_1 = mux(z, a, b, sel)

def test():
    print "z a b sel"
    for i in range(8):
        a.next, b.next, sel.next = randrange(8), randrange(8), randrange(2)
        yield delay(10)
        print "%s %s %s %s" % (z, a, b, sel)
        
Simulation(MUX_1, test()).run() 
   
\end{verbatim}

Because of the randomness, the simulation output varies between runs
\footnote{It also possible to have a reproducible random output, by
explicitly providing a seed value. See the documentation of the
\code{random} module}. One particular run produced the following
output:

\begin{verbatim}
% python mux.py
z a b sel
6 6 1 1
7 7 1 1
7 3 7 0
1 2 1 0
7 7 5 1
4 7 4 0
4 0 4 0
3 3 5 1
StopSimulation: No more events
\end{verbatim}


\subsection{Sequential logic}

\subsubsection{Template}
Sequential RTL models are sensitive to a clock edge. In addition, they
may be sensitive to a reset signal. We will describe one of the most
common patterns: a template with a rising clock edge and an
asynchronous reset signal. Other templates are similar.

\begin{verbatim}
def sequentialLogic(<arguments>, clock, ..., reset, ...)
    while 1:
        yield posedge(clock), negedge(reset)
        if reset == <active level>:
            <reset code>
        else:
            <functional code>

\end{verbatim}


\subsubsection{Example}
The following code is a description of an incrementer with enable, and
an asynchronous power-up reset.

\begin{verbatim}
ACTIVE_LOW, INACTIVE_HIGH = 0, 1

def Inc(count, enable, clock, reset, n):
    """ Incrementer with enable.
    
    count -- output
    enable -- control input, increment when 1
    clock -- clock input
    reset -- asynchronous reset input
    n -- counter max value
    """
    while 1:
        yield posedge(clock), negedge(reset)
        if reset == ACTIVE_LOW:
            count.next = 0
        else:
            if enable:
                count.next = (count + 1) % n

\end{verbatim}

For the test bench, we will use an independent clock generator, stimulus
generator, and monitor. After applying enough stimulus patterns, we
can raise the \code{myhdl.StopSimulation} exception to stop the
simulation run. The test bench for a small incrementer and a small
number of patterns is a follows:

\begin{verbatim}
count, enable, clock, reset = [Signal(intbv(0)) for i in range(4)]

INC_1 = Inc(count, enable, clock, reset, n=4)

def clockGen():
    while 1:
        yield delay(10)
        clock.next = not clock

def stimulus():
    reset.next = ACTIVE_LOW
    yield negedge(clock)
    reset.next = INACTIVE_HIGH
    for i in range(12):
        enable.next = min(1, randrange(3))
        yield negedge(clock)
    raise StopSimulation

def monitor():
    print "enable  count"
    yield posedge(reset)
    while 1:
        yield posedge(clock)
        yield delay(1)
        print "   %s      %s" % (enable, count)
        
Simulation(clockGen(), stimulus(), monitor(), INC_1).run()

\end{verbatim}

The simulation produces the following output:
\begin{verbatim}
% python inc.py
enable  count
   0      0
   1      1
   0      1
   1      2
   1      3
   1      0
   0      0
   1      1
   0      1
   0      1
   0      1
   1      2
StopSimulation

\end{verbatim}


\section{High level modeling}

test
