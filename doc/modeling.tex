\chapter{Modeling techniques}

\section{RTL modeling}
The present section describes how \myhdl\ supports RTL style modeling
as is typically used for synthesizable models in Verilog or VHDL. This
section is included mainly for illustrative purposes, as this modeling
style is well known and understood.

\subsection{Combinatorial logic}

\subsubsection{Template}

Combinatorial logic is described with a generator function code template as
follows: 

\begin{verbatim}
def combinatorialLogic(<arguments>)
    while 1:
        yield <input signal arguments>
        <functional code>

\end{verbatim}

The overall code is wrapped in a \code{while 1} statement to keep the
generator alive. All input signals are clauses in the \code{yield}
statement, so that the generator resumes whenever one of the inputs
changes. 

\subsubsection{Example}

The following is an example of a combinatorial multiplexer:

\begin{verbatim}
def mux(z, a, b, sel):
    """ Multiplexer.
    
    z -- mux output
    a, b -- data inputs
    sel -- control input: select a if asserted, otherwise b
    """
    while 1:
        yield a, b, sel
        if sel == 1:
            z.next = a
        else:
            z.next = b

\end{verbatim}

To verify, let's simulate this logic with some random patterns. The
\code{random} module in Python's standard library comes in handy for
such purposes. The function \code{randrange(\var{n})} returns a random
natural integer smaller than \var{n}. It is used in the test bench
code to produce random input values:

\begin{verbatim}
from random import randrange

(z, a, b, sel) = [Signal(0) for i in range(4)]

MUX_1 = mux(z, a, b, sel)

def test():
    print "z a b sel"
    for i in range(8):
        a.next, b.next, sel.next = randrange(8), randrange(8), randrange(2)
        yield delay(10)
        print "%s %s %s %s" % (z, a, b, sel)
        
Simulation(MUX_1, test()).run() 
   
\end{verbatim}

Because of the randomness, the simulation output varies between runs
\footnote{It also possible to have a reproducible random output, by
explicitly providing a seed value. See the documentation of the
\code{random} module}. One particular run produced the following
output:

\begin{verbatim}
% python mux.py
z a b sel
6 6 1 1
7 7 1 1
7 3 7 0
1 2 1 0
7 7 5 1
4 7 4 0
4 0 4 0
3 3 5 1
StopSimulation: No more events
\end{verbatim}


\subsection{Sequential logic}

\subsubsection{Template}
Sequential RTL models are sensitive to a clock edge. In addition, they
may be sensitive to a reset signal. We will describe one of the most
common patterns: a template with a rising clock edge and an
asynchronous reset signal. Other templates are similar.

\begin{verbatim}
def sequentialLogic(<arguments>, clock, ..., reset, ...)
    while 1:
        yield posedge(clock), negedge(reset)
        if reset == <active level>:
            <reset code>
        else:
            <functional code>

\end{verbatim}


\subsubsection{Example}
The following code is a description of an incrementer with enable, and
an asynchronous power-up reset.

\begin{verbatim}
ACTIVE_LOW, INACTIVE_HIGH = 0, 1

def Inc(count, enable, clock, reset, n):
    """ Incrementer with enable.
    
    count -- output
    enable -- control input, increment when 1
    clock -- clock input
    reset -- asynchronous reset input
    n -- counter max value
    """
    while 1:
        yield posedge(clock), negedge(reset)
        if reset == ACTIVE_LOW:
            count.next = 0
        else:
            if enable:
                count.next = (count + 1) % n

\end{verbatim}

For the test bench, we will use an independent clock generator, stimulus
generator, and monitor. After applying enough stimulus patterns, we
can raise the \code{myhdl.StopSimulation} exception to stop the
simulation run. The test bench for a small incrementer and a small
number of patterns is a follows:

\begin{verbatim}
count, enable, clock, reset = [Signal(intbv(0)) for i in range(4)]

INC_1 = Inc(count, enable, clock, reset, n=4)

def clockGen():
    while 1:
        yield delay(10)
        clock.next = not clock

def stimulus():
    reset.next = ACTIVE_LOW
    yield negedge(clock)
    reset.next = INACTIVE_HIGH
    for i in range(12):
        enable.next = min(1, randrange(3))
        yield negedge(clock)
    raise StopSimulation

def monitor():
    print "enable  count"
    yield posedge(reset)
    while 1:
        yield posedge(clock)
        yield delay(1)
        print "   %s      %s" % (enable, count)
        
Simulation(clockGen(), stimulus(), monitor(), INC_1).run()

\end{verbatim}

The simulation produces the following output:
\begin{verbatim}
% python inc.py
enable  count
   0      0
   1      1
   0      1
   1      2
   1      3
   1      0
   0      0
   1      1
   0      1
   0      1
   0      1
   1      2
StopSimulation
\end{verbatim}

\section{High level modeling}

\begin{quote}
\em
This section on high level modeling should become an exciting part of
the manual, but it doesn't contain a lot of material just yet. One
reason is that I concentrated on the groundwork first. Moreover,
though I expect Python to offer some very powerful capabilities in
this domain, I'm just starting to experiment and learn myself. For
example, note that so far we haven't used classes (nor meta-classes
:-)) yet, even though Python has a very powerful object-oriented
model.
\end{quote}

\subsection{Modeling memories with built-in types}

Python has powerful built-in data types that can be useful to model
hardware memories. This can be merely a matter of putting an interface
around some data type operations.

For example, a \dfn{dictionary} comes in handy to model sparse memory
structures. (In other languages, this data type is called \dfn{
associative array}, or \dfn{hash table}.) A sparse memory is one in
which only a small part of the addresses is used in a particular
application or simulation. Instead of statically allocating the full
address space, which can be large, it is better to dynamically
allocate the needed storage space. This is exactly what a dictionary
provides. The following is an example of a sparse memory model:

\begin{verbatim}


def sparseMemory(dout, din, addr, we, en, clk):

    """ Sparse memory model based on a dictionary.

    Ports:
    dout -- data out
    din -- data in
    addr -- address bus
    we -- write enable: write if 1, read otherwise
    en -- interface enable: enabled if 1
    clk -- clock input
    
    """
    memory = {}
    while 1:
        yield posedge(clk)
        if not en:
            continue
        if we:
            memory[addr] = din.val
        else:
            dout.next = memory[addr]

\end{verbatim} 

Note how we use the \code{val} attribute of the \code{din} signal, as
we don't want to store the signal object itself, but its current
value. (In many cases, \myhdl\ can use a signal's value automatically
when there is no ambiguity: for example, this happens whenever a
signal is used in expressions. When in doubt, you can always use the
\code{val} attribute explicitly.)

As a second example, we will demonstrate how to use a list to model a
synchronous fifo:

\begin{verbatim}
def fifo(dout, din, re, we, empty, full, clk, maxFilling=sys.maxint):

    """ Synchronous fifo model based on a list.

    Ports:
    dout -- data out
    din -- data in
    re -- read enable
    we -- write enable
    empty -- empty indication flag
    full -- full indication flag
    clk -- clock input

    Optional parameter:
    maxFilling -- maximum fifo filling, "infinite" by default

    """
    memory = []
    while 1:
        yield posedge(clk)
        if we:
            memory.insert(0, din.val)
        if re:
            dout.next = memory.pop()
        empty.next = (len(memory) == 0)
        full.next = (len(memory) == maxFilling)

\end{verbatim}

Again, the model is merely a \myhdl\ interface around some operations
on a list: \function{insert()} to insert entries, \function{pop()} to
retrieve them, and \function{len()} to get the size of a Python
object.

\subsection{Modeling errors using exceptions}

In the previous section, we used Python data types for modeling. If
such a type is used inappropriately, Python's run time error system
will come into play. For example, if we access an address in the
\function{sparseMemory} model that was not initialized before, we will
get a traceback similar to the following (some lines omitted for
clarity):

\begin{verbatim}
Traceback (most recent call last):
...
  File "sparseMemory.py", line 30, in sparseMemory
    dout.next = memory[addr]
KeyError: 51

\end{verbatim}

Similarly, if the \code{fifo} is empty, and we attempt to read from
it, we get:

\begin{verbatim}
Traceback (most recent call last):
...
  File "fifo.py", line 34, in fifo
    dout.next = memory.pop()
IndexError: pop from empty list

\end{verbatim}

Instead of these low level errors, it may be preferable to define
errors at the functional level. In Python, this is typically done by
defining a custom \code{Error} exception at the module level, by
subclassing the standard \code{Exception} class. This exception is
then raised explicitly when an error condition occurs.

For example, we can change function \function{sparseMemory} as follows
(with the doc string is omitted for brevity):

\begin{verbatim}
class Error(Exception):
    pass

def sparseMemory(dout, din, addr, we, en, clk):
    memory = {}
    while 1:
        yield posedge(clk)
        if not en:
            continue
        if we:
            memory[addr] = din.val
        else:
            try:
                dout.next = memory[addr]
            except KeyError:
                raise Error, "Uninitialized address %s" % hex(addr)

\end{verbatim}

This works by catching the low level data type exception, and raising
the custom exception with an appropriate error message instead.  If
the \function{sparseMemory} function is defined in a module with the
same name, an access error is now reported as follows:

\begin{verbatim}
Traceback (most recent call last):
...
  File "sparseMemory.py", line 57, in sparseMemory
    raise Error, "Initialize address %s" % hex(addr)
sparseMemory.Error: Initializes address 0x33

\end{verbatim}

Likewise, the \function{fifo} function can be adapted as follows, to
report underflow and overflow errors:

\begin{verbatim}
class Error(Exception):
    pass

def fifo(dout, din, re, we, empty, full, clk, maxFilling=sys.maxint):
    memory = []
    while 1:
        yield posedge(clk)
        if we:
            memory.insert(0, din.val)
        if re:
            try:
                dout.next = memory.pop()
            except IndexError:
                raise Error, "Underflow - Read from empty fifo"
        empty.next = (len(memory) == 0)
        full.next = (len(memory) == maxFilling)
        if len(memory) > maxFilling:
            raise Error, "Overflow - Max filling %s exceeded" % maxFilling

\end{verbatim}

In this case, the underflow error is detected as before, by catching a
low level exception on the list data type. On the other hand, the
overflow error is detected by a regular check on the length of the
list.


\begin{quote}
\em
A lot to be added ...
\end{quote}
