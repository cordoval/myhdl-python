\documentclass{manual}
\usepackage{palatino}
\renewcommand{\ttdefault}{cmtt}
\renewcommand{\sfdefault}{cmss}
\newcommand{\myhdl}{\protect \mbox{MyHDL}}

\title{The \myhdl\ manual}

\author{Jan Decaluwe}
\authoraddress{
Email: \email{jan@jandecaluwe.com}
}

\date{December 19, 2005}	% XXX update before release!
\release{0.5b1}  		% software release, not documentation
\setreleaseinfo{}		% empty for final release
\setshortversion{0.5b1}		% major.minor only for software


\begin{document}

\maketitle

Copyright \copyright{} 2001 Python Software Foundation.
All rights reserved.

Copyright \copyright{} 2000 BeOpen.com.
All rights reserved.

Copyright \copyright{} 1995-2000 Corporation for National Research Initiatives.
All rights reserved.

Copyright \copyright{} 1991-1995 Stichting Mathematisch Centrum.
All rights reserved.

See the end of this document for complete license and permissions
information.


\begin{abstract}

\noindent

\myhdl\ is a Python package for using Python as a hardware description
language. Popular hardware description languages, like Verilog and
VHDL, are compiled languages. \myhdl\ with Python can be viewed as a
"scripting language" counterpart of such languages. However, Python is
more accurately described as a very high level language
(VHLL). \myhdl\ users have access to the amazing power and elegance of
Python for their modeling work.

The key idea behind \myhdl\ is to use Python generators to model the
concurrency required in hardware descriptions. As generators are a
recent Python feature, \myhdl\ requires Python 2.2.2 or higher.

\myhdl\ can be used to experiment with high level modeling, and with
verification techniques such as unit testing. The most important
practical application however, is to use it as a hardware verification
language by co-simulation with Verilog and VHDL.

The present release, \myhdl\ 0.2, enables \myhdl\ for
co-simulation. The \myhdl\ side is designed to work with any simulator
that has a PLI. For each simulator, an appropriate PLI module in C
needs to be provided. The release contains such a module for the
Icarus Verilog simulator.


\end{abstract}

\tableofcontents

\chapter{Background information}

\section{Prerequisites}

You need a basic understanding of Python to use \myhdl\.
If you don't know Python, you will take comfort in knowing
that it is probably one of the easiest programming languages to
learn \footnote{You must be bored by such claims, but in Python's
case it's true.}. Learning Python is also one of the better time
investments that engineering professionals can make \footnote{I am not
biased.}.

For beginners, \url{http://www.python.org/doc/current/tut/tut.html} is
probably the best choice for an on-line tutorial. For alternatives,
see \url{http://www.python.org/doc/Newbies.html}.

A working knowledge of a hardware description language such as Verilog
or VHDL is helpful. Chances are that you know one of those anyway, if
you are interested in \myhdl{}.

\section{A small tutorial on generators}

Generators are a recent feature in Python. They were introduced in
Python 2.2, which is the most recent stable version at the time of
this writing. Therefore, there isn't a lot of tutorial material
available yet. Because generators are the key concept in
\myhdl{}, I include a small tutorial here.

Consider the following nonsensical function:

\begin{verbatim}
def function():
    for i in range(5):
        return i

\end{verbatim}

You can see why it doesn't make a lot of sense. As soon as the first
loop iteration is entered, the function returns:

\begin{verbatim}
>>> function()
0
\end{verbatim}

Returning is fatal for the function call. Further loop iterations
never get a chance, and nothing is left over from the function call
when it returns.

To change the function into a generator function, we replace
\keyword{return} with \keyword{yield}:

\begin{verbatim}
def generator():
    for i in range(5):
        yield i

\end{verbatim}

Now we get:

\begin{verbatim}
>>> generator()
<generator object at 0x815d5a8>

\end{verbatim}

When a generator function is called, it returns a generator object. A
generator object supports the iterator protocol, which is an expensive
way of saying that you can let it generate subsequent values by
calling its \function{next()} method:

\begin{verbatim}
>>> g = generator()
>>> g.next()
0
>>> g.next()
1
>>> g.next()
2
>>> g.next()
3
>>> g.next()
4
>>> g.next()
Traceback (most recent call last):
  File "<stdin>", line 1, in ?
StopIteration

\end{verbatim}

Now we can generate the subsequent values from the for
loop on demand, until they are exhausted. What happens is that the
\keyword{yield} statement is like a
\keyword{return}, except that it is non-fatal: the generator remembers
its state and the point in the code when it yielded. A higher order
agent can decide when to get a further value by calling the
generator's \function{next()} method. We say that generators are
\dfn{resumable functions}.

If you are familiar with hardware description languages, this may
ring a bell. In hardware simulations, there is also a higher order
agent, the Simulator tool, that interacts with such resumable
functions; they are called processes in VHDL and always blocks in
Verilog. Like in those languages, Python generators provide an elegant
and efficient method to model concurrency, without having to resort to
some form of threading.

The use of generators to model concurrency is the first key concept in
\myhdl{}. The second key concept is a related one: in \myhdl{}, the
yielded values are used to define the condition upon which the generator
should be resumed. In other words, the \keyword{yield} statements work
as generalized sensitivity lists. If by now you are still interested,
read on to learn more!

If you want to know more about generators, consult the on-line Python
documentation, e.g. at \url{http://www.python.org/doc/2.2.2/whatsnew}. 

\begin{notice}[warning]
At the beginning of this section I said that generators were
introduced in Python 2.2. This is not entirely correct: in fact,
generators will only be enabled as a standard feature in Python 2.3.
However, a stable version of Python 2.3 has not been released yet at
the time of this writing. So, what to do?

Fortunately, Python lets you import features from its future releases
(provided that the future is not too distant). So, until you use
Python 2.3 or higher, you have to include the following line at the
start of all code that defines generator functions:

\begin{verbatim}
from __future__ import generators

\end{verbatim}

From Python 2.3 on, this line may still be in the code, though it
will not have an effect anymore.
\end{notice}


\chapter{Introduction to \myhdl\ }

\section{A basic \myhdl\ simulation}

We will introduce \myhdl\ with a classical \code{Hello World} style
example. Here are the contents of a \myhdl\ simulation script called
\file{Hello1.py}:

\begin{verbatim}
from myhdl import delay, now, Simulation

def sayHello():
    while 1:
        yield delay(10)
        print "%s Hello World!" % now()

gen = sayHello()
sim = Simulation(gen)
sim.run(30)

\end{verbatim}

When we run this script, we get the following output: 

\begin{verbatim}
% python Hello1.py
10 Hello World!
20 Hello World!
30 Hello World!
StopSimulation: Simulated for duration 30

\end{verbatim}

The first line of the script imports a
number of objects from the \code{myhdl} package. In good Python style, and
unlike most other languages, we can only use identifiers that are
\emph{literally} defined in the source file \footnote{I don't want to
explain the \samp{import *} syntax}.

Next, we define a generator function called
\code{sayHello}. This is a generator function (as opposed to
a classic Python function) because it contains a \code{yield}
statement (instead of \code{return} statement). In \myhdl\, a
\code{yield} statement has a similar purpose as a \code{wait}
statement in VHDL: the statement suspends execution of the function,
and its clauses specify when the function should resume. In this case,
there is a \code{delay} clause, that specifies the required delay.

To make sure that the generator runs ``forever'', we wrap its behavior
in a \code{while 1} loop. This is as standard Python idiom, and it is
the \myhdl\ equivalent to a Verilog \code{always} block or a
VHDL \code{process}.

In \myhdl\, the basic simulation objects are generators. Generators
are created by calling a generator function. For example, variable
\code{gen} refers to a generator. To simulate this generator, we pass
it as an argument to a code{Simulation} object constructor.  We then
run the simulation for the desired amount of time.


\section{Concurrent generators and signals}

In the previous section, we simulated a single generator. Of course,
real hardware descriptions are not like that: in fact, they are
typically massively concurrent. \myhdl\ supports this by allowing an
arbitrary number of concurrent generators. More specifically, a
\code{Simulation} constructor can take an arbitrary number of
arguments, each of which can be a generator or a nested list of
generators.

With concurrency comes the problem of determinism. Therefore, hardware
languages use special objects to support deterministic communication
between concurrent regions. \myhdl\ has as \code{Signal} object which
is roughly modelled after VHDL signals.

We will demonstrate these concepts by extending our first example. We
introduce a clock signal, driven by a second generator. The
\code{sayHello} generator function is modified to wait for a rising
edge (\code{posedge}) of the clock instead of a delay. The resulting
script is as follows:

\begin{verbatim}
from myhdl import Signal, delay, posedge, now, Simulation

clk = Signal(0)

def clkGen():
    while 1:
        yield delay(10)
        clk.next = 1
        yield delay(10)
        clk.next = 0

def sayHello():
    while 1:
        yield posedge(clk)
        print "%s Hello World!" % now()

sim = Simulation(clkGen(), sayHello())
sim.run(50)

\end{verbatim}

When we run this script, we get:

\begin{verbatim}
% python Hello2.py
10 Hello World!
30 Hello World!
50 Hello World!
StopSimulation: Simulated for duration 50

\end{verbatim}

The \code{clk} signal is constructed with an initial value
\code{0}. In the clock generator function \code{clkGen}, it is then
continuously toggled after a certain delay. In \myhdl{}, a the next
value of a signal is specified by assigning to its \code{next}
attribute. This is the \myhdl\ equivalent of VHDL signal assignments
and Verilog's non-blocking assignments.

The \code{sayHello} generator function shows a second form of a
\code{yield} statement: \samp{yield posedge(\var{aSignal})}. Again,
the generator will suspend execution at that point, but in this case
it specifies that it should resume when there is a rising edge on the
signal.

The \code{Simulation} constructor now takes two generator arguments
that run concurrently throughout the simulation.

\section{Parameters and instantiations}

So far, the generator function examples had no parameters. The signals
they operated on were defined in their enclosing scope. However, 
to make the code reusable we will want to pass arguments through a
parameter list. For example, we can change the clock generator
function as follows to make it more general and reusable:

\begin{verbatim}
def clkGen(clock, period=20):
    lowTime = int(period/2)
    highTime = period - lowTime
    while 1:
        yield delay(lowTime)
        clock.next = 1
        yield delay(highTime)
        clock.next = 0

\end{verbatim}

The clock signal is now a parameter of the function. Also, the clock
period is a parameter with a default value of \code{20}.

Similarly, the \code{sayHello} function can be made more general:

\begin{verbatim}
def sayHello(clock, to="World!"):
    while 1:
        yield posedge(clock)
        print "%s Hello %s" % (now(), to)

\end{verbatim}

XXX

Multiple generators can be created by multiple calls to a generator
function, possibly with different parameters. This is analogous to the
concept of \emph{instantiation} in hardware description
languages. \myhdl\ supports hierarchy and instantiations through
higher level functions that return multiple generators.

\begin{verbatim}
def talk():

    clk1 = Signal(0)
    clk2 = Signal(0)
    
    clkGen1 = clkGen(clk1)
    clkGen2 = clkGen(clock=clk2, period=19)
    sayHello1 = sayHello(clock=clk1)
    sayHello2 = sayHello(to="MyHDL", clock=clk2)

    return (clkGen1, clkGen2, sayHello1, sayHello2) 

sim = Simulation(talk())
sim.run(50)

\end{verbatim}

This produces the following output:

\begin{verbatim}
% python Hello3.py
9 Hello MyHDL
10 Hello World!
28 Hello MyHDL
30 Hello World!
47 Hello MyHDL
50 Hello World!
StopSimulation: Simulated for duration 50

\end{verbatim}

Like in standard Python, positional or named parameter association can
be used, or a mix of the two \footnote{All positional parameters have
to come before any named parameter.}. These styles are demonstrated in
the example. Named association is very useful if there are a lot of
parameters. 


\chapter{Modeling techniques}

\section{RTL modeling}
The present section describes how \myhdl\ supports RTL style modeling
as is typically used for synthesizable models in Verilog or VHDL.

In this domain, \myhdl\ doesn't offer advantages compared to
other solutions. However, as this modeling style is well-known,
this section may be useful for illustrative purposes.

\subsection{Combinatorial logic}

\subsubsection{Template}

Combinatorial logic is described with a generator function code template as
follows: 

\begin{verbatim}
def combinatorialLogic(<arguments>)
    while 1:
        yield <input signal arguments>
        <functional code>

\end{verbatim}

The overall code is wrapped in a \code{while 1} statement to keep the
generator alive. All input signals are clauses in the \code{yield}
statement, so that the generator resumes whenever one of the inputs
changes. 

\subsubsection{Example}

The following is an example of a combinatorial multiplexer:

\begin{verbatim}
def mux(z, a, b, sel):
    """ Multiplexer.
    
    z -- mux output
    a, b -- data inputs
    sel -- control input: select a if asserted, otherwise b
    """
    while 1:
        yield a, b, sel
        if sel == 1:
            z.next = a
        else:
            z.next = b
\end{verbatim}

To verify, let's simulate this logic with some random patterns. The
\code{random} module in Python's standard library comes in handy for
such purposes. The function \code{randrange(\var{n})} returns a random
natural integer smaller than \var{n}. It is used in the test bench
code to produce random input values:

\begin{verbatim}
from random import randrange

(z, a, b, sel) = [Signal(0) for i in range(4)]

MUX_1 = mux(z, a, b, sel)

def test():
    print "z a b sel"
    for i in range(8):
        a.next, b.next, sel.next = randrange(8), randrange(8), randrange(2)
        yield delay(10)
        print "%s %s %s %s" % (z, a, b, sel)
        
Simulation(MUX_1, test()).run() 
   
\end{verbatim}

Because of the randomness, the simulation output varies between runs
\footnote{It also possible to have a reproducible random output, by
explicitly providing a seed value. See the documentation of the
\code{random} module}. One particular run produced the following
output:

\begin{verbatim}
% python mux.py
z a b sel
6 6 1 1
7 7 1 1
7 3 7 0
1 2 1 0
7 7 5 1
4 7 4 0
4 0 4 0
3 3 5 1
StopSimulation: No more events
\end{verbatim}


\subsection{Sequential logic}

\subsubsection{Template}
Sequential RTL models are sensitive to a clock edge. In addition, they
may be sensitive to a reset signal. We will describe one of the most
common patterns: a template with a rising clock edge and an
asynchronous reset signal. Other templates are similar.

\begin{verbatim}
def sequentialLogic(<arguments>, clock, ..., reset, ...)
    while 1:
        yield posedge(clock), negedge(reset)
        if reset == <active level>:
            <reset code>
        else:
            <functional code>

\end{verbatim}


\subsubsection{Example}
The following code is a description of an incrementer with enable, and
an asynchronous power-up reset.

\begin{verbatim}
ACTIVE_LOW, INACTIVE_HIGH = 0, 1

def Inc(count, enable, clock, reset, n):
    """ Incrementer with enable.
    
    count -- output
    enable -- control input, increment when 1
    clock -- clock input
    reset -- asynchronous reset input
    n -- counter max value
    """
    while 1:
        yield posedge(clock), negedge(reset)
        if reset == ACTIVE_LOW:
            count.next = 0
        else:
            if enable:
                count.next = (count + 1) % n

\end{verbatim}

For the test bench, we will use an independent clock generator, stimulus
generator, and monitor. After applying enough stimulus patterns, we
can raise the \code{myhdl.StopSimulation} exception to stop the
simulation run. The test bench for a small incrementer and a small
number of patterns is a follows:

\begin{verbatim}
count, enable, clock, reset = [Signal(intbv(0)) for i in range(4)]

INC_1 = Inc(count, enable, clock, reset, n=4)

def clockGen():
    while 1:
        yield delay(10)
        clock.next = not clock

def stimulus():
    reset.next = ACTIVE_LOW
    yield negedge(clock)
    reset.next = INACTIVE_HIGH
    for i in range(12):
        enable.next = min(1, randrange(3))
        yield negedge(clock)
    raise StopSimulation

def monitor():
    print "enable  count"
    yield posedge(reset)
    while 1:
        yield posedge(clock)
        yield delay(1)
        print "   %s      %s" % (enable, count)
        
Simulation(clockGen(), stimulus(), monitor(), INC_1).run()

\end{verbatim}

The simulation produces the following output:
\begin{verbatim}
% python inc.py
enable  count
   0      0
   1      1
   0      1
   1      2
   1      3
   1      0
   0      0
   1      1
   0      1
   0      1
   0      1
   1      2
StopSimulation

\end{verbatim}


\section{High level modeling}

test

\chapter{Unit testing}

\section{Introduction}

Many aspects in the design flow of modern digital hardware design can
be viewed as a special kind of software development. From that
viewpoint, it is a natural question whether advances in software
design techniques can not also be applied to hardware design.

One software design approach that gets a lot of attention recently is
\emph{Extreme Programming} (XP). It is a fascinating set of techniques and
guidelines that often seems to go against the conventional wisdom. On
other occasions, XP just seems to emphasize the common sense, which
doesn't always coincide with common practice. For example, XP stresses
the importance of normal workweeks, if we are to have the
fresh mind needed for good software development.

It is not my intention nor qualification to present a tutorial on
Extreme Programming. Instead, in this section I will highlight one XP
concept which I think is very relevant to hardware design: the
importance and methodology of unit testing.

\section{The importance of unit tests}

Unit testing is one of the corner stones of Extreme Programming. Other
XP concepts, such as collective ownership of code and continuous
refinement, are only possible by having unit tests. Moreover, XP
emphasizes that writing unit tests should be automated, that they should
test everything in every class, and that they should run perfectly all
the time. 

I believe that these concepts apply directly to hardware design. In
addition, unit tests are a way to manage simulation time. For example,
a state machine that runs very slowly on infrequent events may be
impossible to verify at the system level, even on the fastest
simulator. On the other hand, it may be easy to verify it exhaustively
in a unit test, even on the slowest simulator.

It is clear that unit tests have compelling advantages. On the other
hand, if we need to test everything, we have to write
lots of unit tests. So it should be easy and pleasant
to create, manage and run them. Therefore, XP emphasizes the need for
a unit test framework that supports these tasks. In this chapter,
we will explore the use of the \code{unittest} module from
the standard Python library for creating unit tests for hardware
designs.


\section{Unit test development}

In this section, we will informally explore the application of unit
test techniques to hardware design. We will do so by a (small)
example: testing a binary to Gray encoder as introduced in
section~\ref{gray}. 

\subsection{Defining the requirements}

We start by defining the requirements. For a Gray encoder, we want to
the output to comply with Gray code characteristics. Let's define a
\dfn{code} as a list of \dfn{codewords}, where a codeword is a bit
string. A code of order \code{n} has \code{2**n} codewords.

A well-known characteristic is the one that Gray codes are all about:

\newtheorem{reqGray}{Requirement}
\begin{reqGray} 
Consecutive codewords in a Gray code should differ in a single bit.
\end{reqGray}

Is this sufficient? Not quite: suppose for example that an
implementation returns the lsb of each binary input. This would comply
with the requirement, but is obviously not what we want. Also, we don't
want the bit width of Gray codewords to exceed the bit width of the
binary codewords.

\begin{reqGray} 
Each codeword in a Gray code of order n must occur exactly once in the
binary code of the same order.
\end{reqGray}

With the requirements written down we can proceed.

\subsection{Writing the test first}

A fascinating guideline in the XP world is to write the unit test
first. That is, before implementing something, first write the test
that will verify it. This seems to go against our natural inclination,
and certainly against common practices. Many engineers like to
implement first and think about verification afterwards.

But if you think about it, it makes a lot of sense to deal with
verification first. Verification is about the requirements only --- so
your thoughts are not yet cluttered with implementation details. The
unit tests are an executable description of the requirements, so they
will be better understood and it will be very clear what needs to be
done. Consequently, the implementation should go smoother. Perhaps
most importantly, the test is available when you are done
implementing, and can be run anytime by anybody to verify changes.

Python has a standard \code{unittest} module that facilitates writing,
managing and running unit tests. With \code{unittest}, a test case is 
written by creating a class that inherits from
\code{unittest.TestCase}. Individual tests are created by methods of
that class: all method names that start with \code{test} are
considered to be tests of the test case.

We will define a test case for the Gray code properties, and then
write a test for each of the requirements. The outline of the test
case class is as follows:

\begin{verbatim}
from unittest import TestCase

class TestGrayCodeProperties(TestCase):

    def testSingleBitChange(self):
     """ Check that only one bit changes in successive codewords """
     ....


    def testUniqueCodeWords(self):
        """ Check that all codewords occur exactly once """
    ....
\end{verbatim}

Each method will be a small test bench that tests a single
requirement. To write the tests, we don't need an implementation of
the Gray encoder, but we do need the interface of the design. We can
specify this by a dummy implementation, as follows:

\begin{verbatim}
def bin2gray(B, G, width):
    ### NOT IMPLEMENTED YET! ###
    yield None
\end{verbatim}

For the first requirement, we will write a test bench that applies all
consecutive input numbers, and compares the current output with the
previous one for each input. Then we check that the difference is a
single bit. We will test all Gray codes up to a certain order
\code{MAX_WIDTH}.

\begin{verbatim}
    def testSingleBitChange(self):
        """ Check that only one bit changes in successive codewords """
        
        def test(B, G, width):
            B.next = intbv(0)
            yield delay(10)
            for i in range(1, 2**width):
                G_Z.next = G
                B.next = intbv(i)
                yield delay(10)
                diffcode = bin(G ^ G_Z)
                self.assertEqual(diffcode.count('1'), 1)
        
        for width in range(1, MAX_WIDTH):
            B = Signal(intbv(-1))
            G = Signal(intbv(0))
            G_Z = Signal(intbv(0))
            dut = bin2gray(B, G, width)
            check = test(B, G, width)
            sim = Simulation(dut, check)
            sim.run(quiet=1)
\end{verbatim}

Note how the actual check is performed by a \code{self.assertEqual}
method, defined by the \code{unittest.TestCase} class.

Similarly, we write a test bench for the second requirement. Again, we
simulate all numbers, and put the result in a list. The requirement
implies that if we sort the result list, we should get a range of
numbers:

\begin{verbatim}
    def testUniqueCodeWords(self):
        """ Check that all codewords occur exactly once """

        def test(B, G, width):
            actual = []
            for i in range(2**width):
                B.next = intbv(i)
                yield delay(10)
                actual.append(int(G))
            actual.sort()
            expected = range(2**width)
            self.assertEqual(actual, expected)
       
        for width in range(1, MAX_WIDTH):
            B = Signal(intbv(-1))
            G = Signal(intbv(0))
            dut = bin2gray(B, G, width)
            check = test(B, G, width)
            sim = Simulation(dut, check)
            sim.run(quiet=1)
\end{verbatim}


\subsection{Test-driven implementation}

With the test written, we begin with the implementation. For
illustration purposes, we will intentionally write some incorrect
implementations to see how the test behaves.

The easiest way to run tests defined with the \code{unittest}
framework, is to put a call to its \code{main} method at the end of
the test module:

\begin{verbatim}
unittest.main()
\end{verbatim}

Let's run the test using the dummy Gray encoder shown earlier:

\begin{verbatim}
% python test_gray.py -v
Check that only one bit changes in successive codewords ... FAIL
Check that all codewords occur exactly once ... FAIL
<trace backs not shown>
\end{verbatim}

As expected, this fails completely. Let us try an incorrect
implementation, that puts the lsb of in the input on the output:

\begin{verbatim}
def bin2gray(B, G, width):
    ### INCORRECT - DEMO PURPOSE ONLY! ###
    while 1:
        yield B
        G.next = B[0]
\end{verbatim}


Running the test produces:

\begin{verbatim}
% python test_gray.py -v
Check that only one bit changes in successive codewords ... ok
Check that all codewords occur exactly once ... FAIL

======================================================================
FAIL: Check that all codewords occur exactly once
----------------------------------------------------------------------
Traceback (most recent call last):
  File "test_gray.py", line 109, in testUniqueCodeWords
    sim.run(quiet=1)
...
  File "test_gray.py", line 104, in test
    self.assertEqual(actual, expected)
  File "/usr/local/lib/python2.2/unittest.py", line 286, in failUnlessEqual
    raise self.failureException, \
AssertionError: [0, 0, 1, 1] != [0, 1, 2, 3]

----------------------------------------------------------------------
Ran 2 tests in 0.785s
\end{verbatim}

Now the test passes the first requirement, as expected, but fails the
second one. After the test feedback, a full traceback is shown that
can help to debug the test output.

Finally, if we use the correct implementation as in
section~\ref{gray}, the output is:

\begin{verbatim}
% python test_gray.py -v
Check that only one bit changes in successive codewords ... ok
Check that all codewords occur exactly once ... ok

----------------------------------------------------------------------
Ran 2 tests in 6.364s

OK
\end{verbatim}



\subsection{Changing requirements}

In the previous section, we concentrated on the general requirements
of a Gray code. It is possible to specify these without specifying the
actual code. It is easy to see that there are several codes
that satisfy these requirements. In good XP style, we only tested
the requirements and nothing more.

It may be that more control is needed. For example, the requirement
may be for a particular code, instead of compliance with general
properties. As an illustration, we will show how to test for
\emph{the} original Gray code, which is one specific instance that
satisfies the requirements of the previous section. In this particular
case, this test will actually be easier than the previous one.

We denote the original Gray code of order \code{n} as \code{Ln}. Some
examples: 

\begin{verbatim}
L1 = ['0', '1']
L2 = ['00', '01', '11', '10']
L3 = ['000', '001', '011', '010', '110', '111', '101', 100']
\end{verbatim}

It is possible to specify these codes by a recursive algorithm, as
follows:

\begin{enumerate}
\item L1 = ['0', '1']
\item Ln+1 can be obtained from Ln as follows. Create a new code Ln0 by
prefixing all codewords of Ln with '0'. Create another new code Ln1 by
prefixing all codewords of Ln with '1', and reversing their
order. Ln+1 is the concatenation of Ln0 and Ln1.
\end{enumerate}

Python is well-known  for its elegant algorithmic
descriptions, and this is a good example. We can write the algorithm
in Python as follows:

\begin{verbatim}
def nextLn(Ln):
    """ Return Gray code Ln+1, given Ln. """
    Ln0 = ['0' + codeword for codeword in Ln]
    Ln1 = ['1' + codeword for codeword in Ln]
    Ln1.reverse()
    return Ln0 + Ln1
\end{verbatim}

The code \samp{['0' + codeword for ...]} is called a \dfn{list
comprehension}. It is a concise way to describe lists built by short
computations in a for loop.

The requirement is now that the output code matches the
expected code Ln. We use the \code{nextLn} function to compute the
expected result. The new test case code is as follows:

\begin{verbatim}
class TestOriginalGrayCode(TestCase):

    def testOriginalGrayCode(self):
        """ Check that the code is an original Gray code """

        Rn = []
        
        def stimulus(B, G, n):
            for i in range(2**n):
                B.next = intbv(i)
                yield delay(10)
                Rn.append(bin(G, width=n))
        
        Ln = ['0', '1'] # n == 1
        for n in range(2, MAX_WIDTH):
            Ln = nextLn(Ln)
            del Rn[:]
            B = Signal(intbv(-1))
            G = Signal(intbv(0))
            dut = bin2gray(B, G, n)
            stim = stimulus(B, G, n)
            sim = Simulation(dut, stim)
            sim.run(quiet=1)
            self.assertEqual(Ln, Rn)
\end{verbatim}

As it happens, our implementation is apparently an original Gray code:

\begin{verbatim}
% python test_gray.py -v TestOriginalGrayCode
Check that the code is an original Gray code ... ok

----------------------------------------------------------------------
Ran 1 tests in 3.091s

OK
\end{verbatim}


 


\chapter{MyHDL as a hardware verification language}

\section{Introduction}

One of the most exciting possibilities of \myhdl\
is to use it as a hardware verification language (HVL).
A HVL is a language used to write test benches and
verification environments, and to control simulations.

Nowadays, it is generally acknowledged that HVLs should be equipped
with modern software techniques, such as object orientation. The
reason is that verification it the most complex and time-consuming
task of the design process: consequently every useful technique is
welcome. Moreover, test benches are not required to be
implementable. Therefore, unlike synthesizable code, there
are no constraints on creativity.

Technically, verification of a design implemented in
another language requires cosimulation. \myhdl\ is 
enabled for cosimulation with any HDL simulator that
has a procedural language interface (PLI). The \myhdl\
side is designed to be independent of a particular
simulator, On the other hand, for each HDL simulator a specific
PLI module will have to be written in C. Currently,
the \myhdl\ release contains a PLI module to interface
to the Icarus Verilog simulator. This interface will
be used in the examples.

\section{The HDL side}

To introduce cosimulation, we will continue to use the Gray encoder
example from the previous chapters. Suppose that we want to
synthesize it and write it in Verilog for that purpose. Clearly we would
like to reuse our unit test verification environment. This is exactly
what \myhdl\ offers.

To start, let's recall how the Gray encoder in \myhdl{} looks like:

\begin{verbatim}
def bin2gray(B, G, width):
    """ Gray encoder.

    B -- input intbv signal, binary encoded
    G -- output intbv signal, gray encoded
    width -- bit width
    """
    while 1:
        yield B
        for i in range(width):
            G.next[i] = B[i+1] ^ B[i]

\end{verbatim}

To show the cosimulation flow, we don't need the Verilog
implementation yet, but only the interface.  Our Gray encoder in
Verilog would have the following interface:

\begin{verbatim}
module bin2gray(B, G);

   parameter width = 8;
   input [width-1:0]  B;     
   output [width-1:0] G;
   ....

\end{verbatim}

To write a test bench, one creates a new module that instantiates the
design under test (DUT).  The test bench declares nets and
regs (or signals in VHDL) that are attached to the DUT, and to
stimulus generators and response checkers. In an all-HDL flow, the
generators and checkers are written in the HDL itself, but we will
want to write them in \myhdl{}. To make the connection, we need to
declare which regs \& nets are driven and read by the \myhdl\
simulator. For our example, this is done as follows:

\begin{verbatim}
module dut_bin2gray;

   reg [`width-1:0] B;
   wire [`width-1:0] G;

   initial begin
      $from_myhdl(B);
      $to_myhdl(G);
   end

   bin2gray dut (.B(B), .G(G));
   defparam dut.width = `width;

endmodule

\end{verbatim}

The \code{\$from_myhdl} task call declares which regs are driven by
\myhdl{}, and the \code{\$to_myhdl} task call which regs \& nets are read
by it. These tasks take an arbitrary number of arguments.  They
are defined in a PLI module written in C. They are made available to
the simulation in a simulator-dependent manner.  In Icarus Verilog,
the tasks are defined in a \code{myhdl.vpi} module that is compiled
from C source code.

\section{The \myhdl\ side}

\myhdl\ supports cosimulation by a \code{Cosimulation} object. 
A \code{Cosimulation} object must know how to run a HDL cosimulation.
Therefore, the first argument to its constructor is a command string
to execute a simulation. The way to generate and run an
simulation executable is simulator dependent.
For example, in Icarus Verilog, a simulation executable for our
example can be obtained obtained by running the \code{iverilog}
compiler as follows:

\begin{verbatim}
% iverilog -o bin2gray -Dwidth=4 bin2gray.v dut_bin2gray.v

\end{verbatim}

This generates a \code{bin2gray} executable for a parameter \code{width}
of 4, by compiling the contributing verilog files.

The simulation itself is run by the \code{vvp} command:

\begin{verbatim}
% vvp -m ./myhdl.vpi bin2gray

\end{verbatim}

This runs the \code{bin2gray} simulation, and specifies to use the
\code{myhdl.vpi} PLI module present in the current directory. (This is 
just a command line usage example; actually simulating with the
\code{myhdl.vpi} module is only meaningful from a
\code{Cosimulation} object.)

We can use a \code{Cosimulation} object to provide a HDL cosimulation
version of a design to the \myhdl\ simulator. Instead of a generator
function, we write a function that returns a \code{Cosimulation}
object. For our example and the Icarus Verilog simulator, this is done
as follows:

\begin{verbatim}
import os

from myhdl import Cosimulation

cmd = "iverilog -o bin2gray -Dwidth=%s bin2gray.v dut_bin2gray.v"
      
def bin2gray(B, G, width):
    os.system(cmd % width)
    return Cosimulation("vvp -m ./myhdl.vpi bin2gray", B=B, G=G)

\end{verbatim}

After the executable command argument, the \code{Cosimulation}
constructor takes an arbitrary number of keyword arguments. Those
arguments make the link between \myhdl\ Signals and HDL nets, regs, or
signals, by named association. The keyword is the name of the argument
in a \code{\$to_myhdl} or \code{\$from_myhdl} call; the argument is
the \myhdl\ Signal.

With all this in place, we can now use the existing unit test
to verify the Verilog implementation. Note that we kept the
same name and parameters for the the \code{bin2gray} function:
all we need to do is to provide this alternative definition
to the existing unit test.

Let's quickly try it just to be sure:

\begin{verbatim}
module bin2gray(B, G);

   parameter width = 8;
   input [width-1:0]  B;
   output [width-1:0] G;
   reg [width-1:0] G;
   integer i;

   always @(B) begin
      for (i=0; i < width-1; i=i+1)
        G[i] <= B[i+1] ^ B[i];
   end

endmodule

\end{verbatim}

If we run our unit test we get:

\begin{verbatim}

% python test_bin2gray.py   
Check that only one bit changes in successive codewords ... ERROR
Check that all codewords occur exactly once ... FAIL
Check that the code is an original Gray code ... ERROR
...

\end{verbatim}

Oops! It seems we still have a bug! Oh yes, but of course, 
we need to zero-extend the input to get the msb output bit
correctly:

\begin{verbatim}
module bin2gray(B, G);

   parameter width = 8;
   input [width-1:0]  B;
   output [width-1:0] G;
   reg [width-1:0] G;
   integer i;
   wire [width:0] extB;

   assign extB = {1'b0, B};

   always @(extB) begin
      for (i=0; i < width; i=i+1)
        G[i] <= extB[i+1] ^ extB[i];
   end

endmodule

\end{verbatim}

And now:

\begin{verbatim}
% python test_bin2gray.py 
Check that only one bit changes in successive codewords ... ok
Check that all codewords occur exactly once ... ok
Check that the code is an original Gray code ... ok

----------------------------------------------------------------------
Ran 3 tests in 2.729s

OK

\end{verbatim}


\section{Restrictions}

In the ideal case, it should be possible to simulate
any HDL description seamlessly with \myhdl{}. Moreover
the communicating signals at each side should act
transparently as a single one, enabling fully race free
operation.

For various reasons, it may not be possible or desirable
to achieve full generality. As anyone that has developed
applications with the Verilog PLI can testify, the
restrictions in a particular simulator, and the 
differences over various simulators, can be quite 
frustrating. Moreover, full generality may require
a disproportionate amount of development work compared
to a slightly less general solution that may
be sufficient for the target application.

Consequently, I have tried to achieve a solution
which is simple enough so that one can reasonably 
expect that any PLI-enabled simulator can support it,
and that is relatively easy to verify and maintain.
At the same time, the solution is sufficiently general 
to cover the target application space.

The result is a compromise that places certain restrictions
on the HDL code. In this section, these restrictions 
are presented.

\subsection{Only passive HDL can be cosimulated}

The most important restriction of the \myhdl\ cosimulation solution is
that only ``passive'' HDL can be cosimulated.  This means that the HDL
code should not contain any statements with time delays. In other
words, the \myhdl\ simulator should be the master of time; in
particular, any clock signal should be generated at the \myhdl\ side.

At first this may seem like an important restriction, but if one
considers the target application for cosimulation, it probably
isn't. 

\myhdl\ support cosimulations so that test benches for HDL
designs can be written in Python.
Let's consider the nature of the target HDL designs. For high-level,
behavioral models that are not intended for implementation, it should
come as no surprise that I would recommend to write them in \myhdl\
directly; that is exactly the target of the \myhdl\ effort. Likewise,
gate level designs with annotated timing are not the target
application: static timing analysis is a much better verification
method for such designs.

Rather, the targeted HDL designs are naturally models that are
intended for implementation.  Most likely, this will be through
synthesis. As time delays are meaningless in synthesizable code, the
restriction is compatible with the target application.

\subsection{Race sensitivity issues}

In a typical TTL code, some events cause other events to occur in the
same time step. For example, when a clock signal triggers some signals
may change in the same time step. For race-free operation, an HDL
must differentiate between such events within a time step. This is done
by the concept of ``delta'' cycles. In a fully general, race free
cosimulation, the cosimulators would communicate at the level of delta
cycles. However, in \myhdl\ cosimulation, this is not entirely the
case.

Delta cycles from the \myhdl\ simulator toward the HDL cosimulator are
preserved. However, in the opposite direction, they are not. The
signals changes are only returned to the \myhdl\ simulator after all delta
cycles have been performed in the HDL cosimulator.

What does this mean? Let's start with the good news. As explained in
the previous section, the logic of the \myhdl\ cosimulation implies
that clocks are generated at the \myhdl\ side.  \emph{When using a
\myhdl\ clock and its corresponding HDL signal directly as a clock,
cosimulation operation is race free.} In other words, the case
that most closely reflects the \myhdl\ cosimulation approach, is race free.

The situation is different when you want to use a signal driven by the
HDL (and the corresponding MyHDL signal) as a clock. 
Communication triggered by such a clock is not race free. The solution
is to treat such an interface as a chip interface instead of an RTL
interface.  For example, when data is triggered at positive clock
edges, it can safely be sampled at negative clock edges.
Alternatively, the \myhdl\ data signals can be declared with a delay
value, so that they are guaranteed to change after the clock
edge.


\section{Implementation notes}

\begin{quote}
\em
This section requires some knowledge of PLI terminology.
\end{quote}

Enabling a simulator for cosimulation requires a PLI module
written in C. In Verilog, the PLI is part of the ``standard''.
However, different simulators implement different versions 
and portions of the standard. Worse yet, the behavior of
certain PLI callbacks is not defined on some essential points. 
As a result, one should plan to write a specific PLI module
for any simulator.

The present release contains a PLI module for the 
open source Icarus simulator. I would like to add
modules for any popular simulator in the future,
either from external contributions, or by getting
access to them myself. The same holds for VHDL
simulators: it would be great to have an interface
to the Modelsim VHDL simulator.

This section documents
the current approach and status of the PLI module
implementation in Icarus, and some reflections
on future implementations in other simulators.

\subsection{Icarus Verilog}

To make cosimulation work, a specific type of PLI callback is
needed. The callback should be run when all pending events have been
processed, while allowing the creation of new events in the current
time step (e.g. by the \myhdl\ simulator).  In some Verilog simulators,
the \code{cbReadWriteSync} callback does exactly that. However,
in others, including Icarus, it does not. The callback's behavior is
not fully standardized; some simulators run the callback before
non-blocking assignment events have been processed.

Consequently, I had to look for a workaround. One half of the solution
is to use the \code{cbReadOnlySync} callback.  This callback runs
after all pending events have been processed.  However, it does not
permit to create new events in the current time step.  The second half
of the solution is to map \myhdl\ delta cycles onto Verilog time steps.
Note that there is some freedom here because of the restriction that
only passive HDL code can be cosimulated.

I chose to make the time granularity in the Verilog simulator a 1000
times finer than in the \myhdl{} simulator. For each \myhdl\ time step,
1000 Verilog time steps are available for  \myhdl\ delta cycles. In practice,
only a few delta cycles per time step should be needed. More than 1000
almost certainly indicates an error. This limit is checked at
run-time. The factor 1000 also makes it easy to distinguish ``real''
time from delta cycle time when printing out the Verilog time.

\subsection{Other Verilog simulators}

The Icarus module is written with VPI calls, which are provided by the
most recent generation of the Verilog PLI. Some simulators may only
support TF/ACC calls, requiring a complete redesign of the interface
module.

If the simulator supports VPI, the Icarus module should be reusable to
a large extent. However, it may be possible to improve on it.  The
workaround described in the previous section may not be necessary. In
some simulators, the \code{cbReadWriteSync} callback occurs after all
events (including non-blocking assignments) have been processed. In
that case, the functionality can be supported without a finer time
granularity in the Verilog simulator.

There are also Verilog standardization efforts underway to resolve the
ambiguity of the \code{cbReadWriteSync} callback. The solution will be
to introduce new, well defined callbacks. From reading some proposals,
I conclude that the \code{cbEndOfSimTime} callback would provide the
required functionality.

\subsection{VHDL}

It would be great to have an interface to the Modelsim VHDL
simulator. This will require a redesign from scratch with the
appropriate PLI.  One feature which I would like to keep if possible
is the way to declare the communicating signals.  In the Verilog
solution, it is not necessary to define and instantiate any special
entity (module). Rather, the participating signals can be declared
directly in the \code{to_myhdl} and \code{from_myhdl} task calls.

\chapter{Reference manual}


\myhdl\ is implemented as a Python package called \code{myhdl}. This
chapter describes the objects that are exported by this package.

\section{The \class{Simulation} class}
\begin{classdesc}{Simulation}{arg \optional{, arg \moreargs}}
Class to construct a new simulation. Each argument is either be a
\myhdl\ generator, or a nested sequence of such generators. (A nested
sequence is defined as a sequence in which each item may itself be a
nested sequence.) See section~\ref{myhdl-generators} for the
definition of \myhdl\ generators and their interaction with a
\class{Simulation} object.
\end{classdesc}

A \class{Simulation} object has the following method:

\begin{methoddesc}[Simulation]{run}{\optional{duration}}
Run the simulation forever (by default) or for a specified duration.
\end{methoddesc}

\section{The \class{Signal} class}
\label{signal}
\begin{classdesc}{Signal}{val \optional{, delay}}
This class is used to construct a new signal and to initialize its
value to \var{val}. Optionally, a delay can be specified.
\end{classdesc}

A \class{Signal} object has the following attributes:

\begin{memberdesc}[Signal]{next}
Read-write attribute that represents the next value of the signal.
\end{memberdesc}

\begin{memberdesc}[Signal]{val}
Read-only attribute that represents the current value of the signal.

This attribute is always available to access the current value;
however in many practical case it will not be needed. Whenever there
is no ambiguity, the Signal object's current value is used
implicitly. In particular, all Python's standard numeric, bit-wise,
logical and comparison operators are implemented on a Signal object by
delegating to its current value. The exception is augmented
assignment. These operators are not implemented as they would break
the rule that the current value should be a read-only attribute. In
addition, when a Signal object is directly assigned to the \code{next}
attribute of another Signal object, its current value is assigned
instead.
\end{memberdesc}



\section{\myhdl\ generators and trigger objects}
\label{myhdl-generators}
\myhdl\ generators are standard Python generators with specialized
\keyword{yield} statements. In hardware description languages, the equivalent
statements are called \emph{sensitivity lists}. The general format
of \keyword{yield} statements in in \myhdl\ generators is:

\hspace{\leftmargin}\keyword{yield} \var{clause \optional{, clause ...}}

After a simulation object executes a \keyword{yield} statement, it
suspends execution of the generator. At the same time, each
\var{clause} is a \emph{trigger object} which defines the condition
upon which the generator should be resumed. However, per invocation of a
\keyword{yield} statement, the generator is resumed exactly once,
regardless of the number of clauses. This happens as soon as one
of the objects triggers; subsequent triggers are
neglected. (However, as a result of the resumption, it is possible
that the same \keyword{yield} statement is invoked again, and that a
subsequent trigger still triggers the generator.)

In this section, the trigger objects and their functionality will be
described. 

\begin{funcdesc}{posedge}{signal}
Return a trigger object that specifies that the generator should
resume on a rising edge on the signal. A rising edge means a change
from false to true.
\end{funcdesc}

\begin{funcdesc}{negedge}{signal}
Return a trigger object that specifies that the generator should
resume on a falling edge on the signal. A falling edge means a change
from true to false.
\end{funcdesc}

\begin{funcdesc}{delay}{t}
Return a trigger object that specifies that the generator should
resume after a delay \var{t}.
\end{funcdesc}

\begin{funcdesc}{join}{arg \optional{, arg \moreargs}}
Join a number of trigger objects together and return a joined
trigger object.  The effect is that the joined trigger object will
trigger when \emph{all} of its arguments have triggered.
\end{funcdesc}

In addition, some objects can directly function as trigger
objects. These are the objects of the following types:

\begin{datadesc}{Signal}
For the full description of the \class{Signal} class, see
section~\ref{signal}.

A signal is a trigger object. Whenever a signal changes value, the
generator is triggered.
\end{datadesc}

\begin{datadesc}{GeneratorType}
\myhdl\ generators can itself be used as trigger objects. 
This corresponds to spawning a new generator, while the original
generator waits for it to complete.  In other words, the original
generator is triggered when the spawned generator completes.
\end{datadesc}

In addition, as a special case, the Python \code{None} object can be
present in  a \code{yield} statement:

\begin{datadesc}{None}
This is the do-nothing trigger object. The generator immediately
resumes, as if no \code{yield} statement were present. This can be
useful if the \code{yield} statement also has generator clauses: those
generators are spawned, while the original generator resumes
immediately.

\end{datadesc}



\section{Miscellaneous objects}

The following objects can be convenient in \myhdl\ modeling.

\begin{excclassdesc}{StopSimulation}{}
Base exception that is caught by the \code{Simulation.run} method to
stop a simulation. Can be subclassed and raised in generator code.
\end{excclassdesc}

\begin{funcdesc}{now}{}
Return the current simulation time.
\end{funcdesc}

\begin{funcdesc}{downrange}{high \optional{, low=0}}
Generates a downward range list of integers. Modeled after the
standard \code{range} function, but works in the downward
direction. The returned interval is half-open, with the \var{high}
index not included. \var{low} is optional and defaults to zero.

This function is especially useful with the \class{intbv} class, that
also works with downward indexing.
\end{funcdesc}

\begin{funcdesc}{bin}{num \optional{, width}}
Return a representation as a bit string.  If \var{width} is provided,
and if it is larger than the width of the default representation, the
bit string is padded with the sign bit.

This function complements the standard Python conversion functions
\code{hex} and \code{oct}. A binary string representation is often
needed in hardware design.
\end{funcdesc}


\section{The \class{intbv} class}

\begin{classdesc}{intbv}{arg}
This class represents \class{int}-like objects with some additional
features that make it suitable for hardware design. The constructor
argument can be an \class{int}, a \class{long}, an \class{intbv} or a
bit string (a string with only '0's or '1's). For a bit string
argument, the value is calculated as \code{int(\var{bitstring}, 2)}. 
\end{classdesc}

Unlike \class{int} objects, \class{intbv} objects are mutable; this is
also the reason for their existence. Mutability is needed to support
assignment to indexes and slices, as is common in hardware design. For
the same reason, \class{intbv} is not a subclass from \class{int},
even though \class{int} provides most of the desired
functionality. (It is not possible to derive a mutable subtype from
an immutable base type.)

An \class{intbv} object supports the same comparison, numeric,
bitwise, logical, and conversion operations as \class{int} objects. See
\url{http://www.python.org/doc/current/lib/typesnumeric.html} for more
information on such operations. In all binary operations,
\class{intbv} objects can work together with \class{int} objects; in
those cases the return type is an \class{intbv} object.

In addition, \class{intbv} objects support indexing and slicing
operations:

\begin{tableiii}{clc}{code}{Operation}{Result}{Notes}
  \lineiii{\var{bv}[\var{i}]}
	  {item \var{i} of \var{bv}}
	  {(1)}
  \lineiii{\var{bv}[\var{i}] = \var{x}}  
	  {item \var{i} of \var{bv} is replaced by \var{x}} 
          {(1)}
  \lineiii{\var{bv}[\var{i}:\var{j}]} 
          {slice of \var{bv} from \var{i} downto \var{j}} 
          {(2)(3)}
  \lineiii{\var{bv}[\var{i}:\var{j}] = \var{t}} 
  	  {slice of \var{bv} from \var{i} downto \var{j} is replaced
          by \var{t}} 
          {(2)(4)}
\end{tableiii}

\begin{description}
\item[(1)] Indexing follows the most common hardware design
	  conventions: the lsb bit is the rightmost bit, and it has
	  index 0. This has the following desirable property: if the
	  \class{intbv} value is decomposed as a sum of powers of 2,
	  the bit with index \var{i} corresponds to the term
	  \code{2**i}.

\item[(2)] It follows from the indexing convention that slicing ranges
	  are downward, in contrast to standard Python. However, the
	  Python convention of half-open ranges is followed. In
	  accordance with standard Python, the high index is not
	  included. However, it is the \emph{leftmost} index in this
	  case. As in standard Python, this takes care of one-off
	  issues in many practical cases: in particular,
	  \code{bv[\var{i}:]} returns \var{i} bits;
	  \code{bv[\var{i}:\var{j}]} has \code{\var{i}-\var{j}}
	  bits. As \class{intbv} objects have no explicitly defined
	  bit width, the high index \var{j} has no default value and
	  cannot be omitted, while the low index \var{j} defaults to
	  \code{0}.

\item[(3)] The value returned from a slicing operation is always
	  positive; higher order bits are implicitly assumed to be
	  zero. The bit width is implicitly stored in the returned bit
	  width, so that the returned object can be used in
	  concatenations and as an iterator.

\item[(4)] In setting a slice, it is checked whether the slice is wide
	  enough to accept all significant bits of the value.
\end{description}

In addition, \class{intbv} objects support a concatenation method:

\begin{methoddesc}[intbv]{concat}{\optional{arg \moreargs}}
Concatenate the arguments to an \class{intbv} object. Naturally, the
concatenation arguments need to have a defined bit width. Therefore,
if they are \class{intbv} objects, they have to be the return values
of a slicing operation. Alternatively, they may be bit strings.

In contrast to all other arguments, the implicit \var{self} argument
doesn't need to have a defined bit with. This is due to the fact that
concatenation occurs at the lsb (rightmost) side.

It may be clearer to call this method as an unbound method with an
explicit first \class{intbv} argument.
\end{methoddesc}

In addition, an \class{intbv} object supports the iterator protocol. This
makes it possible to iterate over all its bits, from the high index to
index 0. This is only possible for \class{intbv} objects with a
defined bit width.













\end{document}
