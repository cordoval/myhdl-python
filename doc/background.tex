\chapter{Background information}

\section{Prerequisites}

You need a basic understanding of Python to use \myhdl{}.
If you don't know Python, you will take comfort in knowing
that it is probably one of the easiest programming languages to
learn~\footnote{You must be bored by such claims, but in Python's
case it's true.}. Learning Python is also one of the better time
investments that engineering professionals can make~\footnote{I am not
biased.}.

For beginners, \url{http://www.python.org/doc/current/tut/tut.html} is
probably the best choice for an on-line tutorial. For alternatives,
see \url{http://www.python.org/doc/Newbies.html}.

A working knowledge of a hardware description language such as Verilog
or VHDL is helpful. Chances are that you know one of those anyway, if
you are interested in \myhdl{}.

\section{A small tutorial on generators}

Generators are a recent feature in Python. They were introduced in
Python 2.2, which is the most recent stable version at the time of
this writing. Therefore, there isn't a lot of tutorial material
available yet. Because generators are the key concept in
\myhdl{}, I include a small tutorial here.

Consider the following nonsensical function:

\begin{verbatim}
def function():
    for i in range(5):
        return i

\end{verbatim}

You can see why it doesn't make a lot of sense. As soon as the first
loop iteration is entered, the function returns:

\begin{verbatim}
>>> function()
0
\end{verbatim}

Returning is fatal for the function call. Further loop iterations
never get a chance, and nothing is left over from the function call
when it returns.

To change the function into a generator function, we replace
\keyword{return} with \keyword{yield}:

\begin{verbatim}
def generator():
    for i in range(5):
        yield i

\end{verbatim}

Now we get:

\begin{verbatim}
>>> generator()
<generator object at 0x815d5a8>

\end{verbatim}

When a generator function is called, it returns a generator object. A
generator object supports the iterator protocol, which is an expensive
way of saying that you can let it generate subsequent values by
calling its \function{next()} method:

\begin{verbatim}
>>> g = generator()
>>> g.next()
0
>>> g.next()
1
>>> g.next()
2
>>> g.next()
3
>>> g.next()
4
>>> g.next()
Traceback (most recent call last):
  File "<stdin>", line 1, in ?
StopIteration

\end{verbatim}

Now we can generate the subsequent values from the for
loop on demand, until they are exhausted. What happens is that the
\keyword{yield} statement is like a
\keyword{return}, except that it is non-fatal: the generator remembers
its state and the point in the code when it yielded. A higher order
agent can decide when to get a further value by calling the
generator's \function{next()} method. We say that generators are
\dfn{resumable functions}.

If you are familiar with hardware description languages, this may
ring a bell. In hardware simulations, there is also a higher order
agent, the Simulator tool, that interacts with such resumable
functions; they are called processes in VHDL and always blocks in
Verilog. Like in those languages, Python generators provide an elegant
and efficient method to model concurrency, without having to resort to
some form of threading.

The use of generators to model concurrency is the first key concept in
\myhdl{}. The second key concept is a related one: in \myhdl{}, the
yielded values are used to define the condition upon which the generator
should be resumed. In other words, the \keyword{yield} statements work
as generalized sensitivity lists. If by now you are still interested,
read on to learn more!

If you want to know more about generators, consult the on-line Python
documentation, e.g. at \url{http://www.python.org/doc/2.2.2/whatsnew}. 

\begin{notice}[warning]
At the beginning of this section I said that generators were
introduced in Python 2.2. This is not entirely correct: in fact,
generators will only be enabled as a standard feature in Python 2.3.
However, a stable version of Python 2.3 has not been released yet at
the time of this writing. So, what to do?

Fortunately, Python lets you import features from its future releases
(provided that the future is not too distant). So, until you use
Python 2.3 or higher, you have to include the following line at the
start of all code that defines generator functions:

\begin{verbatim}
from __future__ import generators

\end{verbatim}

From Python 2.3 on, this line may still be in the code, though it
will not have an effect anymore.
\end{notice}

